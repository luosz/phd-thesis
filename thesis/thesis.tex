%%%%%%%%%%%%%%%%%%%%%%%%%%%%%%%%%%%%%%%%%%%%%%%%%%%%%%%%%%%%%%%%%%%%%%%%%%%%%
%%%
%%% File: utthesis2.doc, version 2.0jab, February 2002
%%%
%%% Based on: utthesis.doc, version 2.0, January 1995
%%% =============================================
%%% Copyright (c) 1995 by Dinesh Das.  All rights reserved.
%%% This file is free and can be modified or distributed as long as
%%% you meet the following conditions:
%%%
%%% (1) This copyright notice is kept intact on all modified copies.
%%% (2) If you modify this file, you MUST NOT use the original file name.
%%%
%%% This file contains a template that can be used with the package
%%% utthesis.sty and LaTeX2e to produce a thesis that meets the requirements
%%% of the Graduate School of The University of Texas at Austin.
%%%
%%% All of the commands defined by utthesis.sty have default values (see
%%% the file utthesis.sty for these values).  Thus, theoretically, you
%%% don't need to define values for any of them; you can run this file
%%% through LaTeX2e and produce an acceptable thesis, without any text.
%%% However, you probably want to set at least some of the macros (like
%%% \thesisauthor).  In that case, replace "..." with appropriate values,
%%% and uncomment the line (by removing the leading %'s).
%%%
%%%%%%%%%%%%%%%%%%%%%%%%%%%%%%%%%%%%%%%%%%%%%%%%%%%%%%%%%%%%%%%%%%%%%%%%%%%%%

\documentclass[a4paper, 12pt, oneside]{report}         %% LaTeX2e document.
\usepackage {tcdthesis}              %% Preamble.

%%For loading graphic files
\usepackage{graphicx}
%% subfigure
\usepackage{caption}
\usepackage{subcaption}
%% bookmark
\usepackage{hyperref}
%% url
\usepackage{url}
\usepackage{textgreek}
%% image path
\graphicspath{{../img/CGVC2015/}{../img/EuroVis2015/}{../img/TransferReport/}{../img/EurasiaGraphics2014/images/}{../img/EurasiaGraphics2014/images1/}{../img/EurasiaGraphics2014/images2/}{../img/EurasiaGraphics2014/images3/}{../img/EurasiaGraphics2014/images4/}}
\usepackage{lmodern} %Type1-font for non-english texts and characters

%% Math Packages %%%%%%%%%%%%%%%%%%%%%%%%%%%%%%%%%%%%%%%%%%%%
\usepackage{amsmath}
\usepackage{amsthm}
\usepackage{amsfonts}

% other packages
\usepackage{cite}
\usepackage{braket}

% \mastersthesis                     %% Uncomment one of these; if you don't
\phdthesis                         %% use either, the default is \phdthesis.

%\thesisdraft                       %% Uncomment this if you want a draft
                                     %% version; this will print a timestamp
                                     %% on each page of your thesis.

\leftchapter                       %% Uncomment one of these if you want
%\centerchapter                      %% left-justified, centered or
% \rightchapter                      %% right-justified chapter headings.
                                     %% Chapter headings includes the
                                     %% Contents, Acknowledgments, Lists
                                     %% of Tables and Figures and the Vita.
                                     %% The default is \centerchapter.

% \singlespace                       %% Uncomment one of these if you want
\oneandhalfspace                   %% single-spacing, space-and-a-half
% \doublespace                       %% or double-spacing; the default is
                                     %% \oneandhalfspace, which is the
                                     %% minimum spacing accepted by the
                                     %% Graduate School.

\renewcommand{\thesisauthor}{Shengzhou Luo}            %% Your official UT name.
\renewcommand{\thesismonth}{August}                  %% Your month of graduation.
\renewcommand{\thesisyear}{2015}                      %% Your year of graduation.
\renewcommand{\thesistitle}{Information-Guided Transfer Functions and Selective Enhancements for Volume Visualization}            %% The title of your thesis; use mixed-case.
\renewcommand{\thesisauthorpreviousdegrees}{MSc}  %% Your previous degrees, abbreviated; separate multiple degrees by commas.
\renewcommand{\thesissupervisor}{John Dingliana}      %% Your thesis supervisor; use mixed-case and don't use any titles or degrees.
% \renewcommand{\thesiscosupervisor}{}                %% Your PhD. thesis co-supervisor; if any.

% \renewcommand{\thesiscommitteemembera}{}
% \renewcommand{\thesiscommitteememberb}{}
% \renewcommand{\thesiscommitteememberc}{}
% \renewcommand{\thesiscommitteememberd}{}
% \renewcommand{\thesiscommitteemembere}{}
% \renewcommand{\thesiscommitteememberf}{}
% \renewcommand{\thesiscommitteememberg}{}
% \renewcommand{\thesiscommitteememberh}{}
% \renewcommand{\thesiscommitteememberi}{}

\renewcommand{\thesisauthoraddress}{Dublin, Ireland}

\renewcommand{\thesisdedication}{}     %% Your dedication, if you have one; use "\\" for linebreaks.


%%%%%%%%%%%%%%%%%%%%%%%%%%%%%%%%%%%%%%%%%%%%%%%%%%%%%%%%%%%%%%%%%%%%%%%%%%%%%
%%%
%%% The following commands are all optional, but useful if your requirements
%%% are different from the default values in utthesis.sty.  To use them,
%%% simply uncomment (remove the leading %) the line(s).

% \renewcommand{\thesiscommitteesize}{...}
                                     %% Uncomment this only if your thesis
                                     %% committee does NOT have 5 members
                                     %% for \phdthesis or 2 for \mastersthesis.
                                     %% Replace the "..." with the correct
                                     %% number of members.

% \renewcommand{\thesisdegree}{...}  %% Uncomment this only if your thesis
                                     %% degree is NOT "DOCTOR OF PHILOSOPHY"
                                     %% for \phdthesis or "MASTER OF ARTS"
                                     %% for \mastersthesis.  Provide the
                                     %% correct FULL OFFICIAL name of
                                     %% the degree.

% \renewcommand{\thesisdegreeabbreviation}{...}
                                     %% Use this if you also use the above
                                     %% command; provide the OFFICIAL
                                     %% abbreviation of your thesis degree.

% \renewcommand{\thesistype}{...}    %% Use this ONLY if your thesis type
                                     %% is NOT "Dissertation" for \phdthesis
                                     %% or "Thesis" for \mastersthesis.
                                     %% Provide the OFFICIAL type of the
                                     %% thesis; use mixed-case.

% \renewcommand{\thesistypist}{...}  %% Use this to specify the name of
                                     %% the thesis typist if it is anything
                                     %% other than "the author".

%%%
%%%%%%%%%%%%%%%%%%%%%%%%%%%%%%%%%%%%%%%%%%%%%%%%%%%%%%%%%%%%%%%%%%%%%%%%%%%%%



\begin{document}                                  %% BEGIN THE DOCUMENT

%\thesistitlepage                                  %% Generate the title page.

%\thesisdeclarationpage				  %% Generate the declaration page.

%\thesispermissionpage				  %% Generate the copyright permission page

%\thesisdedicationpage                             %% Generate the dedication page.

%\begin{thesisacknowledgments}                     %% Use this to write your
%...ACKNOWLEDGMENTS...                          %% acknowledgments; it can be anything
%\end{thesisacknowledgments}                       %% allowed in LaTeX2e par-mode.

\begin{thesisabstract}
Volume data is widely used in scientific and medical research, and volume visualization has been proven to be an effective and flexible method for visualizing complex structures. This thesis examines the methods for exploring of volume data by optimization of visualization parameters and through the use of focus and context visualization techniques by selectively enhancing important parts of the data sets.


%Volume data is widely used in scientific and medical research, and volume visualization has been proven to be an effective and flexible method for visualizing complex structures within volume data.
%In recent years, volume visualization has received increasing attention in the analysis of dynamics and evolution of phenomena in a variety of application domains, including medicine, meteorology, astrophysics and engineering.
%However, the size and complexity of the parameter space controlling the rendering process makes it challenging to generate an informative rendering.
%In particular, the specification of the transfer function (which is a mapping from data values to visual properties ) is frequently a time-consuming and unintuitive task.

%We propose a novel approach to optimise the transfer functions in volume visualization by exploiting the information inherent within the volume data.
%We hypothesise that the importance of voxels (sample values in volume data) are associated with their information content. Therefore, the transfer functions of volume visualization can be optimized based on the information within the data sets. The user's interests are also taken into account in this approach through interactive input (such as user-selected regions).
%This optimization approach reduces the occlusion in the resulting images, and thus improves the perception of structures in the rendered images.
%In particular we believe our approach will be useful in visualizing time-varying details by adaptively optimizing the visualization parameters according to the changes over time.
%
%NPR techniques are effective forms of abstraction and they have proven useful in expressing features that are difficult to display using realistic depiction of scenes and objects.
%We explore the use of NPR techniques to enhance the perception of structures within the volume data by highlighting important details and simplifying less important details, as well as depicting the dynamic aspects of time-varying volume data.
%We argue that the combination of standard volume visualization and NPR techniques can provide opportunities to deliver meaningful visualization and assist users in accomplishing their underlying problems efficiently.
%



Volume visualization is a powerful technique for depicting layered structures in 3D volume data sets. However, it is a major challenge to obtain clear visualizations of a volume with layers clearly revealed.
In particular, the specification of the transfer function is frequently a time-consuming and unintuitive task in volume rendering.
We describe a global optimization and two user-driven refinement methods for modulating transfer functions in order to assist the exploration of volume data.
This optimization is dependent on the distribution of scalar values of the volume data set and is designed to reduce general occlusion and improve the clarity of layers of structures in the resulting images.
The user can explore a volume by interactively specifying different priority intensity ranges and observe which layers of structures are revealed. In addition we show how the technique can be applied for time-varying volume data sets by adaptively refining the transfer function based on the histogram of each time-step. 
Experimental results on various data sets are presented to demonstrate the effectiveness of our method.


Volume visualization has been widely used to depict complicated 3D structures in volume data sets.
However, obtaining clear visualization of the features of interest in a volume is still a major challenge.
The clarity of features depends on the transfer function, the viewpoint and the spatial distribution of features in the volume data set.
We propose visibility-weighted saliency as a measure of visual saliency of features in volume rendered images, in order to assist users in choosing suitable viewpoints and designing effective transfer functions to visualize the features of interest. Visibility-weighted saliency is based on a computational measure of perceptual importance of voxels and the visibility of features in volume rendered images.
The effectiveness of this scheme is demonstrated by test results on two volume data sets.


Time-varying volume data is used in many areas of science and engineering. However visualizations of such data are not easy for users to visually process due to the amount of information that can be presented simultaneously. We propose a novel visualization approach which modulates focus, emphasizing important information, by adjusting saturation and brightness of voxels based on an importance measure derived from temporal and multivariate information. By conducting a voxel-wise analysis of a number of consecutive frames, we acquire a volatility measure of each voxel. We then use intensity, volatility and additional multivariate information to determine opacity, saturation and brightness of the voxels. The method was tested in visualizing a multivariate hurricane data set. The results suggest that our approach can give the user a deeper understanding of the data by presenting multivariate information variables in one self-contained visualization.
\end{thesisabstract}

\tableofcontents                                  %% Generate table of contents.
%\listoftables                                     %% Uncomment this to generate list of tables.
%\listoffigures                                    %% Uncomment this to generate list of figures.

%%
%% Include thesis chapters here...
%%
\chapter{Introduction \label{section_introduction}}
%\epigraph{``What is above form is called Tao; what is within form is called tool."}{--- \textup{I Ching (the Classic of Changes)}}

%Volume visualization is an active branch of scientific visualization. It is a method of extracting meaningful information from volume data (3D discretely sampled data sets) using interactive graphics and imaging. The study of volume visualization involves volume data representation, modeling, manipulation and rendering \cite{kaufman_volume_1997}.
%First introduced by Levoy \cite{levoy_display_1988}, volume visualization has been widely used in various sciences to create insightful visualizations from both simulated and measured data.
%Volume visualization is a powerful technique which aims to visualize the 3D structures in volume data sets and thus facilitates the user's exploration into the data. It has become an important technique for various applications such as medical imaging and scientific visualization.
%%Furthermore, it is especially useful in diagnostics for physicians in medicine.
%Recent advances in volume data acquisition and scientific simulations have led to dramatically increasing volume data sets, both in size and complexity, that must be visualized and analyzed \cite{beyer_state---art_2015}.

Volume visualization is an active branch of scientific visualization concerned with extracting meaningful information from volume data (3D discretely sampled data sets) using interactive graphics and imaging. The study of volume visualization involves volume data representation, modeling, manipulation and rendering \cite{kaufman_volume_1997} and it aims, in particular, to facilitate visual exploration of 3D structures allowing users to more deeply understand and analyze volume data sets. 
First introduced by Levoy \cite{levoy_display_1988} in 1988, volume visualization has been widely used in various sciences to create insightful visualizations from both simulated and measured data. However, recent advances in volume data acquisition and scientific simulations have led to dramatic increases in both size and complexity of data sets, which present new and ongoing challenges to be addressed \cite{beyer_state---art_2015}.

%Time-varying volume data sets are increasing dramatically both in size and complexity as various data acquisition devices rapidly evolve in recent years.
%However, the visualization of time-varying volume data remains a challenging problem due to the large size and the dynamic nature of the underlying information.
%Previously established techniques, such as in flow visualization, struggle to deal with the increasing complexity of the most recent data sets.

The rendering of volume data requires every sample value (also called voxel, which is a volume element or volumetric pixel) to be mapped to visual properties (e.g. opacity and color). This mapping is done with a transfer function, which can be a simple ramp, a piecewise linear function or an arbitrary table.
The design of an effective transfer function (see Section~\ref{literature_of_transfer_function} for details) is essential for visualizing volume data.
%A wealth of techniques have been developed for transfer function design for static volume data \cite{pfister_transfer_2001} \cite{bernardon_transfer-function_2008} \cite{arens_survey_2010}.

With volume rendering, both the exterior and interior of a volume data set can be revealed semi-transparently by specifying appropriate transfer functions.
However, because of 3D occlusion between structures and the indirect control over the final visualization, it is time-consuming and unintuitive for users to specify appropriate transfer functions.
%In practice, transfer functions are often specified in a trial and error manner with careful observation of changes in the resulting visualization \cite{kniss_interactive_2001}.
In practice, this is typically achieved using a trial-and-error approach: modifications are made to the transfer function and changes in the resulting visualization are carefully observed in order to inform further modifications to the transfer function \cite{kniss_interactive_2001}.
The adjustments users make in transfer function specification are based on subjective perception of important features in a certain viewpoint.

%By speci-
%fying appropriate opacities for extracted features, the ex-
%terior and interior features can be simultaneously revealed
%in a semi-transparent manner. 
% 
%However, in practice it is
%rather difficult and time-demanding to specify appropriate
%opacities, due to 3D occlusion between features. The main
%reason is that interaction in the transfer function domain
%is mainly guided by careful observation of changes in the
%rendered image [KKH01]. Decisions based on this kind
%of adjustment are subjective and view-dependent.
%
%there is no quantitative metric to measure the influ-
%ence of each feature to the rendered image. While most
%of the previous research focused on how to extract fea-
%tures 
%
%very little attention has
%been paid to quantitative analysis of the visibility of classi-
%fied features in the rendered image.
 


%However, transfer function design for time-varying volume data has not been studied thoroughly.
%A fundamental challenge in the analysis and classification of time-varying volume data is the lack of capability to track data change or evolution over time \cite{gu_transgraph_2011}.
%Much of the work in the field of volume visualization has been focused on the synthesis of photorealistic images to assist in the visualization of structures contained in volume data sets.
%However, traditional depictions of the same types of data, such as those found in medical textbooks, deliberately use non-realistic techniques to draw the viewer's attention to important aspects \cite{bruckner_style_2007}. Using abstraction, visual overload is prevented and thus result in a more effective visualization.
%NPR techniques are effective forms of abstraction. They are commonly inspired by artistic styles and techniques that do not focus on a realistic depiction of scenes and objects, therefore, they can express features that cannot be shown using physically correct light transport. Non-photorealistic images are used in preference to photorealistic images in specific circumstances. For instance, an empirical study \cite{schumann_assessing_1996} reported that, when asked to compare a computer-produced sketch against a photorealistically rendered CAD image, architects showed a great preference for the sketch.
%NPR models were adopted in visualization and hence formed the field of illustrative visualization.
%Although illustrative visualization is a relatively novel category of visualization approaches, it has been successfully employed in medical and other visualization sub-fields \cite{svakhine_illustration_2005} \cite{svakhine_illustration-inspired_2009}.
%Illustrative visualization has proven its usefulness in revealing 3D structures due to its ability to hide less relevant details while emphasizing important details. The goal of illustrative visualization is to gain clarity compared to photo-realistic rendering by emphasizing important features and improving data exploration. In order to obtain more comprehensive images, it is necessary to highlight important aspects and omit less relevant details.

\section{Motivation \label{motivation}}
%Understanding and analyzing complex volumetrically varying data is a challenging problem. Many visualization techniques have had only limited success in succinctly portraying the structure of volume data sets.
%There are only limited success have been reported about various visualization techniques which succinctly portray the structure of 3D time-varying volume data.

%The main goal of our research is to investigate the optimization of visualization parameters (in particular transfer functions) and the use of NPR techniques, and develop a methodology which incorporates these two types of techniques to facilitate the user's exploration of the data sets. NPR techniques are effective forms of abstraction and they have proven their usefulness in expressing features that cannot be shown using realistic depiction of scenes and objects. The combination of standard volume visualization and NPR techniques will bring the opportunity to provide expressive visualization and assist the user in accomplishing his/her task efficiently.

Objective measures such as voxel information \cite{bordoloi_view_2005}, visibility histograms \cite{emsenhuber_visibility_2008} \cite{correa_visibility-driven_2009}, feature visiblity \cite{wang_information_2011}, and visual saliency models \cite{emami_selection_2013} such as saliency maps \cite{itti_model_1998} \cite{harel_graph-based_2006} for 2D images and saliency fields \cite{kim_saliency-guided_2006} for volumetric data, provide the basis for powerful feedback mechanisms in volume rendering.
In current volume rendering systems, appropriate transfer functions are often obtained by trial-and-error \cite{pfister_transfer_2001}.
It is desirable to take advantage of these objective measures in order to automate the specification of transfer functions for emphasizing features of interest in volume visualization.

The main goal of our research is to investigate the optimization of visualization parameters (in particular transfer functions) with information derived from volume data based on feedback mechanisms from the volume rendering process.
We hypothesize that the importance of voxels (sample values in volume data) are associated with their information content. Therefore, the transfer functions of volume visualization can be optimized based on the information inherent within the data sets and user input which indicates the user's interest.
Furthermore, we hypothesize that combining automated optimization techniques with feedback mechanisms such as visibility and visual saliency can provide a more intuitive means for obtaining clear visualization of features of interest in volume data.

% (e.g. visibility and visual saliency in resulting images) 

%feasibility
%In all of the above, the issue of visibility is more a
%consequence of transfer function design than a design
%parameter. 

%In addition, we investigate the feasibility of propagating the optimization approach from static volume data to time-varying data.

%A small number of approaches have been proposed for using information theory in volume visualization. However, most available approaches are not designed for the visualization of time-varying data and thus they have not taken the coherence issues into account.

\section{Scope}
%This thesis focuses on methods for enhancing user understanding of volume data by optimizing visualization parameters and incorporating NPR techniques.
%I investigate the optimization of transfer functions by exploiting the information inherent within the volume data and input from user interaction.
%In addition, we investigate NPR techniques which we believe are well suited for highlighting important details as well as simplifying less important details. Hence, we combine the optimization of visualization parameters and NPR techniques in order to provide meaningful exploration of complex data.

The focus of this thesis is on methods for enhancing user understanding of features of interest in volume visualization by optimizing transfer functions based on information derived from volume data (e.g. entropy and saliency of voxels). In addition to the information inherent in the volume data, view-dependent information (e.g. visibility of voxels) obtained in the volume rendering process is also exploited in the optimization of transfer functions.

In this research, we focus on the visualization of volume data sets, particularly the scalar field data acquired from medical imaging (e.g. CT and MRI scans) and generated from flow simulations (e.g. computational fluid dynamics).

The features of interest in a volume data set are specified by initial user-defined transfer functions.
Therefore, manual segmentation by domain experts or computationally expensive automatic segmentation techniques are not in the scope of this thesis.
Moreover, this thesis focuses on direct volume rendering techniques. Indirect volume rendering techniques, which require the reconstruction of 3D surfaces, are not in scope of this thesis.

%Flow data which are often in the form of vector fields are not under the focus of our research.
%Furthermore, we investigate techniques which are applicable to consumer level devices rather than expensive dedicated visualization hardware. As such, this constrains the amount of memory and processing power available, and therefore fast techniques with low memory requirement are necessary.

%\section{Research Question and Design \label{section_research_question}}
%\paragraph{Research Question}
%Visualization is concerned with the creation of images from data to enhance the user's ability to reason and understand properties related to the user's underlying problem.
%This research project focuses on methods to optimize important parameters of volume visualization (in particular transfer functions) and utilize NPR techniques in order to facilitate the user's exploration and understanding of the volume data. We address the following problem in volume visualization.
%
%\begin{itemize}
%\item How do we improve the understanding of volume data by optimization of visualization parameters and through the use of NPR techniques?
%	\begin{itemize}
%	\item We investigate techniques that will allow us to exploit the information inherent within the volume data to automatically and also semi-automatically (in user driven ways) optimise the transfer function in volume visualization. 
%%	Additionally we explore illustrative rendering strategies to enhance the perception of structures within the volume data.
%	
%	Additionally we explore illustrative rendering strategies to enhance the perception of structures within the volume data as well as depict the dynamic aspects of time-varying volume data.
%%	How do we exploit the information in the volume data to optimize the transfer function in volume visualization and enhance the perception of structures using illustrations?
%	\end{itemize}
%%\item How do we convey motion information of time-varying volume data through non-photorealistic techniques?
%%	\begin{itemize}
%%	\item How do we exploit illustrations and painterly stylization (such as stroke orientation, size and color) to express motion information (such as direction and speed)?
%%	\end{itemize}
%\end{itemize}
%
%%\paragraph{Sub-questions stem from the overall research questions}
%%
%%\begin{itemize}
%%\item How de we identify features of interest in volume data?
%%\item How do we extract a succinct representation from a time-step of time-varying volume data?
%%\\How do we design coherent transfer functions for time-varying volume rendering?
%%\item How do we track features through a series of time-steps in time-varying volume data?
%%\\How does clustering techniques help gain insights into time-varying volume data?
%%\item How do we enhance time-invariant features of volume data with illustrative techniques such as boundary enhancement and oriented feature enhancement?
%%\item How do we enhance temporal features of time-varying volume data with illustration-inspired techniques?
%%\item How do we render time-varying time-varying volume data with coherence?
%%\\How do we tackle the coherence issues in transfer function design and illustrative rendering for time-varying volume data?
%%\item How do we evaluate the effectiveness of our approaches?
%%\end{itemize}
%
%\paragraph{Research Design}
%In this thesis, we use entropy from information theory to measure the information content associated with sample values in volume data and establish importance measurements based on information within the volume data as well as input from user-driven techniques.
%
%NPR techniques are adopted in our approaches to visualize both static and time-varying data in order to enhance the expressiveness of the visualization. The combination of standard volume rendering techniques and NPR techniques provides us with the freedom to depict data content as abstraction according to its importance.
%
%The most fundamental objective of visualization is to enhance the user's ability to reason and understand properties related to the underlying problem. Therefore, we will conduct user studies that will measure the user's task performance and accuracy in order to evaluate the effectiveness of the proposed visualization approaches.

%\section{Methodology}
%\section{Scope}

\section{Contributions}
We present a transfer function refinement approach, which exploits the entropy of voxels derived from volume to equalize the opacity transfer function, in order to reduce general occlusion and improve the clarity of layers of structures in the resulting images.
%Moreover, the user can explore a volume by interactively specifying different priority intensity ranges and observe which layers of structures are revealed.
Moreover, this approach assists the user in exploring and enhancing features of interest by interactively specifying different priority intensity ranges.

In addition to view-independent information (i.e. entropy of voxels), we propose visibility-weighted saliency for measuring the view-dependent saliency of features of interest for volume visualization.
This metric aims to assist users in choosing suitable viewpoints and designing effective transfer functions to visualize the features of interest.
(The formal definition of a feature is provided in Section~\ref{feature_definition}.)

Subsequently, we describe an automated transfer function optimization method based on the visibility-weighted saliency metric. This method takes into account the perceptual importance of voxels and the visibility of features, and automatically adjusts the transfer function to match the target saliency levels specified by the user. In addition, a parallel line search strategy is presented to improve the performance of the optimization algorithm.

Finally, we develop a novel visualization approach which modulates focus, emphasizing important information, by adjusting saturation and brightness of voxels based on an importance measure derived from temporal and multivariate information.

\section{Summary of Chapters}
The rest of this thesis is structured as follows:

Chapter~\ref{related_work_chapter}
provides an overview of the background and related work in the field of volume visualization, with particular focus on the design and optimization of transfer functions.

Chapter~\ref{transfer_function_refinement}
presents a novel approach for transfer function refinement, which is an optimization of transfer functions based on the distribution (i.e. the histogram) of the volume data. This optimization also allows the user to prioritize specific regions by generating weightings for transfer function components based on user-selected regions. The work described in this chapter has been published as a short paper in Eurographics 2014 \cite{luo_information-guided_2014} and as a full paper in Eurasia 2014 \cite{shengzhou_luo_transfer_2014}.

Chapter~\ref{visibility-weighted_saliency}
describes visibility-weighted saliency as an important measure of visual saliency of features in volume rendered images, in order to assist users in choosing suitable viewpoints and designing effective transfer functions to visualize the features of interest. Visibility-weighted saliency is based on a computational measure of perceptual importance of voxels and the visibility of features in volume rendered images. The visibility-weighted saliency metric has been published as a full paper in Computer Graphics \& Visual Computing (CGVC) 2015 \cite{borgo_visibility-weighted_2015}.

Chapter~\ref{transfer_function_optimization}
provides a detailed description of an automated transfer function optimization approach based on the visibility-weighted saliency metric, which indicates the perceptual importance of voxels and the visibility of features in volume rendered images.
%The work described in this chapter and the perceptual experiment in Chapter~\ref{visibility-weighted_saliency} has been submitted to Computer Graphics International 2016.
The work described in this chapter has been presented as a poster at EG / VGTC Conference on Visualization (EuroVis) 2016 \ref{isenberg_transfer_2016}.

Chapter~\ref{selective_saturation_brightness}
outlines a novel visualization approach which modulates focus, emphasizing important information, by adjusting saturation and brightness of voxels based on an importance measure derived from temporal and multivariate information.
The work described in this chapter has been presented as a poster at EG / VGTC Conference on Visualization (EuroVis) 2015 \cite{shengzhou_luo_selective_2015}.

Chapter~\ref{conclusions}
summarizes our contributions and provides a discussion of possible avenues of future work.

%...
%
%	\section{Section 1.1}
%
%	...

%-------------------------------------------------------------------------
                                
\chapter{Related Work \label{related_work_chapter}}
In this chapter, we present a brief review of the literature related to the concepts that we discuss in this thesis.

\section{Volume Rendering \label{volume_rendering}}
%Volume rendering is used to display a 2D image of a three-dimensional (3D) data set. It can be considered as a projection of a 3D volumetric data set into a two-dimensional (2D) image \cite{garcia_parallel_2006}.
%% without extracting intermediate polygonal representations \cite{garcia_parallel_2006}.
%The majority of data sets are discretely sampled along 3D grids and contain scalar values usually acquired from medical imaging devices such as CT or MRI machines or various scientific simulations such as fluid simulation.
%The data then takes the form a 3D array of voxels (a three dimensional extension of pixels).
%%There are various kinds of volumetric data sets. Typical volumetric data sets in medical visualization are groups of two-dimensional (2D) slice images acquired by a CT, MRI, or MicroCT scanner.
%In flow visualization, the data sets are often generated from simulations.
%Volume rendering is called direct volume rendering, where no polygonal representations are generated in the process, whereas rendering from polygonal representations extracted from volumetric data sets is called indirect volume rendering.
%Volume rendering can be performed using two main techniques, either by extracting a number of surfaces from the data and rendering these surfaces to the screen, called isosurface rendering or by rendering the volume itself as a complete block of data with no intermediary structures, usually called direct volume rendering (DVR).

Volume rendering is used to display a two-dimensional (2D) image of three-dimensional (3D) data set. It can be considered as a projection of a 3D volumetric data set to a 2D image \cite{garcia_parallel_2006}.
The majority of data sets are discretely sampled along 3D grids and contain scalar values usually acquired from medical imaging devices such as CT or MRI machines or computed from scientific simulations such as fluid simulation.
The data sets have the form of 3D arrays and the elements in the data sets are called voxels, which correspond to pixels in 2D images.
An example of volume rendering is provided in Figure~\ref{fig:multiple_VisMale}, which shows a sliced image and a volume rendered image of a head data set.

%\begin{figure}
%	\centering
%	\begin{minipage}{0.25\textwidth}
%		\centering
%		\includegraphics[width=1\linewidth]{images/VisMale_slice.jpg}
%		\caption{A sliced image of the data set}
%		\label{fig:VisMale_slice}
%	\end{minipage}~
%	\begin{minipage}{0.25\textwidth}
%		\centering
%		\includegraphics[width=1\linewidth]{images/VisMale.jpg}
%		\caption{Volume rendering of the data set}
%		\label{fig:VisMale}
%	\end{minipage}
%	\caption{The VisMale data set \cite{website:Roettger_volume_2013}}
%	\label{fig:multiple_VisMale}
%\end{figure}

Traditionally, volume rendering techniques are categorized as direct volume rendering and indirect volume rendering.
Indirect volume rendering is done be extracting surfaces from the data sets and rendering these surfaces, while direct volume rendering render the volume data set as a complete block of data without extracting intermediary structures.
Since indirect volume rendering is not in the scope of this thesis, direct volume rendering would henceforth be referred to as volume rendering.

Volume visualization is another term for volume rendering, sometimes with emphasis on the realistic aspects of volume rendering. In addition to the realistic aspects of volume rendering, there is non-photorealistic volume rendering, which emphasizes the illustrative and artistic aspects of volume rendering.
Volume visualization was initially used in medical imaging, and later became an essential techinque in many sciences for portraying complex phenomena such as clouds, water flows, and molecular and biological structure \cite{rosenblum_scientific_1994}.

\begin{figure}
\centering
\begin{minipage}{.25\textwidth}
\includegraphics[width=1\linewidth]{images/VisMale_slice.jpg}
\caption{A sliced image of the data set}
\label{fig:VisMale_slice}
\end{minipage}~
\begin{minipage}{.25\textwidth}
\includegraphics[width=1\linewidth]{images/VisMale.jpg}
\caption{Volume rendering of the data set}
\label{fig:VisMale}
\end{minipage}
\caption{The VisMale data set \cite{website:Roettger_volume_2013}}
\label{fig:multiple_VisMale}
\end{figure}

%Whereas rendering from polygonal representations extracted from volumetric data sets is also called indirect volume rendering.

%There are four typical volume rendering techniques: ray casting, splatting, shear warp and texture mapping. In recent years, many variants and combinations of these techniques have been proposed, especially approaches which utilise GPU hardware to improve performance or even achieve real-time interactivity.

%A volume renderer maps every voxel (an element in volume data) to an opacity and a color with a transfer function, which is a piecewise linear function or an arbitrary table. Once converted to an RGBA value, the composed RGBA result is projected on corresponding pixel of the frame buffer with certain volume rendering techniques.

\subsubsection{Non-Photorealistic Rendering (NPR)}
In contrast to traditional computer graphics, which has focused largely on creating photorealistic images of synthetic objects, non-photorealistic rendering (NPR) is an area of computer graphics that focuses on creating abstract images with a wide variety of expressive styles \cite{haeberli_paint_1990}. NPR has been an active research area for a long time. A number of approaches have been proposed to produce convincing artistic styles for both off-line and on-line rendering. For example, there are various types of commonly used styles including painterly rendering, edge stylization, sketch-shading, cel-shading, hatching.
% Although there are plenty of works on non-photorealistic rendering, most of them focus on how to imitate artistic styles with emphasis on aesthetic aspects.
%Researchers in modelling and rendering in computer graphics have focused for many years on producing photorealistic images, which are indistinguishable from photographs captured from real-world scene. Nevertheless, there are other compelling methods of visual discourse such as paintings, sketches and cel animation.
In certain situations, non-photorealistic renderings are considered more effective and expressive than an equivalent photograph \cite{healey_perceptually_2004}.

%As a basic part of most approaches to non-photorealistic volume rendering, contours delineate object shape and clarify sites of occlusion by emphasizing the transition between front-face and back-facing surface locations.
%Kindlmann et al. \cite{kindlmann_curvature-based_2003} proposed curvature-based transfer function to enhance the expressive and informative power of volume rendering. In their approach, volume data is rendered with contours to exhibit constant thickness in image space.

\subsubsection{Illustrative Volume Visualization}
NPR models were adopted in visualization and hence formed the field illustrative visualization.
Illustrative visualization, as a novel category of visualization, aims at visualizing data in a clear and understandable way using techniques from traditional hand-crafted illustrations.
Illustrative visualization has been successful employed in medical visualization \cite{svakhine_illustration_2005} \cite{viola_importance-driven_2005} \cite{svakhine_illustration-inspired_2009}.

Illustration-based styles are believed to be effective in conveying information. Researchers in the field of computer graphics and visualization have applied illustration-based styles in order to produce effective and expressive visualization. Stompel et al. \cite{stompel_visualization_2002} introduced feature enhancement techniques, such as strokes based, temporal domain enhancement, to enhance time-varying data obtained from the field of computational fluid dynamics (CFD).

In scientific visualization, features of interest are often inner structures of the data sets, e.g. visualizing certain structures in anatomical data sets \cite{diaz_iriberri_enhanced_2013}.
Inspired by techniques from illustration, various approaches are proposed to reveal different level of structures simultaneously in volume data sets.
Two level rendering \cite{hauser_two-level_2001} \cite{hadwiger_high-quality_2003} \cite{corcoran_perceptual_2010} is a method of merging several volume rendering techniques into a single rendering. This method is useful when inner structures needs to be rendered along with semitransparent outer parts.
Focus and context \cite{wang_magic_2005} \cite{bruckner_illustrative_2006} \cite{chen_intelligent_2008} provides a means of revealing the interior of a volume data set in a feature-driven way and also retaining context information.
Another alternative is the cutaway technique \cite{burns_feature_2007} \cite{sigg_intelligent_2012}, which makes inner structures clearly visible along with its spatial relation to the surrounding material.

\subsubsection{Other NPR Techniques in Volume Visuailzation \label{painterly_rendering}}
Streamlines and textures are often used to represent flow directions \cite{urnessy_techniques_2004}. Interrante and Grosch \cite{interrante_strategies_1997} introduced volume LIC (Line Integral Convolution) to visualize 3D flow via volume textures.
%Figure~\ref{fig:interrante_strategies_1997} shows a volume texture generated with LIC. In this figure, vorticity magnitude is mapped to streamline color and the striations along the axial direction reveal the presence of periodic waves propagated down the jet axis.
Since volume LIC is limited to steady flows, Liu and Moorhead \cite{liu_texture-based_2005} introduced an accelerated unsteady flow LIC algorithm to generate volume flow textures.
% (Figure~\ref{fig:liu_texture-based_2005}).
In their approach, magnitude-based transfer functions and cutting planes in volume rendering are employed to show the flow structure and the flow evolution.

Artists recognize patterns and flows in a target scene and express them through brush strokes. They use stroke orientation corresponding to the actual movement \cite{lee_motion_2009}. This type of techniques from master painting and human perception are used to visualize multidimensional data sets \cite{healey_perceptually_2004}. Tateosian et al. \cite{tateosian_engaging_2007} use the color, orientation and size of strokes to represent the magnitude, flow orientation and pressure of a 2D slice of a simulated supernova collapse.
% (Figure~\ref{fig:tateosian_engaging_2007}).
Lee et at. \cite{lee_motion_2009} presented a painterly rendering technique based on the motion information (magnitude, direction, standard deviation) extracted from an image sequences of the same view.
Besides stroke-based techniques and texture-based techniques, particle-based techniques \cite{busking_particle-based_2007} \cite{van_pelt_illustrative_2010} were also proposed to produce user-configurable stylized renderings from volume data sets, imitating traditional pen and ink drawings.

%\begin{figure}
%	\centering
%	\begin{minipage}{.49\textwidth}
%		\centering
%		\includegraphics[width=1\linewidth]{images/interrante_strategies_1997.png}
%		\caption{Streamlines are represented as a volume texture, which provides an intuitive impression of the 3D flow. \cite{interrante_strategies_1997}}
%		\label{fig:interrante_strategies_1997}
%	\end{minipage}~
%	\begin{minipage}{.49\textwidth}
%		\centering
%		\includegraphics[width=1\linewidth]{images/liu_texture-based_2005.png}
%		\caption{Streamlines are represented as a volume texture, which provides an intuitive impression of the 3D flow. \cite{liu_texture-based_2005}}
%		\label{fig:liu_texture-based_2005}
%	\end{minipage}
%\end{figure}
%
%\begin{figure}
%	\centering
%	\begin{minipage}{.49\textwidth}
%		\centering
%		\includegraphics[width=1\linewidth]{images/tateosian_engaging_2007.png}
%		\caption{A visual complexity style visualization of flow patterns in a 2D slice though a simulated supernova collapse, using the mappings: flow orientation $ \rightarrow $ stroke orientation, magnitude $ \rightarrow $ order and pressure $ \rightarrow $ stroke size \cite{tateosian_engaging_2007}.}
%		\label{fig:tateosian_engaging_2007}
%	\end{minipage}~
%	\begin{minipage}{.49\textwidth}
%		\centering
%		\includegraphics[width=1\linewidth]{images/lee_motion_2009.png}
%		\caption{The motion directions determine stroke orientations in the regions with significant motions, and image gradients determine stroke orientations where little motion is observed \cite{lee_motion_2009}.}
%		\label{fig:lee_motion_2009}
%	\end{minipage}
%\end{figure}

\section{Transfer Functions \label{literature_of_transfer_function}}
Volume data are 3D entities with information inside them, but the data might not consist of surfaces and edges.
%, or might be too voluminous to be represented geometrically.
%The goal of volume visualization is to gain insightful depictions of volume data.
Because of the lack of explicit geometric information, %and limited semantics, 
it is a major challenge to provide clear visualizations of the structures contained in a volume dataset.
Volume data may be rendered directly by mapping scalar values to visual properties (e.g. opacity and color), or an intermediate geometric representation may be extracted using techniques like Marching Cubes \cite{lorensen_marching_1987} and then rendered as geometric surfaces. The mapping, which assigns visual properties to volume data, is called a transfer function.

Transfer function specification is an essential part in volume visualization.
%In terms of dimensionality, transfer functions are divided into two categories, one-dimensional (1D) and multidimensional.
A simple one-dimensional transfer function is a mapping from scalar values to RGB and alpha values.
The resulting visualization largely depends on how well the transfer function captures features of interest \cite{kniss_multidimensional_2002}.
%Due to the complex nature of volumetric data sets, abstraction techniques is often used in order to provide better understanding of the data sets.
However, it is non-trivial to obtain an effective transfer function. The specification is often a trial-and-error process, which involves a significant amount of tweaking of color and opacity. Figure~\ref{fig:multiple_glk_transfunction} shows how slight changes in the transfer function lead to significant changes in the resulting images. The adjustment of transfer functions is unintuitive and often difficult.

\begin{figure}
        \centering
        \begin{minipage}{0.5\textwidth}
                \centering
                \includegraphics[width=1\linewidth]{images/glk_transfunction_tf.png}
                \caption{Two transfer functions (TF)}
                \label{fig:glk_transfunction_tf}
        \end{minipage}~
        \begin{minipage}{0.25\textwidth}
                \centering
                \includegraphics[width=1\linewidth]{images/glk_transfunction_1.png}
                \caption{The result from the TF on the left in \ref{fig:glk_transfunction_tf}}
                \label{fig:glk_transfunction_1}
        \end{minipage}~
        ~ %add desired spacing between images, e. g. ~, \quad, \qquad etc.
          %(or a blank line to force the subfigure onto a new line)        
        \begin{minipage}{0.25\textwidth}
                \centering
                \includegraphics[width=1\linewidth]{images/glk_transfunction_2.png}
                \caption{The result from the TF on the right in \ref{fig:glk_transfunction_tf}}
                \label{fig:glk_transfunction_2}
        \end{minipage}    
        \caption{Slight changes in the transfer function causes significant difference in the resulting images \cite{kindlmann_transfer_2002}}
        \label{fig:multiple_glk_transfunction}
\end{figure}

In practice, major factors that have a great influence on transfer function setting are: partial volume effect \footnote{During the acquisition of data, the finite resolution causes contributions of different materials combined into the value of a single voxel. This is generally referred to as the partial volume effect, which results in blurred boundaries and hampers the detection of small or thin structures. \cite{serlie_classifying_2007}}, non-uniform distribution of materials and noise \cite{serlie_computed_2003}.
Among these, two challenging problems that need to be tackled could be elaborated as follows: firstly, for volume data sets, e.g. those obtained by MRI and CT, different tissues are represented in similar or even overlapping ranges of scalar values; secondly, interesting interior structures are often partly or completely occluded by surrounding tissue.
%This is common in visualizing interior structures. 
Consequently, feature detection and understanding volume data become a big challenge.

These problems are handled by transfer functions, which have played a crucial role in volume visualization.
%Transfer functions have played a crucial role in volume visualization.
Good transfer functions reveal important structures in the data without obscuring them with less important regions.
%Traditional one-dimensional transfer function approaches, which assign optical properties only based on scalar value, are inadequate to extract inner structures of interest from volume data.
%Various strategies have been proposed to simplify transfer function specification \cite{pfister_transfer_2001}.
The design of transfer functions to generate informative visualizations has been a significant challenge addressed by a number of researchers \cite{pfister_transfer_2001}.
Various strategies have been proposed for transfer function design \cite{hadwiger_real-time_2006}.
%Data-centric strategies examine the properties of volume data sets.
Data-centric strategies examine the properties of volume data sets.
However, certain features are difficult to be extracted and visualized with 1D transfer functions, e.g. CT and MRI data sets which contain complex boundaries between multiple materials.
Overlapping intensity intervals corresponding to different materials make boundary separation difficult.
When one intensity value or interval is associated with multiple boundaries, a 1D transfer function is unable to render them in isolation \cite{kniss_multidimensional_2002}.
%Overlapping intensity intervals corresponding to different materials make boundary detection difficult.
%However, certain features of interest in volume data are difficult to extract and visualize with 1D transfer functions. For instance, many medical data sets created from CT or MRI scans contain a complex combination of boundaries between multiple materials. This situation is problematic for 1D transfer functions because of the potential for overlap between the data value intervals spanned by the different boundaries. When one data value or data range is associated with multiple boundaries, a 1D transfer function is unable to render them in isolation \cite{kniss_multidimensional_2002}.

Classical approaches to this problem try to detect boundary information between tissues by introducing derived attributes such as first and second-order derivatives to isolate materials \cite{kindlmann_semi-automatic_1998} \cite{kniss_multidimensional_2002}. In this case, the transfer functions are extended to multidimensional feature spaces. 
The introduction of multidimensional transfer functions alleviates the material separation problem.
Instead of classifying a sample based on a single scalar value, multi-dimensional transfer functions allow a sample to be classified based on a combination of values.
Multidimensional transfer functions are very effective means to extract materials and their boundaries for both scalar and multivariate data. However, the parameter spaces of multidimensional transfer functions are more complex (compared to 1D transfer functions) and thus introduce problems such as requirement for large amount of user interaction, missing precision or the interaction being complex and unintuitive \cite{arens_survey_2010}.
%As a result, the interaction of transfer functions becomes more complex and unintuitive.

There are other multi-dimensional transfer functions approaches, such as spatialized gradient-based transfer functions \cite{roettger_spatialized_2005}, distance-based transfer functions \cite{tappenbeck_distance-based_2006}, size-based transfer function \cite{correa_size-based_2008}, texture-based transfer functions \cite{caban_texture-based_2008} \cite{alper_selver_exploring_2015} and curvature based transfer functions \cite{kindlmann_curvature-based_2003}.
Bruckner and Gr{\"o}ller introduced the concept of style transfer functions \cite{bruckner_style_2007}, which aim to produce more comprehensible images by using transfer functions that map input values to different NPR rendering styles.

Another strategy is based on the selection of rendered images. This strategy lets the user select one or more favorite images to guide the further search of transfer functions \cite{marks_design_1997} \cite{wu_interactive_2007}. More recent approaches introduced visibility \cite{correa_visibility_2011} or measures derived from information theory \cite{haidacher_information-based_2008} \cite{bruckner_isosurface_2010} \cite{ruiz_automatic_2011} \cite{bramon_information_2013}. Zhou et al. \cite{zhou_transfer_2012} studied the combination of 2D transfer functions with occlusion and size-based transfer functions.

Despite the advances of these methods, transfer function design for volume rendering is still an open research problem.
The creation of transfer functions needs to be simplified and the functionality of transfer functions needs to be extended in order to realize the full potential of volume rendering. For instance, more sophisticated transfer functions are required in medical imaging, in order to address various domain specific visualization problems \cite{lindholm_spatial_2010}.

Moreover, transfer function specification in general is an unintuitive or even monotonous task for average users, because it is usually an iterative process of trial and error.
For instance, there are skin and fat tissues around the brain, and their intensities lie in the same range as the brain. If we want to visualize the brain by setting the scalar value range of the brain to opaque, the surrounding skin and fat tissue will also become opaque. Then the brain will be occluded by the surrounding soft tissues which make it difficult to explore the brain structure.
Common approaches to this problem are to introduce explicit segmentation of structures of interest before the volume rendering process \cite{rezk-salama_opacity_2006}. In fact, the process of applying the transfer function could be interpreted as a segmentation problem.

\subsection{Multidimensional Transfer Functions}
Multidimensional transfer functions \cite{kniss_interactive_2001}, which are mappings from intensity and other variables, such as first and second derivatives to color and opacity, have demonstrated their effectiveness in distinguishing boundaries between materials in volume data.

%Overlapping intensity intervals corresponding to different materials make boundary detection difficult. Classical approaches try to detect boundary information between tissues by introducing derived attributes such as first and second derivatives to isolate materials \cite{kindlmann_semi-automatic_1998} \cite{kniss_multidimensional_2002} \cite{kindlmann_transfer_2002}.
%In this case, the transfer functions are extended to multidimensional feature spaces. As a result, the interaction of transfer functions becomes more complex and unintuitive as the dimensionality becomes higher.
%%Even two-dimensional transfer functions require a considerable amount of user interaction to find a meaningful shape \cite{arens_survey_2010}.

In volume data, boundaries are regions between areas of relatively homogeneous material. It is difficult to detect boundaries because different materials often consist of overlapping intensity intervals. To address this problem, multidimensional transfer functions used derived attributes such as gradient magnitudes and second derivatives along with scalar values, in order to detect transitions between relatively homogeneous areas \cite{kindlmann_semi-automatic_1998} \cite{kniss_multidimensional_2002} \cite{kindlmann_transfer_2002}.
%Classical approaches try to detect boundary information between tissues by introducing derived attributes such as first and second derivatives to isolate materials \cite{kindlmann_semi-automatic_1998} \cite{kniss_multidimensional_2002} \cite{kindlmann_transfer_2002}.
In this case, the transfer functions are extended to multidimensional feature spaces.
For higher-dimensional transfer functions, the generation of transfer functions could be memory intensive and costly to compute,
%and exploration of the transfer function domain might not be intuitive.
and the interaction of transfer functions becomes more complex and unintuitive as the dimensionality becomes higher. 
%As a result, the interaction of transfer functions becomes more complex and unintuitive as the dimensionality becomes higher.

Therefore, two-dimensional (2D) histograms are often used in multidimensional transfer functions \cite{maciejewski_structuring_2009}. An example is a 2D histogram with axes representing a subset of the feature space (e.g. scalar value vs gradient magnitude), with each entry in the 2D histogram being the number of voxels for a given feature space pair.
%A great number of transfer function approaches have also been merged and one-dimensional transfer function is extended to multi-dimensional transfer function.
Even in the case of two-dimensional transfer functions, a considerable amount of user interaction is required in order to come up with meaningful results \cite{arens_survey_2010}.
%Even in the case of two-dimensional transfer functions, a considerable amount of user interaction is required in order to come up with meaningful results \cite{arens_survey_2010}.
%There are other multi-dimensional transfer functions approaches, such as spatialized gradient-based transfer functions \cite{roettger_spatialized_2005}, distance-based transfer functions \cite{tappenbeck_distance-based_2006}, size-based transfer function \cite{correa_size-based_2008}, texture-based transfer functions \cite{caban_texture-based_2008} and curvature based transfer functions \cite{kindlmann_curvature-based_2003}.

As one of the most common representations of voxel distributions, histograms are used in transfer function design to assign visual properties to voxels \cite{pfister_transfer_2001}. Bajaj et al. \cite{bajaj_contour_1997} introduced the contour spectrum to determine voxels corresponding to important isosurfaces in the volume. To overcome the difficulty of using one-dimensional transfer functions (solely based on scalar values stored in the voxels) to extract inner structures of interest from the volume data, Levoy proposed the use of gradient magnitude to emphasize strong boundaries between different tissues \cite{levoy_display_1988}.

The introduction of gradient magnitude as a data metric aims to detect voxels that are of large deviation compared with other voxels by approximating gradient magnitude at each sample point in the volume, because the exact distribution of data is unknown due to information lost in the discrete sampling process.

Kindlmann and Durkin extended Levoy's work by introducing a higher dimensional transfer function domain based on gradient magnitudes and second derivatives \cite{kindlmann_semi-automatic_1998}. To emphasize different structures, Kniss et al. \cite{kniss_interactive_2001} presented a technique for interactively manipulating 2D histograms of gradient magnitudes and data values. In their work, material boundaries appear as arcs in the 2D histogram and can be selected with interactive widgets \cite{kniss_multidimensional_2002}.
Kindlmann et al. \cite{kindlmann_curvature-based_2003} proposed curvature-based transfer function to enhance the expressive and informative power of volume rendering. In their approach, volume data is rendered with contours to exhibit constant thickness in image space.
Maciejewski et al. proposed a non-parametric method to generate transfer function \cite{maciejewski_structuring_2009}.
In their later work \cite{maciejewski_abstracting_2013}, instead of using the attributes, metrics representing relationships and correlations in the underlying data were used in the method.
Wang et al. introduced clustering of 2D density plots in their automating transfer function \cite{wang_automating_2012}.
Ip et al. \cite{ip_hierarchical_2012} described a multilevel segmentation technique that mimics user exploration behaviors by recursively segmenting intensity-gradient histograms. The use of parallel coordinates is introduced to assist the design of transfer functions in multidimensional parameter spaces \cite{zhao_multi-dimensional_2010} \cite{guo_multi-dimensional_2011}.

\subsection{Transfer Functions for Time-Varying Volume Visualization}
Although researchers have developed a great number of visualization techniques for time-invariant volume data, how to effectively explore and understand time-varying volume data remains a challenging problem. Finding good transfer functions for time-invariant volume data itself has proven difficult \cite{pfister_transfer_2001}.
%Finding good transfer functions for time-varying volume data is even more difficult, as data value ranges and distributions change over time.
%Although researchers have developed a great number of visualization techniques for static volume data, how to effectively explore and understand time-varying volume data remains a challenging problem.
Finding good transfer functions for time-varying volume data is more difficult than for static volume data, as data value ranges and distributions change over time.

Coherence is an important issue of transfer function design for time-varying volume data. Ideally, a single transfer function should be used for the whole time-varying data set in order to obtain coherent visualization. More than one color or opacity map can be misleading or physically meaningless, because the transition from one transfer function to another may cause sudden changes in the resulting images. However, this practice is not always applicable to general time-varying data sets.

Volume data sets are inherently 3D representations. Automated analysis methods, such as temporal trends or statistical aggregates e.g. mean values and standard deviations, are often applied in order to abstract dynamic characteristics of the data sets \cite{kehrer_visualization_2013}.

%Jankun-Kelly and Ma \cite{jankun-kelly_study_2001} examined how to combine transfer functions for different time-steps to generate a coherent transfer function.
Jankun-Kelly and Ma \cite{jankun-kelly_study_2001} examined how to combine transfer functions for different time-steps to generate a coherent transfer function.
Woodring et al. \cite{woodring_high_2003} considered time-varying volume data as four-dimensional data field and provided a user interface to specify hyperplanes in 4D.
Woodring and Shen \cite{woodring_chronovolumes_2003} introduced an alternative approach to render multiple time-steps in a sequence with different colors into a single image. This approach provides the context of surrounding time steps but coherence of color among time-steps is hard to maintain.
%Tikhonova et al. \cite{tikhonova_exploratory_2010} presented an exploratory approach based on a compact representation of each time step of the dataset in the form of ray attenuation functions. Ray attenuation functions are subsequently used for transfer function generation.
Tikhonova et al. \cite{tikhonova_exploratory_2010} presented an exploratory approach based on a compact representation of each time step of the dataset in the form of ray attenuation functions. Ray attenuation functions are subsequently used for transfer function generation.
Akiba et al. \cite{akiba_simultaneous_2006} introduced the time histogram which allows simultaneous classification and specification of temporal transfer functions for the entire time series.

A time-varying volume data set can be considered as a 3D array where each voxel contains a time-activity curve (TAC). Fang et al. \cite{fang_visualization_2007} described an approach for classifying time-varying volume data based on the temporal behavior of voxels and three different similarity measures that can be used in their approach.
Woording and Shen \cite{woodring_multiscale_2009} presented a method that filters time-varying volume data into several time scales using wavelet transform and classifies the voxels by clustering the entire time series by time scale.
Lee and Shen \cite{lee_visualizing_2009} proposed a method for classifying time-varying features using time activity curves with the dynamic time warping distance metric.
Woodring et al. \cite{woodring_semi-automatic_2009} utilized a method called temporal clustering and sequencing to find dynamic features in value space and create dynamic transfer functions through time-series analysis.
Ward and Guo \cite{ward_visual_2011} presented a method for visualizing time-series data that reveals a wide variety of features in the data, by mapping short sub-sequences of the time-varying volume data into a high-dimensional shape space, and then performing a dimension reduction process to allow projection into screen space.
Gu and Wang \cite{gu_transgraph:_2011} proposed an approach to organize a time-varying data set into a hierarchical graph, which captures the transition relationships in the data set. This approach assists the user in comprehending the correspondence between volume regions over time and allows interaction of the graph through brushing and liking.

\begin{figure}
\centering
\includegraphics[width=1\linewidth]{images/woodring_semi-automatic_2009}
\caption{A single static transfer function cannot capture dynamic features. In the two images at the top, the features appear to vanish over time. On the other hand, the features are visible over time if a dynamic transfer function is used (in the two images at the bottom) \cite{woodring_semi-automatic_2009}.}
\label{fig:woodring_semi-automatic_2009}
\end{figure}

\subsubsection{Visualising Time-Varying Volume Data with NPR}
An essential problem in time-varying volume visualization is to visualize temporal variation and analysis of features. Traditionally, time-varying data has been visualized as snapshots of individual time steps or animation of snapshots of a sequence of time steps. These techniques are effective in making time-varying data understandable. However, they struggle when the complexity of data sets increased dramatically in recent years \cite{brambilla_illustrative_2012}.

Compared to flow visualization, which is a well established branch of scientific visualization \cite{brambilla_illustrative_2012}, general time-varying volume visualization is still a relative young field.
Illustrations for time-varying rendering could be divided into two categories, one is to enhance time-invariant features and the other is to enhance temporal features of time-varying volume data. The techniques in the first category focus on enhancing structural perception of volume models through the amplification of features and the addition of illumination effects \cite{rheingans_volume_2001} \cite{joshi_illustration-inspired_2005}. Examples of these techniques include boundary enhancement, oriented feature enhancement (silhouettes, fading, and sketch lines). The techniques in the second category focus on illustrating dynamic aspects such as movement of features. A few kinds of techniques have been proposed for this purpose. For example, there are speed lines, flow ribbons and strobe silhouettes, which are inspired by traditional animation \cite{joshi_illustration-inspired_2005} \cite{joshi_evaluation_2008} \cite{joshi_case_2009}; and there are extended silhouette and boundary enhancement domains, which are inspired by the techniques used by illustrators and other artists \cite{svakhine_illustration_2005}. Nevertheless, illustrations of temporal features of time-varying data requires more attention from researchers in the visualization community. The usefulness of illustrative approaches in time-varying volume visualization has not been studied as thoroughly as in other areas.

\section{Automatd Transfer Function Design}
Researches have proposed various approaches to automate the design of transfer functions and provide acceptable suggestions which can be further edited by users. However, the usefulness of a transfer function mostly depends on the underlying question the user wants to answer. Moreover, the users' tasks vary drastically from one domain to another. Therefore, most techniques work semi-automatically and very few techniques consider domain knowledge in the design process \cite{zudilova-seinstra_trends_2008}.

He et al. \cite{he_generation_1996} addressed transfer function exploration as a parameter optimization problem and presented an approach to assist the user in exploring appropriate transfer functions using stochastic search techniques starting from an initial population.
Another strategy is based on the selection of rendered images. This strategy lets the user select one or more favorite images to guide the further search of transfer functions \cite{marks_design_1997}.
Rezk-Salama et al. \cite{rezk-salama_automatic_2000} presented high-level semantics to abstract parametric models of transfer functions in order to automatically assign transfer function templates.

\cite{tzeng_novel_2003}
\cite{tzeng_cluster-space_2004}
\cite{tzeng_intelligent_2005}

\cite{jani_opacity_2005}

Wu and Qu \cite{wu_interactive_2007} developed a method that uses editing operations and stochastic search of the transfer function parameters to maximize the similarity between volume-rendered images given by the user.

\cite{wang_efficient_volume_2011}

\cite{zhou_automatic_2009}

\cite{marchesin_per-pixel_2010}

\cite{peng_optimal_2011}
\cite{lathen_automatic_2012}

\cite{woo_feature-driven_2012}

\cite{jung_opacity-driven_2014}

\cite{alper_selver_semiautomatic_2009}


\section{Visibility Histograms and Visibility-Driven Transfer Functions}
\cite{jung_dual-modal_2012}
\cite{schlegel_visibility-difference_2013}

The visibility of structures revealed in volume rendering is more of a consequence of adjusting transfer functions than a design parameter \cite{preim_visual_2013}. In contrast, Correa and Ma \cite{correa_visibility-driven_2009} suggested to use visibility to guide transfer function design for both manual and automatic searches.

\begin{figure}
\centering
\includegraphics[width=1\linewidth]{images/correa_visibility-driven_2009}
\caption{Visibility histograms for two data sets \cite{correa_visibility-driven_2009}}
\label{fig:correa_visibility-driven_2009}
\end{figure}

With conventional transfer function design, the visibility of certain anatomical structures is a consequence of adjusting transfer function parameters. 
Correa and Ma suggest to control visibility directly by computing visibility histograms as a basis for visibility-driven transfer functions. Visibility-based transfer functions are actually an instance of a data-driven transfer function specification.

\cite{bordoloi_view_2005}
\cite{takahashi_feature-driven_2005}

\cite{emsenhuber_visibility_2008}
\cite{correa_visibility_2011}
\cite{wang_efficient_2011}

\cite{wan_fast_2010}

\begin{figure}
	\centering
	\includegraphics[width=1\linewidth]{images/wang_efficient_2011}
	\caption{Visibility based opacity specification \cite{wang_efficient_2011}}
	\label{fig:wang_efficient_2011}
\end{figure}

\cite{mak_visibility-aware_2011}

\cite{bronstad_visibility_2012}

\cite{jung_visibility-driven_2013}
\cite{zheng_visibility_2013}
\cite{ruiz_automatic_2011}
\cite{bramon_information_2013}

\cite{tang_depth-based_2011}
\cite{zhou_opacity_2014}

\section{Multivariate Volume Visualization}
Analyzing multivariate data is an importance and challenging topic in many scientific disciplines. For instance, applications in medicine, engineering and meteorology often require analyzing multivariate data.
However, multivariate volume data sets are usually mapped to a scalar dimension and visualized separately with standard volume rendering techniques.

Kniss and Hansen \cite{kniss_volume_2002} applied volume rendering with multidimensional transfer function to visualize multivariate weather simulations. In their approach, they combined the temperature and humidity as a multivariate field in order to assist the meteorologists in identifying the frontal zones. 

Akiba et al. presented the use of time histogram for simultaneous classification of time-varying data in order to find transfer functions that classify all the time steps of the data set \cite{akiba_simultaneous_2006}.

Woodring and Shen presented a method for the comparison of different data fields through the expression of a volume shader that composes data fields together with set operations \cite{woodring_multi-variate_2006}.

Wang et al. introduced an importance measure based on conditional entropy and categorize temporal behaviors by clustering the importance curves over time \cite{wang_importance-driven_2008}.

Lee and Shen \cite{lee_visualization_2009} extended the Dynamic Time Warping (DTW) approach \cite{lee_visualizing_2009} to SUBDTW, in order to estimate when a trend appears and vanishes in a given time series. They modeled the temporal relationships as a state machine based on the beginning and ending times of the trends.

Khlebnikov et al. \cite{khlebnikov_noise-based_2013} described a novel method that allows simultaneous rendering of multivariate data by redistributing the opacity within a voxel. This method uses procedural texture synthesis \cite{khlebnikov_procedural_2012} for opacity redistribution pattern and is similar in spirit to color weaving.

Data analysis techniques for high dimensional spaces, such as parallel coordinates \cite{akiba_visualizing_2007} \cite{guo_scalable_2012} and principal component analysis \cite{liu_multivariate_2014}, were also investigated for exploring multivariate time-varying data sets.

%\cite{woodring_multi-variate_2006}
%\cite{akiba_visualizing_2007}
%\cite{guo_scalable_2012}
%\cite{liu_multivariate_2014}

\section{Information Theory in Visualization}
Information theory \cite{shannon_mathematical_1948} was originally introduced to study the fundamental limit of reliable transmission of messages through a noisy communication channel. Traditional applications of information theory, such as data compression and data communication, focus on the efficient throughput of a communication channel, whilst visualization focuses on the effectiveness in aiding the perceptual and cognitive process for data understanding and knowledge discovery.

In recent years, there is an emerging direction towards using the principles of information theory to solve challenging problems in scientific visualization \cite{wang_information_2011}. These problems include view selection \cite{bordoloi_view_2005} \cite{takahashi_feature-driven_2005} \cite{feixas_unified_2009}, streamline seeding and selection \cite{xu_information-theoretic_2010} \cite{lee_view_2011}, transfer function for multimodal data \cite{bramon_multimodal_2012}, representative isosurface selection \cite{wang_lod_2006}, time-varying and multivariate data analysis \cite{wang_importance-driven_2008} and information channel between objects and viewpoints \cite{ruiz_viewpoint_2010}.

Chen and J{\"a}nicke \cite{chen_information-theoretic_2010} presented an information-theoretic framework for visualization. They examined the theoretical aspect of information and its relation to data communication and interpret different stages of the visualization pipeline using the taxonomy of information.

Haidacher et at. \cite{haidacher_information-based_2008} proposed an approach of transfer function specification for multi-modal data visualization. They considered the joint occurrence of multiple features from one or multiple variables in order to separate statistical features that only occur in a single variable from those that are present in both.

Want et al. \cite{wang_importance-driven_2008} introduced an approach to characterized the dynamic temporal behaviors of spatial blocks using importance curves, which are based on conditional entropy. Clustering is performed on the importance curves of all the spatial blocks to classify the underlying volume data set.

Ruiz et al. \cite{ruiz_automatic_2011} presented an approach to generate transfer functions from a target distribution provided by the user. Their approach is based on a communication channel between a set of viewpoints and a set of bins of a volume data set, and supports both 1D and 2D transfer functions including the gradient information.

Bramon et al. \cite{bramon_information_2013} proposed an automatic method to visualize multi-modal data by combining several information-theoretic strategies to define colors and opacity values of the multi-modal transfer function.
They set an information channel between two registered input data sets to define the fused color and minimize the informational divergence between the visibility distribution captured by a set of viewpoints and a target distribution proposed by the user to obtain the opacity.

\section{Computational Saliency Models}
Predicting salient locations has many real world applications, such as navigational assistance, robot control, surveillance systems, object detection and scene understanding \cite{zhao_learning_2013}.
Inspired by mechanisms of the human visual system, various computational models of visual saliency have been proposed to predict gaze allocation \cite{itti_model_1998} \cite{parkhurst_modeling_2002} \cite{harel_graph-based_2006} \cite{chikkerur_what_2010} \cite{mahadevan_spatiotemporal_2010} \cite{duan_visual_2011}.

Lee et al. \cite{lee_mesh_2005} presented mesh saliency, which is defined in a scale-dependent manner using a center-surround operator on Gaussian-weighted mean curvatures. They observed that their approach was able to capture the most visually interesting regions on a mesh.
Kim and Varshney \cite{kim_saliency-guided_2006} introduced the user of center-surround operators to compute saliency fields of volume data. In their user study, they found that their approach was better at eliciting viewer attention than the traditional Gaussian regional enhancement approaches.
Shen et al. \cite{shen_save:_2014} proposed the use of saliency to assist volume exploration. They described a method for inferring interaction position in volume visualization, in order to help users pick focused features conveniently.
Shen et al. \cite{shen_spatiotemporal_2015} described spatiotemporal volume saliency, which extended the saliency field \cite{kim_saliency-guided_2006} to time-varying volume data.

%\cite{kim_mesh_2010}
%\cite{song_mesh_2014}

%A saliency-based enhancement of volume visualization inspired by the center-surround mechanisms of the human
%visual system
%For volume \cite{kim_saliency-guided_2006}
%\cite{kim_saliency-guided_2008}

%saliency assisted
%\cite{wang_saliency-aware_2013}

%\cite{treue_visual_2003}
%\cite{healey_combining_2001}
%\cite{hall_trainable_2004}
%\cite{kadir_saliency_2001}
%\cite{cui_measuring_2006}

\section{Perceptual Evaluation}
%The term visualization may have various meanings.
%Chen et al. \cite{chen_what_2013} presented a definition, which is "visualization is a study of transformation from data to visual representations in order to facilitate effective and efficient cognitive processes in performing tasks involving data". Therefore, based on this definition, the fundamental measure for effectiveness is correctness and that for efficiency is the time required for accomplishing a task.

Due to the complex nature of the data being studied, simply displaying all available information does not adequately meet the demands of domain scientists \cite{anderson_evaluating_2012}.
%Determining the best use of visualization techniques is one of the goals of visualization evaluations.
User studies can be used to evaluate the strengths and weaknesses of visualization methods \cite{christopher_thoughts_2003}.
The evaluation of visualization methods that focus on human factors often employ user studies or expert evaluations to determine their effects on interpretation and usability.

There are a number of different evaluation strategies, such as measuring user performance, accuracy and experience \cite{redmond_influencing_2010}. Laidlaw et al. \cite{laidlaw_quantitative_2001} compared six methods for visualizing 2D vector fields and measured user performance on three flow-related tasks for each of the six methods. They used the evaluation results to identify what makes a 2D vector fields visualization effective.
Joshi and Rheingans \cite{joshi_evaluation_2008} evaluated the effective of their illustrative techniques by measuring user accuracy, time required to perform a task and user confidence.

Lu \cite{lu_volume_2010} described a method for automatically selecting rendering parameters to simplify user interaction and improve usability. Subsequently, a user study was conducted to evaluate the effectiveness of this method. Two data sets were rendered in three styles with three predefined portions of the data sets were highlighted respectively. The users' eye gaze patterns were analyzed to determine if they were able to accurately identify the highlighted areas in the images.

Kersten-Oertel et al. \cite{kersten-oertel_evaluation_2014} presented empirical studies on the effect of six different perceptual cues for enhancing depth. In the user study, the subjects were asked to determine which one of two indicated vessels was closer to them, and were asked to respond as accurately and quickly as possible. Both the percentage of correct answers and response times were analyzed.

Also for depth perception, D{\'i}az et al. \cite{diaz_perceptual_2015} conducted a user study to investigate the impact of well-known volumetric shading models in stereoscopic desktop-based environments. In the results, the average time spent and average correctness of answers were analyzed.

\section{Feature Tracking}
Feature extraction and tracking is an established technique for the analysis of time-varying data in various research fields, such as video analysis, computer vision and flow visualization \cite{muelder_interactive_2009}.
In time-varying data, features are objects that evolve over time. Feature tracking aims to determine the correspondence between features in successive time steps and describe the evolution of features through time \cite{post_state_2003}.

In practice feature extraction and tracking are often employed in the exploration and analysis of time-varying volume visualization in order to better understand the dynamic nature of the underlying phenomena \cite{tzeng_intelligent_2005} \cite{woodring_multiscale_2009} \cite{lee_visualizing_2009} \cite{gu_transgraph_2011}.
Feature extraction methods are often based on an analytical description of the feature of interest. Consequently, feature extraction and tracking could become a manual-driven and trial-and-error process in the case that the properties cannot be easily defined and are sometimes unknown \cite{ma_machine_2007}.

%%survey
\cite{ma_machine_2007} \cite{wang_information_2008}
%
\cite{caban_texture-based_2007}
%
%%tracking related works
\cite{widanagamaachchi_interactive_2012} \cite{ozer_group_2012}
\cite{hsieh_feature_2013}

\section{Vector Field Visualization}
The visualization of vector fields plays a crucial role in visual interpretation and understand of the underlying flow features and patterns \cite{kuhn_clustering-based_2011} \cite{ma_coherent_2013}. Since flow patterns also exist in time-varying volume data, certain techniques for visualizing vector field could be incorporated into time-varying volume visualization, in order to depict the dynamic aspects of time-varying data.

Line drawings are effective ways to depict complex information with simple means \cite{benard_state_art_2011}. Among vector field visualization techniques, streamline visualization is a simple but common way to convey the structure of 3D vector fields \cite{chen_illustrative_2011}. Streamlines have proven to give expressive visual representation if they are combined with appropriate seeding strategies \cite{annen_vector_2008}. Streamlines intuitively reveal flow patterns by integrating the flow path.

\section{Summary}
We have presented a review of the literature in the field of volume visualization.
The review suggests that it is feasible to optimize the parameters of volume visualization based on the information within the volume data. Existing research on multidimensional transfer functions, classification of volume data and information theory provides us the foundation to explore and understand how this optimization may be achieved.
%Existing research on the analysis and visualization of time-varying data and flow data gives us valuable information on the study of the dynamic aspect of time-varying volume data.
In particular, the analysis and visualization of time-varying data is a compelling problem due to increasing availability of such data in recent years. Although some work has been published in this area, it is clear that there is compelling work left to be done in optimizing the rendering of such data sets.
The use of NPR techniques can improve the expressiveness of the visualization and thus facilitate better user understanding of data.
Further studies are required in order to better integrate NPR techniques into the visualization pipeline and enhance the expressiveness of the visualization by exploiting the information within the data sets.

%-------------------------------------------------------------------------
                                
\chapter{Transfer Function Refinement for Exploring Volume Data}

...

\chapter{Visibility-Weighted Saliency for Volume Visualization \label{visibility-weighted_saliency}}
Volume visualization has been widely used to depict complicated 3D structures in volume data sets.
However, obtaining clear visualization of the features of interest in a volume is still a major challenge.
The clarity of features depends on the transfer function, the viewpoint and the spatial distribution of features in the volume data set.
In this chapter, we propose visibility-weighted saliency as a measure of visual saliency of features in volume rendered images, in order to assist users in choosing suitable viewpoints and designing effective transfer functions to visualize the features of interest. Visibility-weighted saliency is based on a computational measure of perceptual importance of voxels and the visibility of features in volume rendered images.
The effectiveness of this scheme is demonstrated by test results on two volume data sets.

%-------------------------------------------------------------------------
\section{Introduction}
Volume visualization is an active branch of scientific visualization. It is a method of extracting meaningful information from volumetric data sets, which usually contain complex structures of various material.
First introduced by Levoy \cite{levoy_display_1988} for visualization of volume data, volume visualization has been widely used in various sciences to create insightful visualizations from both simulated and measured data.

A crucial step in volume visualization is transfer function specification. Transfer functions assign visual properties, including color and opacity, to the volume data being visualized. Hence transfer functions determine which structures will be visible and how they will be rendered.
An appropriate transfer function can quickly reveal large amounts of information of the data set to the viewer.
However, obtaining an effective transfer function is a non-trivial task, which involves a significant amount of tweaking of color and opacity.
A cause of this problem is the lack of an objective measure to quantify the quality of transfer functions \cite{correa_visibility_2011}.

Although user studies are useful in evaluating some fundamental characteristics of visualization techniques, it is not possible to conduct a user study for each individual visualization every time it is created.
Several computational measures of visual saliency that model human attention have been developed \cite{itti_model_1998} \cite{harel_graph-based_2006}.
Kim and Varshney \cite{kim_saliency-guided_2006} introduced the saliency field, which measures visual saliency of voxels using the center-surround operator based on the difference of Gaussian-weighted averages at a fine and a coarse scale.
However, salient voxels may be occluded by other voxels close to the viewer in certain viewpoints and thus these salient voxels become invisible in the volume rendered image. In order to measure the visual saliency of features in volume rendered images, it is necessary to consider both the saliency and the visibility of the voxels which form the feature.

In this chapter, we propose visibility-weighted saliency as an improved measure of the visual saliency of features in volume rendered images. Visibility-weighted saliency is a combination of feature visibility \cite{wang_efficient_2011} and the saliency field \cite{kim_saliency-guided_2006}.
Feature visibility measures the contribution of each feature to the volume rendered image and saliency field measures the visual saliency of each voxel in its local neighborhood.
The visibility-weighted saliency are presented in two different ways, i.e. visibility-weighted saliency fields and feature saliency histograms. Visibility-weighted saliency fields display the spatial distribution of visual saliency of features and feature saliency histograms provide quantitative information about the perceptual importance of the features.
With visibility-weighted saliency, the saliency of features rendered in different viewpoints with different transfer functions can be measured in a quantitative and fully automated way.
Thus, this technique can be used to guide users in choosing appropriate viewpoints and designing effective transfer functions for the features of interest in volume visualization.
This technique is also useful for understanding how much different parts of the volume contribute to the final image and how different tissues occlude each other and interfere with each other's visibility.

%-------------------------------------------------------------------------
\section{Related Work}
Transfer functions have played a crucial role in volume visualization and the design of transfer functions to generate informative visualizations has been a significant challenge addressed by a number of researchers \cite{pfister_transfer_2001}.
Various strategies have been proposed to simplify transfer function specification \cite{hadwiger_real-time_2006}.
In volume data, boundaries are regions between areas of relatively homogeneous material. It is difficult to detect boundaries because different materials often consist of overlapping intensity intervals. To address this problem, multidimensional transfer functions used derived attributes such as gradient magnitudes and second derivatives along with scalar values, in order to detect transitions between relatively homogeneous areas \cite{kindlmann_semi-automatic_1998} \cite{kniss_multidimensional_2002} \cite{kindlmann_transfer_2002}.
%Classical approaches try to detect boundary information between tissues by introducing derived attributes such as first and second derivatives to isolate materials \cite{kindlmann_semi-automatic_1998} \cite{kniss_multidimensional_2002} \cite{kindlmann_transfer_2002}.
In this case, the transfer functions are extended to multidimensional feature spaces. As a result, the interaction of transfer functions becomes more complex and unintuitive as the dimensionality becomes higher.
Even in the case of two-dimensional transfer functions, a considerable amount of user interaction is
required in order to come up with meaningful results \cite{arens_survey_2010}.

Several computational models of visual saliency for modeling human attention have been developed.
Itti et al. \cite{itti_model_1998} developed a computational model of visual attention based on the center-surround operators in an image. This center-surround mechanism has the intuitive appeal of being able to identify regions that are different from their surrounding context.
Lee et al. \cite{lee_mesh_2005} proposed saliency for meshes based on a multi-scale center-surround mechanism that operates on local curvature. Kim and Varshney \cite{kim_saliency-guided_2006} presented the use of a center-surround operator using the Laplacian of Gaussian-weighted averages of appearance attributes to enhance selected regions of a volume and validated their work using an eye-tracking user study. Shen et al. \cite{shen_spatiotemporal_2015} extended this technique to spatiotemporal volume saliency to detect both spatial and temporal changes.

Visibility measures the impact of individual voxels on the image generated by a volumetric object and visibility distribution can be utilized as a measure on the quality of transfer functions as users explore the transfer function space. Visibility has been studied to measure the quality of a given viewpoint \cite{bordoloi_view_2005} \cite{viola_importance-driven_2004} and to enhance the rendering process with cutaway views.
Correa and Ma \cite{correa_visibility_2011} introduced visibility histogram, which describes the distribution of visibility in a volume rendered image.
Ruiz et al. \cite{ruiz_automatic_2011} proposed an automatic method to generate a transfer function by minimizing the Kullback-Leibler divergence between the observed visibility distribution and a target distribution provided by the user. Want et al. \cite{wang_efficient_2011} extended the idea of visibility histogram to feature visibility and introduced an interaction scheme where the opacity of each feature was generated automatically based on user-defined visibility values. Visibility distribution is also used in automating color mapping \cite{cai_automatic_2013} and 2D transfer functions \cite{qin_voxel_2015}.

%A number of general quality metrics have been proposed, including abstraction \cite{chen_measuring_2005} \cite{van_wijk_value_2005} and aesthetics \cite{filonik_measuring_2009} and visual saliency \cite{janicke_salience-based_2010}.
%Giesen et al. \cite{giesen_conjoint_2007} presented an user study design and analysis strategy geared to measure preceived quality in volume rendering.
%Wu et al. \cite{wu_quantitative_2010} described four types of quantitative assessments for volume rendered images, which are distinguishability measure, edge consistency measure, contour clarity measure and depth coherence measure.


%Various strategies have been proposed to simplify transfer function specification \cite{pfister_transfer_2001}.
%Data-centric strategies examine the properties of volume data sets.
%Overlapping intensity intervals corresponding to different materials make boundary detection difficult. Classical approaches try to detect boundary information between tissues by introducing derived attributes such as first and second derivatives to isolate materials \cite{kindlmann_semi-automatic_1998} \cite{kniss_multidimensional_2002} \cite{kindlmann_transfer_2002}.
%In this case, the transfer functions are extended to multidimensional feature spaces. As a result, the interaction of transfer functions becomes more complex and unintuitive as the dimensionality becomes higher.
%%Even two-dimensional transfer functions require a considerable amount of user interaction to find a meaningful shape \cite{arens_survey_2010}.
%Even in the case of two-dimensional transfer functions, a considerable amount of user interaction is
%required in order to come up with meaningful results \cite{arens_survey_2010}.


%Compared to visibility histograms, visibility volumes have a distinctive advantage, i.e. they maintain the spatial information of voxels in the volume.


%A number of approaches have been proposed to automate the design of transfer functions.
%Wu and Qu \cite{wu_interactive_2007} developed a method that uses editing operations and stochastic search of the transfer function parameters to maximize the similarity between volume-rendered images given by the user.
%Maciejewski et al. \cite{maciejewski_structuring_2009} described a method to structure attribute space in order to guide users to regions of interest within the transfer function histogram.
%Chan et al. \cite{chan_perception-based_2009} developed a system to optimize transparency automatically in volume rendering based on Metelli's episcotister model to improve the perceptual quality of transparent structures.
%Correa and Ma \cite{correa_visibility-driven_2009} proposed the visibility histogram to guide the transfer function design.

%-------------------------------------------------------------------------
\section{Method}
For 2D images, intensity and color are the most important attributes. In volume visualization, the intensity and color in the final images result from the blending of alpha and color determined by user-specified transfer functions in a specific viewpoint.
The saliency field is a view-independent scalar field that contains the visual saliency of each voxel in the volume data. The visual saliency of voxels represents the perceptual importance in 3D space, however it does not reflect how visible the voxels are in the final 2D images.

In order to take into account both the visual saliency of voxels in 3D space and the contribution of the voxels to final 2D images, we propose a visibility-based saliency metric, which attempts to measure the impact of individual voxels as well as user-specified features on volume rendered images.
This technique aims to assist users in gaining insight into the internal structure of the data set and understanding the contribution of different features to the final image.

The visibility-based saliency field is based on a visibility field and a saliency field. The visibility field contains the opacity contribution of each voxel to the final 2D image and the saliency field represents the visual saliency of each voxel in the 3D volume data.
%Our method uses the center-surround mechanism to compute visual saliency of voxels. The center-surround mechanism has the intuitive appeal of being able to identify regions that are different from their surrounding context and it has been successfully applied to problems on 2D images \cite{itti_model_1998}, 3D meshes \cite{lee_mesh_2005} and 3D volumetric data \cite{kim_saliency-guided_2006}.

In this section, we describe visibility fields, saliency fields, visibility-weighted saliency fields and weighted feature saliency. In order to better illustrate the effects of these techniques, we present the results of applying these techniques to an synthetic data set from two different viewpoints in our discussion.

\subsection{Visibility Fields \label{visibility_fields}}
Direct volume rendering (e.g. ray-casting) is a technique that renders a 2D projection of the 3D volume data set. The rendering of a volume, which essentially is a block of 3D data, involves alpha blending and color composition of voxels. The resulting 2D image is acquired by blending the color and opacity of voxels along the view direction. The transfer function determines the color and opacity of voxels based on their data attributes such as intensity. However, the contribution of a voxel to the rendered image is determined by both the opacity of this voxel and the opacity of those voxels in front of the current voxel in the view direction.
This mechanism is described in the front-to-back compositing equations \cite{emsenhuber_visibility_2008}.
\[
C_{i}=(1-A_{i-1})c_{i}+C_{i-1}
\addtag \]
\[
A_{i}=(1-A_{i-1})a_{i}+A_{i-1}
\addtag \]
where $ a_{i} $ and $ c_{i} $ are opacity and color of voxel $ i $, and $A_{i}$ and $C_{i}$ are the accumulated opacity and color at  voxel $ i $.

Therefore the visibility of voxel $ i $ can be calculated as
\[ v_{i}=A_{i}-A_{i-1}=(1-A_{i-1})a_{i} 
\addtag \]
and the visibility field is simply the visibility of all the voxels in the volume $ V $
\[ V=\set{v_{i}|i \in V} 
\addtag \]

The visibility field is dependent on both the viewpoint and the transfer function, therefore it can be used to analyze the structure of the volume data. The visibility field is particularly useful for understanding what parts of the data set are being rendered and how different tissues occlude each other (Figure~\ref{fig:disks_combined}).

\begin{figure}
	\centering
	\includegraphics[width=1\linewidth]{images/disks_combined}
	\caption{A synthetic volume data consists of three solid disk-like objects. The images in the first row shows a final image and the corresponding visibility field from a viewpoint on the left.
		The images in the second row shows the the final image and visibility field from a viewpoint on the right.
		The visibility fields display what parts of the volume contribute most to the images and how tissues in the front occlude those in the back.}
	\label{fig:disks_combined}
\end{figure}

In terms of implementation, the computation of visibility fields can be performed in real-time on a GPU.
Correa and Ma \cite{correa_visibility_2011} employed a scattering approach for GPU-assisted computation of the visibility histogram, which scatters the pixel points to the right bin in the histogram. Wang et al. \cite{wang_efficient_2011} used the multiple rendering targets (MRT) extension of OpenGL 2.0 and above to achieve the computation of visibility for up to 32 features.
Instead of grouping visibility values into intensity bins to acquire visibility distribution over intensity ranges (histograms), we are interested in the actual spatial visibility distribution, i.e. visibility field. We perform slice-based rendering on a GPU by rendering a series of quads which are parallel to the viewing plane, one for each slice. The fragments which do not belong to the volume are discarded. Then the visibility values are computed by subtracting the accumulated opacity of the previous slice from that of the current slice. After collecting the visibility values of all voxels, the visibility field can be constructed.

\subsection{Saliency Fields \label{saliency_fields}}
Because viewers pay greater visual attention to regions that they find salient \cite{palmer_vision_1999}, many models of visual attention and saliency have been evaluated by their ability to predict eye movements. The saliency for a volume can be computed either by using eye-tracking data or through computational models of human perception. Once the saliency for a volume is acquired, it can be used to better inform the visualization process.
%Based on the center-surround hypothesis that a salient region stands out from its surroundings \cite{koch_predicting_1999}, Kim and Varshney \cite{kim_saliency-guided_2006} introduced the saliency field, which is computed using a center-surround operator of the Laplacian of Gaussian-weighted averages.

We use a center-surround operator that is similar to the work by Shen et al. \cite{shen_spatiotemporal_2015} to compute the saliency field.
%define the We also uses the center-surround mechanism to compute saliency field of volume data.
Let the neighborhood $ N(i,\sigma) $ for a voxel $ i $ be the set of voxels within a distance $ \sigma $. Thus, $ N(i,\sigma)=\set{j|\|j-i\|<\sigma} $, where $ j $ is a voxel. Let $ G(O,i,\sigma) $ denote the Gaussian weighted average, then we have
\[ G(O,i,\sigma)=\sum_{j \in N(i,\sigma)} O(j) g(i,j,\sigma)
\addtag \]
where
\[ g(i,j,\sigma)=\frac{exp[-\|j-i\|^{2}/(2\sigma)^{2}]}{\sum_{k\in N(i,\sigma)} exp[-\|k-i\|]^{2}/(2\sigma)^{2}}
\addtag \]
and $ O $ is a field of appearance attributes of every voxel in the volume and $ O(j) $ is the appearance attribute of voxel $ j $.

Then the saliency field is defined as the absolute difference of Gaussian-weighted averages
\[ L(O,i,\sigma)=|w_{1}G(O,i,\sigma)-w_{2}G(O,i,2\sigma)|
\addtag \]
where $ w_{1} $ and $ w_{2} $ indicate the weights of the Gaussian-weighted averages at a fine scale and a coarse scale respectively.
%Positive weights emphasize the center and de-emphasize the surroundings and vice versa.

Visual properties such as opacity and color values (e.g. brightness, saturation, hue) can be used as appearance attributes in the computation of a saliency field.
Figure~\ref{fig:disks_saliency} displays the saliency fields computed from brightness and saturation of voxels respectively.
Although opacity is an important visual property, the visibility field described in the previous section is derived from alpha blending, which has taken the opacity of voxels into account. Therefore, we compute the saliency fields using brightness and saturation instead of opacity. Brightness and saturation are also the appearance attributes Kim and Varshney \cite{kim_saliency-guided_2006} used in their saliency-based enhancement operator.
The saliency field is acquired by applying the center-surround operator to appearance attributes of voxels.
The effect of the center-surround operator is to emphasize the center and de-emphasize the surroundings of voxels. This is shown in the exploded view of the saliency field (computed from brightness) and the volume data in Figure~\ref{fig:disks_saliency_half}.

In our implementation, we use perceptually uniform color spaces, e.g. CIELab and CIELCh.
In CIELCh, instead of Cartesian coordinates a*, b*, the cylindrical coordinates C* (chroma, relative saturation) and h (hue angle in the CIELab color wheel) are specified, and the brightness L* remains the same.
The advantage of using perceptually uniform color spaces is that the relative perceptual differences between two colors can be approximated by the Euclidean distance between the two colors in a three-dimensional space consisting of the three color components \cite{fairchild_color_2013}.

\begin{figure}
	\centering
	\begin{minipage}{.15\textwidth}
		\includegraphics[width=1\linewidth]{images/disks_saliency}
	\end{minipage}~
	\begin{minipage}{.15\textwidth}
		\includegraphics[width=1\linewidth]{images/disks_saliency_saturation}
	\end{minipage}
	\caption{The saliency fields computed from brightness (left) and saturation of voxels (right) respectively.}
	\label{fig:disks_saliency}
\end{figure}

\begin{figure}
	\centering
	\begin{minipage}{.15\textwidth}
		\includegraphics[width=1\linewidth]{images/disks_saliency_half}
	\end{minipage}~
	\begin{minipage}{.15\textwidth}
		\includegraphics[width=1\linewidth]{images/disks_half}
	\end{minipage}
	\caption{The saliency fields emphasize the center and de-emphasize the surroundings of voxels. As in the clipped views of the saliency field (left) and the volume data set (right), the three solid disks are represented as hollow shapes in the saliency field.}
	\label{fig:disks_saliency_half}
\end{figure}

\subsection{Visibility-Weighted Saliency Fields of Features \label{visibility_weighted_saliency}}
The visibility field indicates the contribution of voxels, which is how much each voxel contributes to the final image, and the saliency field indicates the conspicuity of voxels, which is how much each voxel stands out from its surroundings.
The conspicuity in a 3D volume is similar to that in an 2D image and can be measured by the difference of visible properties between each location (voxel) and its surroundings \cite{duan_visual_2011}.
It would be desirable to have an indicator that represents both the contribution and conspicuity of the voxels.
Therefore we propose a visibility-weighted saliency field, by weighting the saliency of voxels by their visibility, given the volume is rendered with a specific transfer function from a specific viewpoint.
The visibility-weighted saliency for voxel $ i $ is
\[ s_{i}(O,i,\sigma)= v_{i} L(O,i,\sigma)
\addtag \]
Hence we define $ S $ as the visibility-weighted saliency field of the volume $ V $.
\[ S=\set{s_{i}(O,i,\sigma)|i \in V} 
\addtag \]
Therefore we define visibility-weighted saliency field of a feature $ F $ in the volume $ V $ as ($ F\subseteq V $).
\[ S_{F}=\set{s_{i}(O,i,\sigma)|i \in F} 
\addtag \]

Then we define visibility-weighted saliency of feature $ F $ as
\[ W_{F}(O,i,\sigma)=\frac{\sum_{i \in F}{s_{i}(O,i,\sigma)}}{\sum_{i \in V}{s_{i}(O,i,\sigma)}} 
\addtag \]
Since $ F $ is a subset of $ V $, $ W_{F}(O,i,\sigma) $ must be in the interval $ [0,1] $.
$ S_{F} $ can be used as a score to indicate the saliency of feature $ F $ in terms of the appearance attribute $ O $.

Features can be defined by user-specified transfer functions or segmentation of the volume data.
Figure~\ref{fig:disks_visibility_saliency_fields} illustrates the visibility-weighted saliency fields of the three disk-like features.

\subsection{Visibility-Weighted Feature Saliency Histograms \label{weighted_feature_saliency}}
As mentioned in Section~\ref{saliency_fields}, the saliency field can be computed from different appearance attributes.
Multiple saliency fields computed from different appearance attributes can be combined together in order to represent different aspects of the visual saliency of voxels.
In our implementation, we use brightness and saturation respectively to compute visibility-weighted saliency fields and define the weighted sum of the two sets of feature saliency as visibility-weighted feature saliency.

%Similar to the visibility-weighted saliency in the previous section, the visibility-weighted saliency from brightness for voxel $ i $ is
%$ W_{F}(O_{b},i,\sigma) $ $ W_{F}(O_{s},i,\sigma) $
%\[ s_{i}(O_{s},i,\sigma)= v_{i} L(O_{b},i,\sigma)\]
%and the visibility-weighted saliency from saturation for voxel $ i $ is
%\[ s_{i}(O_{s},i,\sigma)= v_{i} L(O_{s},i,\sigma)\]
%Then we define the mean feature saliency as mean of the visibility-weighted saliency values computed using brightness and saturation respectively

\[ W_{F}=u_{1}W_{F}(O_{b},i,\sigma)+u_{2}W_{F}(O_{s},i,\sigma)
\addtag \]
where $ u_{1}$ and $ u_{2}$ are weights of different appearance attributes. $ u_{1}$ and $ u_{2}$ are both in the interval $[0,1] $ and $ u_{1}+u_{2}=1 $. $ W_{F}(O_{b},i,\sigma) $ is the visibility-weighted saliency of feature $ F $ computed using brightness of voxels and similarly $ W_{F}(O_{s},i,\sigma) $ is the visibility-weighted saliency of feature $ F $ from saturation of voxels.

Figure~\ref{fig:disks_saliency_chart_left} and Figure~\ref{fig:disks_saliency_chart_right} display bar charts of our visibility-weighted saliency of the three features and the feature visibility by Wang et al. \cite{wang_efficient_2011} for comparison.
We compute the saliency fields using brightness and saturation respectively and thus acquire two sets of feature saliency of the three features in the synthetic data set.
In Figure~\ref{fig:disks_saliency_chart_left} and Figure~\ref{fig:disks_saliency_chart_right}, the feature saliency from brightness shows similar patterns as the feature visibility. However, the feature saliency from saturation gives the highest score to the middle disk (magenta color), which means the middle disk is significantly more salient than the other two (light green and dark green) in terms of saturation. The visibility-weighted feature saliency combines the feature saliency from brightness and saturation with user-specified weights. 
%which combines the saliency fields derived from brightness and saturation of voxels.
The visibility-weighted feature saliency can be used as a measure to indicate the saliency of features in volume rendered images.
The weights of different appearance attributes should be adjusted according to the user's need in specific tasks.

\begin{figure}
	\centering
	\begin{minipage}{.3\textwidth}
		\includegraphics[width=1\linewidth]{images/disks_visibility_saliency_feature1_left}
	\end{minipage}~
	\begin{minipage}{.3\textwidth}
		\includegraphics[width=1\linewidth]{images/disks_visibility_saliency_feature2_left}
	\end{minipage}~
	\begin{minipage}{.3\textwidth}
		\includegraphics[width=1\linewidth]{images/disks_visibility_saliency_feature3_left}
	\end{minipage}
	\begin{minipage}{.3\textwidth}
		\includegraphics[width=1\linewidth]{images/disks_visibility_saliency_feature1_right}
	\end{minipage}~
	\begin{minipage}{.3\textwidth}
		\includegraphics[width=1\linewidth]{images/disks_visibility_saliency_feature2_right}
	\end{minipage}~
	\begin{minipage}{.3\textwidth}
		\includegraphics[width=1\linewidth]{images/disks_visibility_saliency_feature3_right}
	\end{minipage}
	\caption{Visibility-weighted saliency fields of the three disks. The first column shows the saliency fields of the top disk in the two viewpoints in Figure~\ref{fig:disks_combined}. The second and the third columns show the saliency fields for the middle disk and the bottom disk in the two viewpoints respectively.}
	\label{fig:disks_visibility_saliency_fields}
\end{figure}

\begin{figure}
	\centering
	\begin{minipage}{.45\textwidth}
		\includegraphics[width=1\linewidth]{images/disk_visibility_chart_left.png}
	\end{minipage}~
	\begin{minipage}{.45\textwidth}
		\includegraphics[width=1\linewidth]{images/disk_visibility_saliency_brightness_chart_left.png}
	\end{minipage}
	\begin{minipage}{.45\textwidth}
		\includegraphics[width=1\linewidth]{images/disk_visibility_saliency_saturation_chart_left.png}
	\end{minipage}~
	\begin{minipage}{.45\textwidth}
		\includegraphics[width=1\linewidth]{images/disk_visibility_saliency_weighted_chart_left.png}
	\end{minipage}
	\caption{The feature visibility \cite{wang_efficient_2011} histogram at top-left shows the sum of visibility values of all the voxels belong to each feature.
		The two histograms of visibility-weighted feature saliency from brightness and saturation respectively (at top-right and at bottom-left) show the sum of visibility-weighted saliency of all the voxels belong to each feature ($ W_{F}(O,i,\sigma) $ in Section~\ref{visibility_weighted_saliency}).
		The histogram of weighted feature saliency (at bottom-right) shows the weighted feature saliency with equal weights of brightness and saturation ($ W_{F} $ in Section~\ref{weighted_feature_saliency}).
		Feature visibility (top-left) and visibility-weighted feature saliency from brightness (top-right) both suggest that the top disk is the most visible and the bottom disk is the least visible.
		However, the middle disk with magenta color is significantly more salient than the other two disks (light green and dark green) in terms of saturation (the histogram at bottom-left).
	}
	\label{fig:disks_saliency_chart_left}
\end{figure}

\begin{figure}
	\centering
	\begin{minipage}{.45\textwidth}
		\includegraphics[width=1\linewidth]{images/disk_visibility_chart_right.png}
	\end{minipage}~
	\begin{minipage}{.45\textwidth}
		\includegraphics[width=1\linewidth]{images/disk_visibility_saliency_brightness_chart_right.png}
	\end{minipage}
	\begin{minipage}{.45\textwidth}
		\includegraphics[width=1\linewidth]{images/disk_visibility_saliency_saturation_chart_right.png}
	\end{minipage}~
	\begin{minipage}{.45\textwidth}
		\includegraphics[width=1\linewidth]{images/disk_visibility_saliency_weighted_chart_right.png}
	\end{minipage}
	\caption{Similar to Figure~\ref{fig:disks_saliency_chart_left}, in the viewpoint on the right in Figure~\ref{fig:disks_combined}, the bottom disk is the most visible according to feature visibility (histogram at top-left) and most salient according to feature saliency from brightness (histogram at top-right). However, feature saliency from saturation (histogram at bottom-left) suggests that the middle disk (magenta color) is significantly more salient than the other two (light green and dark green) in terms of saturation. The histogram at bottom-right is the weighted feature saliency with equal weights of brightness and saturation.}
	\label{fig:disks_saliency_chart_right}
\end{figure}

%\begin{figure}
%\centering
%\begin{minipage}{.22\textwidth}
%\includegraphics[width=1\linewidth]{images/disk_visibility_saliency_weighted_chart_left.png}
%\end{minipage}~
%\begin{minipage}{.22\textwidth}
%\includegraphics[width=1\linewidth]{images/disk_visibility_saliency_weighted_chart_right.png}
%\end{minipage}
%\caption{Weighted feature saliency of the three features in the left and the right viewpoint in Figure~\ref{fig:disks_combined}. The left bar chart is the mean of the middle and the right bar charts in Figure~\ref{fig:disks_saliency_chart_left} and the right bar chart is the mean of those two in Figure~\ref{fig:disks_saliency_chart_right}.}
%\label{fig:disks_mean_saliency_chart}
%\end{figure}

%-------------------------------------------------------------------------
\section{Use Case: Measure Feature Saliency with Different Transfer Functions}
In this section, we present results of using our approach to measure visual saliency of features of a volume data set with two different transfer functions.

A tooth data set \cite{website:Roettger_volume_2013} is rendered with two different transfer functions to demonstrate the effectiveness of our approach. The first transfer function (Figure~\ref{fig:tooth}) assigns equal opacity to the three features.
The second transfer function (Figure~\ref{fig:tooth_2}) is designed to emphasize the enamel (the yellow material), thus it assigns high opacity the enamel and low opacity to the other two features (cementum \& pulp chamber and dentine).

By observation, it is clear that the transfer function in Figure~\ref{fig:tooth_2} is better in terms of visualizing the enamel than Figure~\ref{fig:tooth}. The purpose of our approach is to provide an automated objective measure to make this comparison. This is demonstrated through the output of the visibility-weighted salience fields of the two transfer functions (Figure~\ref{fig:tooth_saliency_field} and Figure~\ref{fig:tooth_saliency_field_2}).

In the visibility-weighted saliency fields of the first transfer function (Figure~\ref{fig:tooth_saliency_field}), all three features are reasonably salient.
%The visibility-weighted saliency fields display the contribution of three features to the volume rendered image respectively. In Figure~\ref{fig:tooth}, the material outside occludes the enamel inside, this can be seen from the visibility-weighted saliency fields in Figure~\ref{fig:tooth_saliency_field}.
%On the other hand, in Figure~\ref{fig:tooth_2},
% the two outer features have low opacity and the enamel inside has high opacity.
On the other hand, the visibility-weighted saliency fields of the second transfer function (Figure~\ref{fig:tooth_saliency_field_2}) suggest the enamel has significantly higher visual saliency in the volume rendered image.
%On the other hand, in Figure~\ref{fig:tooth_2}
The feature visibility and weighted feature saliency in Figure~\ref{fig:tooth_saliency_chart} and Figure~\ref{fig:tooth_saliency_chart_2} summarize the visibility and visual saliency of the three features specified by the transfer functions.

\begin{figure}
	\centering
	\begin{minipage}{.2\textwidth}
		\includegraphics[width=1\linewidth]{images/tooth.png}
	\end{minipage}~
	\begin{minipage}{.1\textwidth}
		\includegraphics[width=1\linewidth]{images/tooth_tf.png}
	\end{minipage}
	\caption{A tooth data set with a transfer function revealing three features: cementum \& pulp chamber (blue), dentine (red) and enamel (yellow). Equal opacity is assigned to the three features in the transfer function.}
	\label{fig:tooth}
\end{figure}

\begin{figure}
	\centering
	\begin{minipage}{.15\textwidth}
		\includegraphics[width=1\linewidth]{images/tooth_visibility_saliency_feature1.png}
	\end{minipage}~
	\begin{minipage}{.15\textwidth}
		\includegraphics[width=1\linewidth]{images/tooth_visibility_saliency_feature2.png}
	\end{minipage}~
	\begin{minipage}{.15\textwidth}
		\includegraphics[width=1\linewidth]{images/tooth_visibility_saliency_feature3.png}
	\end{minipage}
	\caption{Visibility-weighted saliency fields of the three features, computed with the transfer function in Figure~\ref{fig:tooth}. From left to right, the features are cementum \& pulp chamber, dentine and enamel.}
	\label{fig:tooth_saliency_field}
\end{figure}

\begin{figure}
	\centering
	\begin{minipage}{.2\textwidth}
		\includegraphics[width=1\linewidth]{images/tooth_visibility_chart.png}
	\end{minipage}~
	%\begin{minipage}{.2\textwidth}
	%\includegraphics[width=1\linewidth]{images/tooth_visibility_saliency_brightness_chart.png}
	%\end{minipage}
	%\begin{minipage}{.2\textwidth}
	%\includegraphics[width=1\linewidth]{images/tooth_visibility_saliency_saturation_chart.png}
	%\end{minipage}~
	\begin{minipage}{.2\textwidth}
		\includegraphics[width=1\linewidth]{images/tooth_visibility_saliency_weighted_chart.png}
	\end{minipage}
	\caption{Feature visibility (darker histogram on the left) and weighted feature saliency (brighter histogram on the right) of the three features, computed with the transfer function in Figure~\ref{fig:tooth}.}
	\label{fig:tooth_saliency_chart}
\end{figure}

\begin{figure}
	\centering
	\begin{minipage}{.2\textwidth}
		\includegraphics[width=1\linewidth]{images/tooth_balance.png}
	\end{minipage}~
	\begin{minipage}{.1\textwidth}
		\includegraphics[width=1\linewidth]{images/tooth_balance_tf.png}
	\end{minipage}
	\caption{A tooth data set with a transfer function particularly highlighting the enamel (yellow)}
	\label{fig:tooth_2}
\end{figure}

\begin{figure}
	\centering
	\begin{minipage}{.15\textwidth}
		\includegraphics[width=1\linewidth]{images/tooth_2_visibility_saliency_feature1.png}
	\end{minipage}~
	\begin{minipage}{.15\textwidth}
		\includegraphics[width=1\linewidth]{images/tooth_2_visibility_saliency_feature2.png}
	\end{minipage}~
	\begin{minipage}{.15\textwidth}
		\includegraphics[width=1\linewidth]{images/tooth_2_visibility_saliency_feature3.png}
	\end{minipage}
	\caption{Visibility-weighted saliency field of the three features, computed with the transfer function in Figure~\ref{fig:tooth_2}. From left to right, the features are cementum \& pulp chamber, dentine and enamel.}
	\label{fig:tooth_saliency_field_2}
\end{figure}

\begin{figure}
	\centering
	\begin{minipage}{.2\textwidth}
		\includegraphics[width=1\linewidth]{images/tooth_2_visibility_chart.png}
	\end{minipage}~
	%\begin{minipage}{.2\textwidth}
	%\includegraphics[width=1\linewidth]{images/tooth_2_visibility_saliency_brightness_chart.png}
	%\end{minipage}
	%\begin{minipage}{.2\textwidth}
	%\includegraphics[width=1\linewidth]{images/tooth_2_visibility_saliency_saturation_chart.png}
	%\end{minipage}~
	\begin{minipage}{.2\textwidth}
		\includegraphics[width=1\linewidth]{images/tooth_2_visibility_saliency_weighted_chart.png}
	\end{minipage}
	\caption{Feature visibility (darker histogram on the left) and weighted feature saliency (brighter histogram on the right) of the three features, computed with the transfer function in Figure~\ref{fig:tooth_2}.}
	\label{fig:tooth_saliency_chart_2}
\end{figure}

%-------------------------------------------------------------------------
\section{Experiment}
To judge the performance of the proposed approach, a human study was conducted to construct a subjective data set for assessing visual saliency of features in volume visualization as perceived by human users. The aim of this set of experiments is to gather human rating of visual saliency data and gather eye tracking data to evaluate and improve a proposed computational visual saliency metric for volume visualization.

\subsection{Source Images and Participants}
This section describes the human study and experiments performed using it.
The images used for the study were rendered by Voreen \cite{meyer-spradow_voreen:_2009} with a variety of transfer functions highlighting different features in various viewpoints.
** females and ** males participated in the experiment all aged between ** and ** years.

\subsection{Methods and Measurements}
Participants would sit in front of a computer display viewing images generated from volume visualization. The first part of the experiment would investigate how participants perceive the visual saliency of different objects in the images. The participants’ task would be to score the images on the scale of 1 to 5 by keyboard input. In the next part of the experiment, the participants would be asked to score the quality (in terms of sharpness and contrast) of the images (also by keyboard input) and their eye movements would be tracked by a head mounted eye tracker (EyeLink II by SR Research). In both parts of the experiment, each image would be shown to the participant for approximately 10 seconds. The experiment would last approximately 30 minutes. There would be a break scheduled midway through the experiment to allow the participant to rest.

%-------------------------------------------------------------------------
\section{Conclusions}
In this paper, we propose visibility-weighted saliency as an improved measure of the visual saliency of features in volume rendered images, in order to assist users in choosing suitable viewpoints and designing effective transfer functions to visualize the features of interest.
%The visibility-weighted saliency are presented in two different ways, i.e. visibility-weighted saliency fields and feature saliency histograms. Visibility-weighted saliency fields display the spatial distribution of visual saliency of features and feature saliency histograms provide quantitative information about the perceptual importance of the features.
With visibility-weighted saliency, the saliency of features rendered in different viewpoints with different transfer functions can be measured in a quantitative and fully automated way.
In future work we plan to validate our work by conducting an eye-tracking-based user study. In addition, we would like to study other appearance attributes apart from brightness and saturation, and study the weighting between these attributes.

\chapter{Transfer Function Optimization Using Visibility-Weighted Saliency \label{transfer_function_optimization}}
In this chapter, we present a transfer function optimization approach using the visibility-weighted saliency metric discussed in Chapter~\ref{visibility-weighted_saliency}.
This is an automated approach that adjusts transfer functions to match the visibility-weighted saliency towards user-specified targets.
In addition, a parallel line search strategy is presented for exploiting the computing power of multi-core processors to improve the performance of the transfer function optimization approach.

\section{Introduction}
Volume visualization is an effective means of discovering meaningful features in volume data sets.
Both exterior and interior of structures can be revealed simultaneously in a semi-transparent manner by specifying opacity values for the features in transfer functions \cite{wang_efficient_2011}.
Features could include intensity intervals in 1D transfer functions, rectangular or other shapes in 2D or higher-dimensional transfer functions.

In the specification of transfer functions for volume visualization, users often have a rough idea of how clear and opaque each feature should be and then adjust the opacity value of the features accordingly.
However, the relationship between the opacity of features and the saliency of the features in the final image is not linear.
The saliency of a feature in the final image depends on the opacity value assigned to the feature as well as the neighborhood of the feature and view-dependent occlusion of the feature.

%Therefore, it is desirable to have an automated method to assist the user in the design of transfer functions. In this chapter, we propose an optimization approach to automatically refine a user-defined transfer function towards target saliency levels specified by the user.

Therefore, it is desirable to have an automated method to assist the user in the design of transfer functions that match target saliency levels specified by the user. In this chapter, we propose an optimization approach that supports this requirement by automatically refining a user-defined transfer function towards any given saliency distribution.
Moreover, we present a parallel line search strategy to improve the performance of the transfer function optimization.

\section{Related Work}
% line search \cite{armijo_minimization_1966}
% conjugate gradient descent \cite{shewchuk_introduction_1994}
Transfer function specification is a non-trivial and unintuitive task in volume visualization. Compared to typical transfer function approaches, which are often subjective, it is desirable to have objective feedback regarding the clarity of features in volume visualization.

%Based on the perceptual principles, Chen et al. \cite{chan_perception-based_2009} introduced several image quality measures to enhance the perceived quality of semitransparent features.
%J{\"a}nicke and Chen \cite{janicke_salience-based_2010} described a quality metric for analyzing the saliency of visualization images and demonstrated its usefulness with examples from information visualization, volume visualization and flow visualization.

Correa and Ma \cite{correa_visibility-driven_2009} introduced visibility histograms to guide transfer function design for both manual and automatic adjustment.
Visibility histograms (Figure~\ref{fig:correa_visibility-driven_2009}), which summarize the distribution of visibility of voxels from a given viewpoint, are a powerful feedback mechanism for volume visualization \cite{emsenhuber_visibility_2008}.
Wang et al. \cite{wang_efficient_2011} extended visibility histograms to feature visibility histograms, in order to measures the influence of each feature to the resulting images. They described a scheme that allows users to specify a desired visibility for features of interest and subsequently the opacity transfer function is optimized using an active set algorithm \cite{polyak_conjugate_1969}.

%\noindent\makebox[\linewidth]{\rule{\paperwidth}{0.4pt}}

Researchers have developed a variety of parallel strategies to accelerate sequential optimization algorithms \cite{spedicato_algorithms_2012}.
%\cite{koko_parallel_1998}
Phua et al. \cite{phua_parallel_1998} proposed a parallel extension to quasi-Newton methods \cite{yang_optimization_2001}. Their approach generates several search directions at each iteration and then applies different line search and scaling strategies in parallel along each search direction.
Peachey et al. \cite{peachey_parallel_2009} presented another approach to parallelize the quasi-Newton methods.
In their applications, the objective function evaluation typically requires minutes or hours of processing time. Therefore, they introduced an approach that evaluates the objective function in parallel over a cluster of computers and continues to the next iteration before all evaluations finish in order to accelerate convergence.

%-------------------------------------------------------------------------
\section{Method}
In Chapter~\ref{visibility-weighted_saliency}, visibility-weighted saliency was proposed as a measure of visual saliency of features in volume visualization. This metric indicates the perceptual importance of voxels and the visibility of features in volume rendered images and can be utilized to assist users in choosing suitable viewpoints and designing effective transfer functions to visualize the features of interest.
In this chapter, we describe a transfer function optimization approach based on the visibility-weighted saliency metric in order to automatically adjust the volume visualization to satisfy user-specified targets set on the visibility-weighted saliency for the features.

The approach described in Chapter~\ref{transfer_function_refinement} is an automated method of optimizing transfer functions, based on the intensity distribution of voxels in the volume data set. However, this approach does not take into account the spatial distribution of voxels and the viewpoint of the visualization. Visibility-weighted saliency, on the other hand, takes into account both of these two aspects. The visibility-weighted saliency consists of two component fields, i.e. saliency field and visiblity fields. Saliency fields are essentially difference of Gaussians, which include the information of local neighborhoods of voxels.
Visibility fields are computed from opacity contribution of voxels to volume rendered images, which indicate viewpoint dependent occlusions of the voxels.

Constraints are introduced in the search of the parameter space. Only the opacity of features are changed in the transfer function domain. The definition of features (e.g. intensity ranges on 1D transfer functions) and the colors of features remain the same.
%done the classification of features.
These constraints are based on the assumption that the user has explored the volume data and done the classification of features.
% set up the transfer function according to his/her needs.
Our approach aims to help the user adjust the saliency distribution and reduce occlusion while preserving the user's knowledge or judgments of the data set.
While the approach discussed in Chapter~\ref{transfer_function_refinement} was designed for exploration and visual search of volume data, the approach in this chapter could aid in analysis, understanding or closer inspection of the data.

%Should maybe say here and/or in the conclusions that (while Ch3 was for exploring, visual search of data) the contributions of this chapter could aid in analysis, understanding or closer inspection of the data

\subsection{Objective Function}
Users define target importance values for each feature in the transfer function domain.
Our transfer function optimizer adjusts the transfer function to match the visibility-weighted saliency with the user-defined target saliency values.
Multiple saliency fields computed from different appearance attributes can be combined together in order to represent different aspects of the visual saliency of voxels.
In our implementation, brightness and saturation are used respectively to compute visibility-weighted saliency fields and define the weighted sum of the two sets of feature saliency as visibility-weighted feature saliency.
The objective function $ F $ is defined as the root mean square of the differences of the visibility-weighted saliency and target importance of each feature.
\[ F=\sqrt{ \frac{\sum_{i=1}^{n} (W_{i}-t_{i})^{2}}{n} } 
\addtag \]
where $ W_{F}=u_{1}W_{F}(O_{b},i,\sigma)+u_{2}W_{F}(O_{s},i,\sigma) $ is the visibility-weighted saliency of feature $ i $, and $ t_{i} $ is the user-defined importance of feature $ i $. These user-defined saliency values are normalized and they add up to 1, in other words, $ t_{i} \in [0.1] $ and $ \sum_{i=1}^{n} t_{i} = 1 $.

As previously described in Section~\ref{weighted_feature_saliency}, multiple saliency fields computed from different appearance attributes can be combined together in order to represent different aspects of the visual saliency of voxels.
In our implementation, $ W_{F}=u_{1}W_{F}(O_{b},i,\sigma)+u_{2}W_{F}(O_{s},i,\sigma) $ is a weighted sum of visibility-weighted saliency values computed using brightness and saturation of voxels respectively, and $ u_{1} $ and $ u_{2} $ are weights of the two appearance attributes.

However, the visibility-weighted saliency $ W_{i} $ is not a variable that can be directly modified. Instead, $ W_{i} $ is a complicated function of the color and opacity of voxels in feature $ i $ and is also influenced by the viewpoint of rendering. A visibility-weighted saliency field is a combination of a visibility field and a saliency field. The saliency field is a view-independent field based on the color of every voxel in the volume data set, while the visibility field is a view-dependent field computed from the opacity contribution of every voxel to the final image when rendered from a certain viewpoint.

The computation of visibility fields is non-trivial. In order to compute a visibility field, a slice-based rendering is performed on a series of quads which are parallel to the viewing plane, one for each slice.
Subsequently, the visibility values are computed by subtracting the accumulated opacity of the previous slice from that of the current slice. After collecting the visibility values of all voxels, the visibility field can be constructed. The details of visibility fields were previously described in Section~\ref{visibility_fields}.

The evaluation of the objective function is computationally expensive. However, in an iterative optimization, the visibility field and visibility-weighted saliency have to be recomputed at each step after the feature opacity values are updated.

\subsection{Parameter Space}
We use a nucleon data set \cite{website:Voreen_datasets_2013} to demonstrate how the visibility-weighted saliency of features change when the feature opacity values change. As displayed in Figure~\ref{fig:nucleon_naive}, three features are defined in the transfer function for the nucleon data set.

The dimension of the parameter space is the same as the number of features defined by the user. In this case, three features are defined for the nucleon data set. The opacity of each feature is mapped to an axis in the parameter space. Therefore, the opacity values of the 3 features are mapped to $ x, y, z $ axes of a 3D scalar field respectively.
%Figure~\ref{fig:nucleon_densityplot} displays three 3D scalar fields, one for each feature, to provide an intuitive overview of the relationship between the feature opacity values and the visibility-weighted saliency values.
Figure~\ref{fig:nucleon_densityplot} displays a visualization of three 3D scalar fields representing this parameter space.
One field is presented for each feature to provide an intuitive overview of the relationship between the feature opacity values and the visibility-weighted saliency values. In order to avoid confusion, please note that this has no spatial relationship to the actual 3D volume data (which is itself a 3D scalar field).
%I suggest saying something like; Figure 5.2 displays a visualization of three 3D scalar fields representing this parameter space. One field is presented for each feature to provide an intuitive overview of ....  . In order to avoid confusion, please note that  this has no spatial relationship to the actual 3D volume data (which is itself a 3D scalar field).

Feature 1 (the purple structure in Figure~\ref{fig:nucleon_naive}) is the exterior of the nucleon data set, the visibility-weighted saliency of this feature is shown in the 3D scalar fields in the same color at the left in Figure~\ref{fig:nucleon_densityplot}.
The visibility-weighted saliency of Feature 1 increases as its opacity increases,
%as shown in the 3D scalar fields that the brightness and opacity increase along $ x $ axis.
as demonstrated by the fact that both the brightness and opacity of the corresponding 3D field increases along the $x$-axis.
Similar patterns also appear in the other 2 scalar fields, the visibility-weighted saliencies of Feature 2 (the red structure in Figure~\ref{fig:nucleon_naive}) and Feature 3 (the green structure in Figure~\ref{fig:nucleon_naive}) also increase as their opacity values increase.

%I Suggest: ", as demonstrated by the fact that both the brightness and opacity of the corresponding 3D field increases along the x-axis."

%Thus it is reasonable to assume that the visibility-weighted saliency of a feature is a monotonic function of its opacity.

Moreover, Feature 1 is the exterior of the nucleon, its visibility-weighted saliency is almost not influenced by the opacity of other features. On the other hand, the visibility-weighted saliency of Feature 2 is influenced by both the opacity of Feature 1 and Feature 2. In addition, the visibility-weighted saliency of Feature 3 is drastically influenced by the opacity of Feature 1, Feature 2 and Feature 3, as Feature 3 is an interior structure and can be easily occluded by the other 2 features.

In order to demonstrate the distribution of the objective function in the parameter space $ x \in [0,1] $, $ y \in [0,1] $ and $ z \in [0,1] $, we sample the parameter space with sampling interval $ 0.1 $, from 0 to 1 along each axis. There are 11 sampling points along each axis, which results in 1331 sampling points in the parameter space. In Figure~\ref{fig:nucleon_parameterspace}, the parameter space is rendered as a density plot with a color function resembling a temperate map which gradually changes from orange to blue.

\begin{figure}
\centering
\begin{minipage}{.24\textwidth}
	\includegraphics[width=1\linewidth]{figures/nucleon_naive_proportional}
	\subcaption{}
\end{minipage}~
\begin{minipage}{.15\textwidth}
	\includegraphics[width=1\linewidth]{figures/tf_nucleon_naive_proportional}
	\subcaption{}
\end{minipage}~
\begin{minipage}{.3\textwidth}
	\includegraphics[width=1\linewidth]{figures/nucleon_naive_proportional_visibility_chart}
	\subcaption{}
\end{minipage}~
\begin{minipage}{.3\textwidth}
	\includegraphics[width=1\linewidth]{figures/nucleon_naive_proportional_visibility_saliency_weighted_chart}
	\subcaption{}
\end{minipage}
\caption[A nucleon data set]{(a) A nucleon data set \cite{website:Voreen_datasets_2013}; (b) A user-specified transfer function with 3 features (c) The feature visibility histogram \cite{wang_efficient_2011}; (d) The visibility-weighted saliency histogram}
\label{fig:nucleon_naive}
\end{figure}

\begin{figure}
	\centering
	\begin{minipage}{.3\textwidth}
		\includegraphics[width=1\linewidth]{figures/nucleon_strong_red_densityplot1}
		\subcaption{VWS of feature 1}
	\end{minipage}~
	\begin{minipage}{.3\textwidth}
		\includegraphics[width=1\linewidth]{figures/nucleon_strong_red_densityplot2}
		\subcaption{VWS of feature 2}
	\end{minipage}
	\begin{minipage}{.3\textwidth}
		\includegraphics[width=1\linewidth]{figures/nucleon_strong_red_densityplot3}
		\subcaption{VWS of feature 3}
	\end{minipage}
	\caption[Visibility-weighted saliency of the 3 features are mapped to brightness and opacity respectively.]{Visibility-weighted saliency of the 3 features are mapped to brightness and opacity of the 3D fields in (a), (b) and (c) respectively. The visibility-weighted saliency of Feature 2 (red) is affected by both the opacity of Feature 1 (purple) and Feature 2. The visibility-weighted saliency of Feature 3 (green) is affected by the opacity of Feature 1, Feature 2 and Feature 3.}
	\label{fig:nucleon_densityplot}
\end{figure}

\begin{figure}
	\centering
	\begin{minipage}{.6\textwidth}
		\includegraphics[width=1\linewidth]{images/nucleon_strong_red_parameterspace}
	\end{minipage}
	\caption[Each position (x, y, z) in the parameter space represents 3 features with opacity values (x, y, z).]{Each position (x, y, z) in the parameter space represents 3 features with opacity values (x, y, z). The value of the objective function (with \{0.1, 0.3, 0.6\} as target) is mapped to the color in the parameter space (with sampling interval $ 0.1 $). For clarity, only the high and low values are visible and the data range in the middle is set to transparent.}
	\label{fig:nucleon_parameterspace}
\end{figure}

\subsection{Optimization Algorithm}
The gradient descent algorithm is employed in our transfer function optimizer.
Gradient descent is a first-order optimization algorithm. It is based on the observation that if a function $ f(x) $ is defined and differentiable in a neighborhood of a point $ x_{1} $, then $ f(x) $ decreases fastest in the direction of the negative gradient of the function \cite{chong_introduction_2013}.

Given a continuously differentiable function $ f(x) $ with $ x \in \mathbb{R}^{n} $, let $ x_{k} $ be the current iteration point and $ g_{k}=g(x_{k})= \nabla f(x_{k}) $ be the gradient of $ f(x) $ at $ x_{k} $. The gradient descent method defines the next iteration point by
\[ x_{k+1}=x_{k}- \alpha_{k} g_{k} , k \geq 0 \]
for $ \alpha_{k} $ small enough, then $ f(x_{k+1}) \leq f(x_{k}) $. The gradient varies as the iteration proceeds, tending to zero as it approaches a local minimum. When the gradient decreases, the iteration step sizes also decrease. So hopefully the sequence $ {x_{k}} $ converges to the desired local minimum after performing the iteration.

In gradient descent methods, we can either take very small step sizes and reevaluate the gradient at every step, or take large steps each time. If the step size is too small, it may end up in a laborious situation that the objective function converges very slowly. If the step size is too large, it results in a more zigzag path and may have the risk of missing the local minimum and thus cannot converge.

\subsection{Estimating Descent Directions \label{estimating_descent_directions}}
In each step of the optimization, the descent direction $ g_{k} $ has to be updated. 
As previously discussed, the visibility-weighted saliency of a feature increases as its feature opacity increases. 
%The visibility-weighted saliency of a feature is a monotonic function of the opacity of the feature. 
However, the relationship between the visibility-weighted saliency and the opacity of a feature also depends on the viewpoint of rendering and the spatial distribution of voxels of every feature in the volume data set. An exact derivative of the visibility-weighted saliency with respect to the opacity of the feature cannot be determined in advance.

In the following subsections, two methods for estimating descent directions are described.

\subsubsection{Gradients with Backward Difference}
The partial derivative of the objective function $ F $ with respect to $ x_{i} $ is
\[ \frac{\partial F}{\partial x_{i}} = \frac{\partial F}{\partial W_{i}} \frac{\partial W_{i}}{\partial x_{i}} \]
where $ x_{i} $ is the opacity and $ W_{i} $ is the visibility-weighted saliency of feature $ i $.
The partial derivative $ \frac{\partial F}{\partial W_{i}} $ can be solved from the objective function $ F $.
However, $ \frac{\partial W_{i}}{\partial x_{i}} $ cannot be determined without knowledge of the actual volume data set.

For a function $ f(x) $, its first-order derivative can be estimated by a backward difference divided by a small step.
\[ \frac{\nabla_{d}[f](x)}{d}=\frac{f(x)-f(x-d)}{d} \]
where $ d $ is a nonzero number.
When $ d $ is small, the backward difference divided by $ d $ approximates the derivative. Assuming that $ f $ is differentiable, the error in this approximation can be derived from Taylor's theorem.
\[ \frac{\nabla_{d}[f](x)}{d}-f'(x)=\mathcal{O}(d) \to 0 , \; as \; d \to 0 \]

The backward difference is used here to approximate $ \frac{\partial W_{i}}{\partial x_{i}} $, hence we have
\[ \frac{\partial W_{i}}{\partial x_{i}} \approx \frac{\nabla_{d}[W_{i}](x_{i})}{d} \]

The evaluation of the objective function $ F $ is very computationally expensive. In our implementation, a small step size $ d $ is adopted, therefore the backward difference can be calculated from values of the objective function, visibility-weighted saliency and steps of the previous iteration. In this case, no extra evaluation of the visibility-weighted saliency and the objective function is required.
%Similarly, the backward difference can calculated from previous values of the visibility-weighted saliency and step positions.

Furthermore, if the function $ W_{i} $ of $ x_{i} $ is approximately a linear function, the partial derivative $ \frac{\partial W_{i}}{\partial x_{i}} $ becomes constant and could be replaced by an empirical constant $ b_{i} $. In this case, the gradient of the objective function with respect to the visibility-weighted saliency is used instead, which should be more computationally efficient.
\[ \frac{\partial F}{\partial x_{i}} \approx \frac{\partial F}{\partial W_{i}} b_{i} \]

\subsubsection{Descent Directions with Second-Order Derivatives}
Newton's method, which is an iterative method for finding the roots of a differentiable function, can be used to find a minimum or maximum of a function. Because the derivative is zero at a minimum or maximum, minima and maxima can be found by applying Newton's method to the derivative.
\[ x_{k+1}=x_{k}- \frac{f'(x_{k})}{f''(x_{k})} \]
This iteration equation gives a similar form of gradient descent, thus $ \frac{\partial^2 F}{\partial x_{i}^2} $ can be adopted in our optimization algorithm as the descent direction.

The first-order derivative can be estimated by backward difference of the objective function $ F $, and the second-order derivative $ F'' $ can be estimated by backward difference of the first-order derivative $ F' $.
\[ \frac{\partial^2 F}{\partial x_{i}^2} 
\approx \dfrac{ \frac{\nabla_{d}[F](x_{i})}{d} }{ \frac{\nabla_{d}[F'](x_{i})}{d} }
= \frac{ \nabla_{d}[F](x_{i}) }{ \nabla_{d}[F'](x_{i}) } \]

\subsubsection{Comparison of Descent Directions}
We have tested the descent directions discussed above with several volume data sets with various step sizes.
All the above described methods worked with small step sizes. As the step size increases, using $ \frac{\partial F}{\partial W_{i}} \frac{\nabla_{d}[W_{i}](x_{i})}{d} $ as descent directions would be unstable. While using $ \frac{\partial F}{\partial W_{i}} b_{i} $ and $ \frac{ \nabla_{d}[F](x_{i}) }{ \nabla_{d}[F'](x_{i}) } $ as descent directions would be still stable even when the step sizes are large. In our implementation, $ b_{i}=1 $ is used and this yields desirable results.

Gradient descent was performed on several volume data sets with various transfer functions, using the two normalized descent directions computed from $ [ \frac{\partial F}{\partial W_{1}} b_{1} ... \frac{\partial F}{\partial W_{n}} b_{n} ] $ (Method 1)
 and 
$ [ \frac{ \nabla_{d}[F](x_{1}) }{ \nabla_{d}[F'](x_{1}) }  ... \frac{ \nabla_{d}[F](x_{n}) }{ \nabla_{d}[F'](x_{n}) } ] $ (Method 2).
In the preliminary results, the convergence speeds of the two methods were very similar. %, with Method 2 slightly faster in some cases.
Figure~\ref{fig:nucleon_naive_tooth_naive_rms} shows that the steps made by the two methods overlap most of the time.

\begin{figure}
	\centering
	\begin{minipage}{.49\textwidth}
		\includegraphics[width=1\linewidth]{figures/nucleon_naive_proportional_rms_fixed_newton}
		\subcaption{}
	\end{minipage}~
	\begin{minipage}{.49\textwidth}
		\includegraphics[width=1\linewidth]{figures/tooth_naive_rms_fixed_newton}
		\subcaption{}	
	\end{minipage}
%	\begin{minipage}{.49\textwidth}
%		\includegraphics[width=1\linewidth]{images/nucleon_naive_table_fixed}
%		\subcaption{}	
%	\end{minipage}~
%	\begin{minipage}{.49\textwidth}
%		\includegraphics[width=1\linewidth]{images/nucleon_naive_table_newton}
%		\subcaption{}	
%	\end{minipage}
	\caption[The two methods for estimating descent directions]
	{The two methods for estimating descent directions are applied to transfer function optimization on the nucleon data set (a) and a tooth data set (b) respectively. Note that the steps of the two methods overlap most of the time.
%		with a naive transfer function (assigning equal opacity to each feature).
%		Method 1 converges ($F<0.03$) at step 10 (as in the iteration log (c)) and method 2 converges at step 9 (as in the iteration log (d)) on the nucleon data set with a naive transfer function (assigning equal opacity to each feature).
		}
	\label{fig:nucleon_naive_tooth_naive_rms}
\end{figure}

\begin{figure}
	\centering
	\begin{minipage}{.9\textwidth}
		\includegraphics[width=1\linewidth]{images/nucleon_strong_red_parameterspace_path}
	\end{minipage}
	\caption[The steps of gradient descent methods with fixed step size and adaptive step size are shown in the parameter space]{The steps of gradient descent methods with fixed step size and adaptive step size are shown in the parameter space in Figure~\ref{fig:nucleon_parameterspace} (the step size is 0.1)}
	\label{fig:nucleon_parameterspace_path}
\end{figure}

\subsection{Line Search and Parallel Line Search}
Performing gradient descent with a small step size may result in converging too slowly and require a lot of evaluations of the objective function, which is rather expensive to compute in our situation. Various approaches have been proposed regarding the choices of step sizes, which lead to various gradient algorithms \cite{yuan_step-sizes_2008}.

\subsubsection{Line Search}
The line search strategy is an iterative approach that adapts the step size in gradient descent in order to achieve a reduction in the objective function while still making sufficiently fast progress.
\[ h( \gamma)=f(x_{k}+\gamma g_{k}) \]
where $ g_{k} $ is the descent direction and $ x_{k} $ is the current point at the $ k$-$th $ iteration.

There are two type of approaches of line search, exact line search and inexact line search \cite{vrahatis_class_2000}.
Exact line search chooses the next iteration point by achieving the least objective function value. However, despite the optimal properties, exact line search often behaves poorly and tends to zigzag in two orthogonal directions, which usually implies deteriorations in convergence \cite{zhou_gradient_2006}.
In contrast, inexact line search only loosely finds a sufficient decrease of the objective function along the descent direction.

The inexact line search we used in our implementation is as follows.

\begin{enumerate}
	\item Set initial iteration count $ i=0 $ and set $ n $ to the maximum iteration count.
	\item Check whether $ f(x_{k}+\gamma_{i+1} g_{k}) < f(x_{k}+\gamma_{i} g_{k}) $ where $ \gamma_{i}=2^{i} $
	\item If so and $ i<n-1 $, $ i=i+1 $ and repeat Line 2, otherwise terminate the line search.
\end{enumerate}

The step size $ \gamma_{n} $ is chosen after the above line search procedure.
This strategy does not find the exact minimum along the line direction, instead it yields reasonable results and descends much faster than using fixed step sizes.

With line search approaches, the optimization algorithm can converge much faster than using fixed step sizes. Figure~\ref{fig:nucleon_parameterspace_path} displays the paths of two gradient descent methods, one progresses regularly with fixed step sizes, the other progresses aggressively with line search. It takes fewer steps for the latter to reach a local minimum.

\subsubsection{Parallel Line Search}
The classical gradient descent is a sequential algorithm. In its iterative procedure, the next iteration takes the result from the previous iteration as input.
However, the line search at each iteration can be computed in parallel to accelerate the optimization. This idea is particularly useful in our transfer function optimization, because the most expensive computation in our transfer function optimization is the evaluation of visibility-weighted saliency, which is required in the evaluation of the objective function at each iteration.

In this subsection, we propose a parallel line search strategy, which evaluates the objective function at different candidate points in parallel along the line search direction. Instead of sequentially searching for a desirable step size, parallel line search simultaneously evaluates the candidate step sizes and choose the best one for the current iteration.
With this parallel approach, the computing power of modern multi-core processors can be better exploited to accelerate the transfer function optimization. Specifically, parallel line search launches multiple threads to perform the line search. Each thread computes the visibility-weighted saliency and the objective function at a candidate point. Subsequently, the results at all the candidate points are aggregated and the candidate point with the minimum objective function value is chosen as the next step.

The parallel line search is shown as follows.

\begin{enumerate}
	\item Generate a list of step sizes $ S= \{ \gamma_{0},\gamma_{1},...,\gamma_{n-1} \} $ where $ \gamma_{i}=2^{i} $
	\item Evaluate $ f(x_{k}+\gamma_{i} g_{k}) $ in parallel for each $ \gamma_{i} $ in $ S $
	\item Find the index $ i $ of the minimum $ f(x_{k}+\gamma_{i} g_{k}) $, then $ \gamma_{i} $ is the chosen step size.
\end{enumerate}

The mechanism of the parallel line search is sightly different from the sequential line search. In the sequential line search, if the current candidate point does not meet the condition, the line search is terminated and the next candidate point would not be evaluated. By contrast, the parallel line search would always evaluate all the candidate points and pick the one with least value of the objective function. However, these two methods would have the same behavior if the objective function is a convex function.

The parallel line search strategy would introduce extra overhead of starting and terminating threads. Moreover, the number of threads should not exceed the number of cores of the processor, otherwise multiple threads have to share the same core and this would cause performance impact. The parallel line search is beneficial only when the evaluation of the objective function is more expensive than the parallel overhead.
In our case, the evaluation of the objective function is very computationally expensive. It requires computing the visibility fields, which in turn requires that a pass of slice-based volume rendering is performed.

Figure~\ref{fig:nucleon_naive_tooth_naive_rms_linesearch} displays results of applying the two line search methods in optimizing transfer functions. The two curves of objective functions are mostly overlapping, which indicates the two methods acts almost the same in choosing step sizes.

\begin{figure}
	\centering
	\begin{minipage}{.5\textwidth}
		\includegraphics[width=1\linewidth]{figures/nucleon_naive_proportional_rms_fixed_linesearch_parallel}
		\subcaption{}
	\end{minipage}~
	\begin{minipage}{.5\textwidth}
		\includegraphics[width=1\linewidth]{figures/tooth_naive_rms_fixed_linesearch_parallel}
		\subcaption{}
	\end{minipage}
	\begin{minipage}{.5\textwidth}
		\includegraphics[width=1\linewidth]{figures/CT-Knee_naive_rms_fixed_linesearch_parallel}
		\subcaption{}
	\end{minipage}~
	\begin{minipage}{.5\textwidth}
		\includegraphics[width=1\linewidth]{figures/vortex_naive_proportional_rms_fixed_linesearch_parallel}
		\subcaption{}
	\end{minipage}	
	\caption[The line search and parallel line search]{The line search and parallel line search are applied to transfer function optimization on the nucleon data set (a), the tooth data set (b), the CT-Knee data set (c) and the first time step of the vortex data set (d) respectively. Note the curves of line search and parallel line search overlap most of the time, as they made the same step choices in most cases.}
	\label{fig:nucleon_naive_tooth_naive_rms_linesearch}
\end{figure}

%\begin{figure}
%	\centering
%	\begin{minipage}{.49\textwidth}
%		\includegraphics[width=1\linewidth]{figures/CT-Knee_naive_rms_fixed_linesearch_parallel}
%		\subcaption{}
%	\end{minipage}~
%	\begin{minipage}{.49\textwidth}
%		\includegraphics[width=1\linewidth]{figures/vortex_naive_proportional_rms_fixed_linesearch_parallel}
%		\subcaption{}
%	\end{minipage}
%	\caption[The line search and parallel line search are applied to transfer function optimization on the CT-Knee data set and the first time step of the vortex data set respectively.]{The line search and parallel line search are applied to transfer function optimization on the CT-Knee data set (a) and the first time step of the vortex data set (b) respectively. Note the two methods made very similar choices in step sizes.}
%	\label{fig:CT-Knee_naive_vortex_naive_rms_linesearch}
%\end{figure}

%-------------------------------------------------------------------------
\section{Results and Discussions}
In this section, we present some results to demonstrate the effectiveness of our approach on the nucleon data set ($ 41 \times 41 \times 41 $), a tooth data set ($ 140 \times 120 \times 161 $), a CT-knee (379 $ \times $ 229 $ \times $ 305) data set \cite{website:Roettger_volume_2013} and one time-step of a simulated turbulent vortex flow (128 $\times$ 128 $\times$ 128, 100 time-steps) \cite{website:Ma_repository_2013}.
%and four volumes of a horse embryo data set, which are the horse embryo at 35, 37, 39 and 42 days.
Results were obtained in our experimental programs written in Wolfram Mathematica 11 on a computer equipped with an Intel Xeon E3-1246 v3 processor, 16GB of RAM and a NVIDIA Quadro K4200 graphics card.
%Tests were performed on a computer equipped with an Intel Core i5-2410M CPU, 8GB of RAM and a NVIDIA GeForce GT 540M graphics card.

In the results, we examine the transfer functions and volume rendered images before and after transfer function optimization, as well as the evolution of visibility-weighted saliency and opacity values of features due to the optimization.
%In the results, we display the transfer functions and volume rendered images before and after transfer function optimization, as well as the evolution of visibility-weighted saliency and opacity values of features during the iterations.
The feature visibility histograms \cite{wang_efficient_2011} are displayed along with the visibility-weighted saliency histograms for comparison.

The two optimization methods shown in the results are gradient descent with fixed step size and gradient descent with parallel line search, both using the Method 1 discussed in Section~\ref{estimating_descent_directions} for estimating descent directions.

Figure~\ref{fig:nucleon_naive_optimized} shows the optimization results of Figure~\ref{fig:nucleon_naive}. In Figure~\ref{fig:nucleon_naive_optimized} (a), the three features from outside to inside appear in different transparency levels, from weak to strong. This reveals a clear perspective of the three structures.
Figure~\ref{fig:nucleon_naive_optimized} (b) to (d) are the optimized transfer function, the feature visibility histogram and the visibility-weighted saliency histogram respectively.
Figure~\ref{fig:nucleon_naive_optimized} (e) and (f) are the evolution of visibility-weighted saliency of each feature in gradient descents with fixed step sizes and parallel line search respectively. Figure~\ref{fig:nucleon_naive_optimized} (g) and (h) show the evolution of the opacity of each feature (opacity of the peak control point) in gradient descents with fixed step sizes and parallel line search respectively.

Figure~\ref{fig:tooth_naive_optimized} (a) to (h) display the volume rendered image and the transfer function, the feature visibility histogram and the visibility-weighted saliency histogram of the tooth data set before and after optimization respectively.
Figure~\ref{fig:tooth_naive_optimized} (i) and (j) are the evolution of visibility-weighted saliency of each feature in the gradient descent with fixed step sizes and parallel line search respectively. Figure~\ref{fig:tooth_naive_optimized} (k) and (l) illustrate the opacity of features in the gradient descent with fixed step sizes and parallel line search respectively.

Figure~\ref{fig:CT-Knee_naive_optimized} ((a) to (h)) shows the volume rendered images of a CT-Knee data set, the transfer functions, and the feature visibility histograms and the visibility-weighted saliency histograms before and after optimization respectively.

Figure~\ref{fig:vortex_naive_optimized} ((a) to (h)) shows the volume rendered images of the first time-step of a vortex data set, the transfer functions, and the feature visibility histograms and the visibility-weighted saliency histograms before and after optimization respectively.

The evolution of visibility-weighted saliency and opacity of each feature, in gradient descents with fixed step sizes and parallel line search, are displayed in Figure~\ref{fig:CT-Knee_naive_optimized} (i) to (l) and Figure~\ref{fig:vortex_naive_optimized} (i) to (l) for the CT-Knee data set and the vortex data set respectively.

In the optimized results of the 4 data sets, the visibility-weighted saliency values of the 3 features are very close to the user-specified targets, as shown in Figure~\ref{fig:nucleon_naive_optimized} (d), Figure~\ref{fig:tooth_naive_optimized} (h), and Figure~\ref{fig:CT-Knee_naive_optimized} (h).

%The evolutions of the objective function on the 4 data sets are also shown in Figure~\ref{fig:nucleon_naive_tooth_naive_rms} (a), (b), (c) and (d) respectively.
%Figure~\ref{fig:CT-Knee_naive_vortex_naive_rms_linesearch} displays the evolution of the objective function in the optimization of the CT-Knee data set and the vortex data set.

%0.05 is used as the step size and the objective function $ F $ falling below 0.03 is regarded as convergence.
Moreover, performance tests of the optimization approaches were conducted on the 4 data sets. In the tests, 0.05 is used as the step size and the objective function $ F $ falling below 0.03 is regarded as convergence. For the 4 data sets we generated an ``ideally optimized'' image using a large number of iterations and compared this, using the SSIM metric ~\cite{Wang2004SSIM}, to images at different progressive stages of optimization. We noted, across the 4 data sets, that for values of the energy function below 0.03, the SSIM scores settled consistently at over 0.99, which was taken as an indicator that further iterations lead to an almost imperceptible change to the rendered image. In practice this threshold can be chosen as demanded by the application or determined using a perceptual study.

Table~\ref{table:performance_table} displays the number of iterations and the time taken for the objective function to converge.
We noticed the two line search methods require much fewer steps to converge than the fixed step-size method and they made the same choices of adaptive step sizes during the iterations (the two curves completely overlapped).
In addition, although the two line search methods use the same numbers of steps, the parallel line search is significantly faster than the sequential line search.

Figure~\ref{fig:parallelsearch_performance} shows the convergence time of the parallel line search approach on the 4 data sets over 1, 2, 4 and 8 CPU threads respectively.
We observed that the computation time decreased as the number of threads increased. However, the speedup from 4 threads to 8 threads is minor, which may due to the fact that the CPU of the experiment computer only has 4 cores.

For the sake of clarity of presentation and convenience of the user study, the results in this section are includes cases of transfer functions with three features and three distinctive feature colors. See Section~\ref{Generality_of_TF} for details of applying the transfer function optimization approach to other transfer functions and other color schemes.

%\begin{table}[h]
%	\begin{tabular}{ c | l | l c r }
%		& & Fixed step size & Line search & Parallel line search \\
%		\hline
%		nucleon & Steps to converge & 37 & 4 & 4 \\
%		& Time (seconds) & 1.91 & 0.83 & 0.55 \\
%		\hline
%		tooth & Steps to converge & 47 & 4 & 4 \\
%		& Time (seconds) & 18.05 & 6.59 & 3.31 \\
%		\hline
%		CT-Knee & Steps to converge & 47 & 6 & 6 \\
%		& Time (seconds) & 120.38 & 59.01 & 33.41 \\
%		\hline
%		vortex & Steps to converge & 56 & 21 & 21 \\
%		& Time (seconds) & 24.95 & 25.42 & 14.92 \\
%	\end{tabular}
%	\caption[Performance of the 3 optimization approaches on the 4 volume data sets]{Performance of the 3 optimization approaches on the 4 volume data sets. ($ F<0.01 $ is regarded as convergence.)}
%	\label{table:performance_table}
%\end{table}

\begin{table}[h]
	\begin{tabular}{ c | l | l c r }
		& & Fixed step size & Line search & Parallel line search \\
		\hline
		nucleon & Steps to converge & 17 & 2 & 2 \\
		& Time (seconds) & 1.07 & 0.59 & 0.38 \\
		\hline
		tooth & Steps to converge & 21 & 2 & 2 \\
		& Time (seconds) & 7.56 & 3.25 & 1.57 \\
		\hline
		CT-Knee & Steps to converge & 17 & 2 & 2 \\
		& Time (seconds) & 33.84 & 17.81 & 9.26 \\
		\hline
		vortex & Steps to converge & 33 & 13 & 13 \\
		& Time (seconds) & 14.00 & 15.22 & 8.40 \\
	\end{tabular}
	\caption[Performance of the 3 optimization approaches]{Performance of the 3 optimization approaches showing steps and time (seconds) taken to converge ($ F<0.03 $ is regarded as convergence.)}
	\label{table:performance_table}
\end{table}

\begin{figure}
	\centering
	\begin{minipage}{.24\textwidth}
		\includegraphics[width=1\linewidth]{figures/nucleon_naive_proportional_optimized_linesearch}
		\subcaption{}
	\end{minipage}~
	\begin{minipage}{.15\textwidth}
		\includegraphics[width=1\linewidth]{figures/tf_nucleon_naive_proportional_optimized_linesearch}
		\subcaption{}
	\end{minipage}~
	\begin{minipage}{.3\textwidth}
		\includegraphics[width=1\linewidth]{figures/nucleon_naive_proportional_optimized_linesearch_visibility_chart}
		\subcaption{}
	\end{minipage}~
	\begin{minipage}{.3\textwidth}
		\includegraphics[width=1\linewidth]{figures/nucleon_naive_proportional_optimized_linesearch_visibility_saliency_weighted_chart}
		\subcaption{}
	\end{minipage}
	
	\begin{minipage}{.49\textwidth}
		\includegraphics[width=1\linewidth]{figures/nucleon_naive_proportional_saliency_fixed}
		\subcaption{}
	\end{minipage}~
	\begin{minipage}{.49\textwidth}
		\includegraphics[width=1\linewidth]{figures/nucleon_naive_proportional_saliency_parallelsearch}
		\subcaption{}
	\end{minipage}
	
	\begin{minipage}{.49\textwidth}
		\includegraphics[width=1\linewidth]{figures/nucleon_naive_proportional_opacity_fixed}
		\subcaption{}
	\end{minipage}~
	\begin{minipage}{.49\textwidth}
		\includegraphics[width=1\linewidth]{figures/nucleon_naive_proportional_opacity_parallelsearch}
		\subcaption{}
	\end{minipage}
	\caption[Optimization results of nucleon]{Optimization results of nucleon. After optimization to target \{0.1, 0.3, 0.6\}, all the 3 features are visible and the green feature inside is particularly emphasized. (a) The optimized volume rendered image of the nucleon data set; (b) The optimized transfer function; (c) The feature visibility histogram \cite{wang_efficient_2011}; (d) The visibility-weighted saliency histogram; (e) \& (f) VWS of features for gradient descent with fixed step sizes and parallel line search respectively; (g) \& (h) Opacities of features (opacities of peak control points) for gradient descent with fixed step sizes and parallel line search respectively.}
	\label{fig:nucleon_naive_optimized}
\end{figure}

\begin{figure}
	\centering
	\begin{minipage}{.29\textwidth}
		\includegraphics[width=1\linewidth]{images/tooth_naive}
		\subcaption{}
	\end{minipage}~
	\begin{minipage}{.2\textwidth}
		\includegraphics[width=1\linewidth]{figures/tf_tooth_naive}
		\subcaption{}
	\end{minipage}~
	\begin{minipage}{.29\textwidth}
		\includegraphics[width=1\linewidth]{images/tooth_naive_optimized_linesearch}
		\subcaption{}
	\end{minipage}~
	\begin{minipage}{.2\textwidth}
		\includegraphics[width=1\linewidth]{figures/tf_tooth_naive_optimized_linesearch}
		\subcaption{}
	\end{minipage}
	
	\begin{minipage}{.25\textwidth}
		\includegraphics[width=1\linewidth]{figures/tooth_naive_visibility_chart}
		\subcaption{}
	\end{minipage}~
	\begin{minipage}{.25\textwidth}
		\includegraphics[width=1\linewidth]{figures/tooth_naive_visibility_saliency_weighted_chart}
		\subcaption{}
	\end{minipage}~
	\begin{minipage}{.25\textwidth}
		\includegraphics[width=1\linewidth]{figures/tooth_naive_optimized_linesearch_visibility_chart}
		\subcaption{}
	\end{minipage}~
	\begin{minipage}{.25\textwidth}
		\includegraphics[width=1\linewidth]{figures/tooth_naive_optimized_linesearch_visibility_saliency_weighted_chart}
		\subcaption{}
	\end{minipage}
	
	\begin{minipage}{.49\textwidth}
		\includegraphics[width=1\linewidth]{figures/tooth_naive_saliency_fixed}
		\subcaption{}
	\end{minipage}~
	\begin{minipage}{.49\textwidth}
		\includegraphics[width=1\linewidth]{figures/tooth_naive_saliency_parallelsearch}
		\subcaption{}
	\end{minipage}
	
	\begin{minipage}{.49\textwidth}
		\includegraphics[width=1\linewidth]{figures/tooth_naive_opacity_fixed}
		\subcaption{}
	\end{minipage}~
	\begin{minipage}{.49\textwidth}
		\includegraphics[width=1\linewidth]{figures/tooth_naive_opacity_parallelsearch}
		\subcaption{}
	\end{minipage}
	\caption[Optimization results of tooth]
	{Optimization results of tooth. After optimization to target \{0.1, 0.3, 0.6\}, all the 3 features are visible and the yellow feature inside is particularly emphasized.
		(a) \& (b) Initial volume rendered image and transfer function; (c) \& (d) Optimized volume rendered image and transfer function; (e) \& (f) Feature visibility and VWS of features before optimization; (g) \& (h) Feature visibility and VWS of features after optimization; (i) \& (j) VWS of features for gradient descent with fixed step sizes and parallel line search respectively; (k) \& (l) Opacities of features for gradient descent with fixed step sizes and parallel line search respectively.}
	\label{fig:tooth_naive_optimized}
\end{figure}

\begin{figure}
	\centering
	\begin{minipage}{.29\textwidth}
		\includegraphics[width=1\linewidth]{images/CT-Knee_naive}
		\subcaption{}
	\end{minipage}~
	\begin{minipage}{.2\textwidth}
		\includegraphics[width=1\linewidth]{figures/tf_CT-Knee_naive}
		\subcaption{}
	\end{minipage}~
	\begin{minipage}{.29\textwidth}
		\includegraphics[width=1\linewidth]{images/CT-Knee_naive_optimized_linesearch}
		\subcaption{}
	\end{minipage}~
	\begin{minipage}{.2\textwidth}
		\includegraphics[width=1\linewidth]{figures/tf_CT-Knee_naive_optimized_linesearch}
		\subcaption{}
	\end{minipage}
	
	\begin{minipage}{.25\textwidth}
		\includegraphics[width=1\linewidth]{figures/CT-Knee_naive_visibility_chart}
		\subcaption{}
	\end{minipage}~
	\begin{minipage}{.25\textwidth}
		\includegraphics[width=1\linewidth]{figures/CT-Knee_naive_visibility_saliency_weighted_chart}
		\subcaption{}
	\end{minipage}~
	\begin{minipage}{.25\textwidth}
		\includegraphics[width=1\linewidth]{figures/CT-Knee_naive_optimized_linesearch_visibility_chart}
		\subcaption{}
	\end{minipage}~
	\begin{minipage}{.25\textwidth}
		\includegraphics[width=1\linewidth]{figures/CT-Knee_naive_optimized_linesearch_visibility_saliency_weighted_chart}
		\subcaption{}
	\end{minipage}	
	
	\begin{minipage}{.49\textwidth}
		\includegraphics[width=1\linewidth]{figures/CT-Knee_naive_saliency_fixed}
		\subcaption{}
	\end{minipage}~
	\begin{minipage}{.49\textwidth}
		\includegraphics[width=1\linewidth]{figures/CT-Knee_naive_saliency_parallelsearch}
		\subcaption{}
	\end{minipage}
	
	\begin{minipage}{.49\textwidth}
		\includegraphics[width=1\linewidth]{figures/CT-Knee_naive_opacity_fixed}
		\subcaption{}
	\end{minipage}~
	\begin{minipage}{.49\textwidth}
		\includegraphics[width=1\linewidth]{figures/CT-Knee_naive_opacity_parallelsearch}
		\subcaption{}
	\end{minipage}
	\caption[Optimization results of CT-Knee]{Optimization results of CT-Knee. After optimization to target \{0.1, 0.3, 0.6\}, all the 3 features are visible, and the green and the purple features inside become clearer.
		(a) \& (b) Initial volume rendered image and transfer function; (c) \& (d) Optimized volume rendered image and transfer function; (e) \& (f) Feature visibility and VWS of features before optimization; (g) \& (h) Feature visibility and VWS of features after optimization; (i) \& (j) VWS of features for gradient descent with fixed step sizes and parallel line search respectively; (k) \& (l) Opacities of features for gradient descent with fixed step sizes and parallel line search respectively.}
	\label{fig:CT-Knee_naive_optimized}
\end{figure}

\begin{figure}
	\centering
	\begin{minipage}{.29\textwidth}
		\includegraphics[width=1\linewidth]{figures/vortex_naive_proportional}
		\subcaption{}
	\end{minipage}~
	\begin{minipage}{.2\textwidth}
		\includegraphics[width=1\linewidth]{figures/tf_vortex_naive_proportional}
		\subcaption{}
	\end{minipage}~
	\begin{minipage}{.29\textwidth}
		\includegraphics[width=1\linewidth]{figures/vortex_naive_proportional_optimized_linesearch}
		\subcaption{}
	\end{minipage}~
	\begin{minipage}{.2\textwidth}
		\includegraphics[width=1\linewidth]{figures/tf_vortex_naive_proportional_optimized_linesearch}
		\subcaption{}
	\end{minipage}
	
	\begin{minipage}{.25\textwidth}
		\includegraphics[width=1\linewidth]{figures/vortex_naive_visibility_chart}
		\subcaption{}
	\end{minipage}~
	\begin{minipage}{.25\textwidth}
		\includegraphics[width=1\linewidth]{figures/vortex_naive_visibility_saliency_weighted_chart}
		\subcaption{}
	\end{minipage}~
	\begin{minipage}{.25\textwidth}
		\includegraphics[width=1\linewidth]{figures/vortex_naive_optimized_fixed_visibility_chart}
		\subcaption{}
	\end{minipage}~
	\begin{minipage}{.25\textwidth}
		\includegraphics[width=1\linewidth]{figures/vortex_naive_optimized_fixed_visibility_saliency_weighted_chart}
		\subcaption{}
	\end{minipage}	
	
	\begin{minipage}{.49\textwidth}
		\includegraphics[width=1\linewidth]{figures/vortex_naive_proportional_saliency_fixed}
		\subcaption{}
	\end{minipage}
	\begin{minipage}{.49\textwidth}
		\includegraphics[width=1\linewidth]{figures/vortex_naive_proportional_saliency_parallelsearch}
		\subcaption{}
	\end{minipage}
	
	\begin{minipage}{.49\textwidth}
		\includegraphics[width=1\linewidth]{figures/vortex_naive_proportional_opacity_fixed}
		\subcaption{}
	\end{minipage}~
	\begin{minipage}{.49\textwidth}
		\includegraphics[width=1\linewidth]{figures/vortex_naive_proportional_opacity_parallelsearch}
		\subcaption{}
	\end{minipage}
	\caption[Optimization results of vortex]{Optimization results of vortex. After optimization to target \{1/3, 1/3, 1/3\}, all the 3 features are visible and the green feature inside is particularly more emphasized in comparison to the unoptimized result.
		(a) \& (b) Initial volume rendered image and transfer function; (c) \& (d) Optimized volume rendered image and transfer function; (e) \& (f) Feature visibility and VWS of features before optimization; (g) \& (h) Feature visibility and VWS of features after optimization; (i) \& (j) VWS of features for gradient descent with fixed step sizes and parallel line search respectively; (k) \& (l) Opacities of features for gradient descent with fixed step sizes and parallel line search respectively.}
	\label{fig:vortex_naive_optimized}
\end{figure}

\begin{figure}
	\centering
	\begin{minipage}{.5\textwidth}
		\includegraphics[width=1\linewidth]{nucleon_performance}
		\subcaption{nucleon}
	\end{minipage}~
	\begin{minipage}{.5\textwidth}
		\includegraphics[width=1\linewidth]{tooth_performance}
		\subcaption{tooth}
	\end{minipage}
	\begin{minipage}{.5\textwidth}
		\includegraphics[width=1\linewidth]{CT-Knee_performance}
		\subcaption{CT-Knee}
	\end{minipage}~
	\begin{minipage}{.5\textwidth}
		\includegraphics[width=1\linewidth]{vortex_performance}
		\subcaption{vortex}
	\end{minipage}
	\caption[Performance of parallel line search]{Performance of parallel line search (seconds taken to converge) over different number of CPU threads}
	\label{fig:parallelsearch_performance}
\end{figure}

\subsection{Transfer Function Optimization For Time-Varying Data Sets}
Similar to the approach discussed in Section~\ref{adaptive_transfer_functions_for_time-varying_data_sets}, we apply our transfer function optimization on all the time steps of the vortex data set. Our optimizer dynamically optimizes the transfer function to the same user-specified target (equal weights i.e. (1/3, 1/3, 1/3) were set as target in this test) for each time step of the time-varying data set. 

Figure~\ref{fig:vorts_static} displays the temporal curves of the visibility-weighted saliency (VWS) and 2D feature saliency (2DFS, discussed in Section~\ref{2d_feature_saliency}) of the visualization with a static transfer function (optimized for the first time step) respectively.
Similarly, Figure~\ref{fig:vorts_dynamic} displays the temporal curves of the VWS and 2DFS of the visualization with a dynamic transfer function (optimized for each time step) respectively.
The VWS curves in Figure~\ref{fig:vorts_dynamic} (a) are more converged than the VWS curves in Figure~\ref{fig:vorts_static} (a) because the dynamic transfer function is constantly optimized towards the target (1/3, 1/3, 1/3).

The 2DFS curves in Figure~\ref{fig:vorts_static} (b) and Figure~\ref{fig:vorts_dynamic} (b) are provided as comparison for the fact that 2DFS is an image space technique independent from VWS. We notice that there are more small changes in the 2DFS curves in Figure~\ref{fig:vorts_static} (b) and the three curves (representing the 2DFS of the three features) cross each other more often than the 2DFS curves in Figure~\ref{fig:vorts_static} (b).

Time step 30 and 80 rendered with the static transfer function and the dynamic transfer function are displayed in Figure~\ref{fig:vorts_50_80_static} and Figure~\ref{fig:vorts_50_80_dynamic} respectively. As coherence can be an important factor in time-varying visualization, we notice the dynamic transfer function can maintain similar level of visual saliency for the purple feature when the sizes and proportions of the features change over time.

%Objective function
%Figure~\ref{fig:nucleon_strong_red_rms}

%Visibility-weighted saliency of each feature
%Figure~\ref{fig:nucleon_strong_red_saliency}
%
%Opacity of each feature
%Figure~\ref{fig:nucleon_strong_red_opacity}

%Before optimization
%Figure~\ref{fig:nucleon_strong_red}
%
%After optimization
%Figure~\ref{fig:nucleon_strong_red_optimized}

%\begin{figure}
%	\centering
%	\begin{minipage}{.35\textwidth}
%		\includegraphics[width=1\linewidth]{images/nucleon_strong_red}
%	\end{minipage}~
%	\begin{minipage}{.2\textwidth}
%		\includegraphics[width=1\linewidth]{images/tf_nucleon_strong_red}	
%	\end{minipage}~
%	\begin{minipage}{.4\textwidth}
%		\includegraphics[width=1\linewidth]{images/nucleon_strong_red_visibility_saliency_weighted_chart}
%	\end{minipage}
%	\caption{Before optimization, the transfer function strongly emphasizes the red feature, which occludes the green feature inside. (Left: the volume rendered image of the nucleon data set; Middle: the transfer function; Right: the visibility-weighted saliency histogram)}
%	\label{fig:nucleon_strong_red}
%\end{figure}

\begin{figure}
	\centering
	\begin{minipage}{.5\textwidth}
		\includegraphics[width=1\linewidth]{figures/vorts_static_VWS}
		\subcaption{}
	\end{minipage}~
	\begin{minipage}{.5\textwidth}
		\includegraphics[width=1\linewidth]{figures/vorts_static_2DFS}
		\subcaption{}
	\end{minipage}
	\caption{(a) VWS and (b) 2DFS of the vortex data set with a static transfer function only optimized for the first time step}
	\label{fig:vorts_static}
\end{figure}

\begin{figure}
	\centering
	\begin{minipage}{.5\textwidth}
		\includegraphics[width=1\linewidth]{figures/vorts_optimized_parallelsearch_VWS}
		\subcaption{}
	\end{minipage}~
	\begin{minipage}{.5\textwidth}
		\includegraphics[width=1\linewidth]{figures/vorts_optimized_parallelsearch_2DFS}
		\subcaption{}
	\end{minipage}
	\caption{(a) VWS and (b) 2DFS of the vortex data set with a dynamic transfer function optimized for each time step}
	\label{fig:vorts_dynamic}
\end{figure}

\begin{figure}
	\centering
	\begin{minipage}{.5\textwidth}
		\includegraphics[width=1\linewidth]{images/vorts30_static}
		\subcaption{}
	\end{minipage}~
	\begin{minipage}{.5\textwidth}
		\includegraphics[width=1\linewidth]{images/vorts80_static}
		\subcaption{}
	\end{minipage}
	\caption{Time step 30 (a) and time step 80 (b) rendered with a static transfer function only optimized for the first time step}
	\label{fig:vorts_50_80_static}
\end{figure}

\begin{figure}
	\centering
	\begin{minipage}{.5\textwidth}
		\includegraphics[width=1\linewidth]{images/vorts30_optimized_parallelsearch}
		\subcaption{}
	\end{minipage}~
	\begin{minipage}{.5\textwidth}
		\includegraphics[width=1\linewidth]{images/vorts80_optimized_parallelsearch}
		\subcaption{}
	\end{minipage}
	\caption{Time step 30 (a) and time step 80 (b) rendered with a dynamic transfer function optimized for each time step}
	\label{fig:vorts_50_80_dynamic}
\end{figure}

%\begin{table}[h]
%	\begin{tabular}{ l | l c r }
%		& Fixed step size & Line search & Parallel line search \\
%		\hline
%		Time (seconds) & 18.05 & 6.59 & 3.31 \\
%		\hline
%		Iterations to converge (at 0.01) & 47 & 4 & 4 \\
%	\end{tabular}
%	\caption[Table caption text]{Performance of the 3 optimization approaches on the tooth data set ($ 140 \times 120 \times 161 $)}
%	\label{table:tooth_table}
%\end{table}
%
%\begin{table}[h]
%	\begin{tabular}{ l | l c r }
%		& Fixed step size & Line search & Parallel line search \\
%		\hline
%		Time (seconds) & 120.38 & 59.01 & 33.41 \\
%		Iterations to converge (at 0.01) & 47 & 6 & 6 \\
%	\end{tabular}
%	\caption[Table caption text]{Performance of the 3 optimization approaches on the CT-Knee data set ($ 379 \times 229 \times 305 $)}
%	\label{table:CT-Knee_table}
%\end{table}
%
%\begin{table}[h]
%	\begin{tabular}{ l | l c r }
%		& Fixed step size & Line search & Parallel line search \\
%		\hline
%		Time (seconds) & 24.95 & 25.42 & 14.92 \\
%		Iterations to converge (at 0.01) & 56 & 21 & 21 \\
%	\end{tabular}
%	\caption[Table caption text]{Performance of the 3 optimization approaches on the vortex data set ($ 128 \times 128 \times 128 $)}
%	\label{table:vortex_table}
%\end{table}

\subsection{Generality of Transfer Functions \label{Generality_of_TF}}

We have provided examples of transfer functions with three features and three distinctive feature colors. However, this was 
purely for the sake of clarity of presentation as well as to be able to specify, for the user study, tasks that could be easily explained to the user. In this section, we provide some examples to demonstrate that the approach can, in fact, be applied equally well to other transfer functions and other color schemes. Furthermore, we wish to examine if the iterative optimization is reasonably robust to changes in initial conditions. 
%Apart from the above results with 3 features and distinctive feature colors, our transfer function optimization approach also works other transfer functions with other color schemes.
We present results of the CT-Knee dataset rendered using transfer functions with a color mapping that is more likely to be used in a real-world application. Specifically, we chose the \emph{CT-Bone} transfer function provided with the medical imaging tool, 3D Slicer~\footnote{www.slicer.org}. For the opacity channel, two types of initial transfer functions are used in this section, namely a function with peaks of equal opacities and another with peaks of linearly increasing opacities. The transfer functions are optimized towards two different VWS targets, i.e. equal VWS and linearly increasing VWS.


Figure~\ref{fig:CT-Knee_CT-Bone5_even} displays a visualization of the CT-Knee dataset before and after optimization to the two VWS targets. In this example, the transfer function is initially set to have equal peak opacity values for all five features. The top row (Figure~\ref{fig:CT-Knee_CT-Bone5_even} (a) - (c)) shows the volume rendered image, the initial transfer function, and the VWS graph respectively. The second row (Figure~\ref{fig:CT-Knee_CT-Bone5_even} (d) - (f))  shows the rendered image, transfer function and the VWS graph after optimizing towards a target with equal conspicuity for all features. The bottom row (Figure~\ref{fig:CT-Knee_CT-Bone5_even} (g) - (h))  shows the results after optimizing towards a VWS target with linearly increasing values for each subsequent feature.

%Figure~\ref{fig:CT-Knee_CT-Bone5_even} displays an initial transfer function with equal opacities and the optimization results of the nucleon data set for 2 different targets. Figure~\ref{fig:CT-Knee_CT-Bone5_even} (a), (b) and (c) are the volume rendered image, the initial transfer function and the VWS graph respectively. Figure~\ref{fig:CT-Knee_CT-Bone5_even} (d), (e) and (f) are the volume rendered image, the transfer function and the VWS graph respectively after optimizing towards an equal VWS target. Figure~\ref{fig:CT-Knee_CT-Bone5_even} (g), (h) and (i) are the results after optimizing towards a linearly increasing VWS target.

Similarly, Figure~\ref{fig:CT-Knee_CT-Bone5_diagonal} displays a similar set of examples for the same dataset, however the initial transfer function in this case is set to linearly increasing peak opacities, in order to test how the initial conditions impact the resulting optimization.

%Similarly, Figure~\ref{fig:CT-Knee_CT-Bone5_diagonal} displays an initial transfer function with a linearly increasing opacities and the optimization results of the nucleon data set for 2 different targets.
%Figure~\ref{fig:CT-Knee_CT-Bone5_diagonal} (a), (b) and (c) are the volume rendered image, the initial transfer function and the VWS graph respectively. Figure~\ref{fig:CT-Knee_CT-Bone5_diagonal} (d), (e) and (f) are the volume rendered image, the transfer function and the VWS graph respectively after optimizing towards an equal VWS target.
%Figure~\ref{fig:CT-Knee_CT-Bone5_diagonal} (g), (h) and (i) are the results after optimizing towards a linearly increasing VWS target.

Although, we observe, by comparing Figure~\ref{fig:CT-Knee_CT-Bone5_even}(e), (h) and  Figure~\ref{fig:CT-Knee_CT-Bone5_diagonal} (e), (h) respectively, that the optimized transfer function for the corresponding targets are slightly different, the final volume rendered images turn out to be very similar in appearance. We also ran tests with the four datasets and opacity transfer functions with differing number of features, ranging from 3 to 9 tent-shaped peaks. We observed similar behavior to the CT-Knee examples above, indicating that the process is not significantly sensitive to initial conditions in terms of both output quality and performance.
%
%We note that although the optimization results presented in Figure~\ref{fig:CT-Knee_CT-Bone5_even} and Figure~\ref{fig:CT-Knee_CT-Bone5_diagonal} are started from different initial transfer functions, and the optimized transfer functions are slightly different, the results of the final volume rendered images turn out to be very similar in appearance. Similar results were observed when tested with other datasets, and differing number of features, indicating that the approach that the optimization process is not significantly sensitive to initial conditions.
%Figure~\ref{fig:nucleon_rms} shows the graphs of objective functions of the 4 optimization processes in Figure~\ref{fig:nucleon_even} and Figure~\ref{fig:nucleon_diagonal}.

\begin{figure}
	\centering
	\begin{minipage}{.9\textwidth}%%%%%%%%%%%%%%%%%%%% ADDED THIS TEMPORARILY AS THIS FIGURE IS OVERFLOWING THE PAGE %%%%%%%%%%
		\begin{minipage}{.3\textwidth}
			\includegraphics[width=1\linewidth]{CT-Knee_CT-Bone5_even}
			\subcaption{}
		\end{minipage}~
		\begin{minipage}{.3\textwidth}
			\includegraphics[width=1\linewidth]{tf_CT-Knee_CT-Bone5_even}
			\subcaption{}
		\end{minipage}~
		\begin{minipage}{.4\textwidth}
			\includegraphics[width=1\linewidth]{CT-Knee_CT-Bone5_even_visibility_saliency_weighted_chart}
			\subcaption{}
		\end{minipage}
		
		\begin{minipage}{.3\textwidth}
			\includegraphics[width=1\linewidth]{CT-Knee_CT-Bone5_even_optimized_parallelsearch_target_even}
			\subcaption{}
		\end{minipage}~
		\begin{minipage}{.3\textwidth}
			\includegraphics[width=1\linewidth]{tf_CT-Knee_CT-Bone5_even_optimized_parallelsearch_target_even}
			\subcaption{}
		\end{minipage}~
		\begin{minipage}{.4\textwidth}
			\includegraphics[width=1\linewidth]{CT-Knee_CT-Bone5_even_optimized_parallelsearch_target_even_visibility_saliency_weighted_chart}
			\subcaption{}
		\end{minipage}
		
		\begin{minipage}{.3\textwidth}
			\includegraphics[width=1\linewidth]{CT-Knee_CT-Bone5_even_optimized_parallelsearch_target_diagonal}
			\subcaption{}
		\end{minipage}~
		\begin{minipage}{.3\textwidth}
			\includegraphics[width=1\linewidth]{tf_CT-Knee_CT-Bone5_even_optimized_parallelsearch_target_diagonal}
			\subcaption{}
		\end{minipage}~
		\begin{minipage}{.4\textwidth}
			\includegraphics[width=1\linewidth]{CT-Knee_CT-Bone5_even_optimized_parallelsearch_target_diagonal_visibility_saliency_weighted_chart}
			\subcaption{}
		\end{minipage}
	\end{minipage}
	\caption[CT-Knee: volume rendered images, transfer functions and VWS graphs]{CT-knee: (a), (b) and (c) are the volume rendered images, transfer functions and VWS graphs respectively; (d), (e) and (f) are the images after VWS-optimization to a even VWS target, where all features have similar VWS; (g), (h) and (i) are the images after VWS-optimization to a diagonal VWS target, where the internal features are clearer.
		%		(e) Gradient descent with fixed step sizes; (f) Gradient descent with parallel line search
	}
	\label{fig:CT-Knee_CT-Bone5_even}
\end{figure}

\begin{figure}
	\centering
	\begin{minipage}{.9\textwidth}%%%%%%%%%%%%%%%%%%%% ADDED THIS TEMPORARILY AS THIS FIGURE IS OVERFLOWING THE PAGE %%%%%%%%%%
		\begin{minipage}{.3\textwidth}
			\includegraphics[width=1\linewidth]{CT-Knee_CT-Bone5_diagonal}
			\subcaption{}
		\end{minipage}~
		\begin{minipage}{.3\textwidth}
			\includegraphics[width=1\linewidth]{tf_CT-Knee_CT-Bone5_diagonal}
			\subcaption{}
		\end{minipage}~
		\begin{minipage}{.4\textwidth}
			\includegraphics[width=1\linewidth]{CT-Knee_CT-Bone5_diagonal_visibility_saliency_weighted_chart}
			\subcaption{}
		\end{minipage}
		
		\begin{minipage}{.3\textwidth}
			\includegraphics[width=1\linewidth]{CT-Knee_CT-Bone5_diagonal_optimized_parallelsearch_target_even}
			\subcaption{}
		\end{minipage}~
		\begin{minipage}{.3\textwidth}
			\includegraphics[width=1\linewidth]{tf_CT-Knee_CT-Bone5_diagonal_optimized_parallelsearch_target_even}
			\subcaption{}
		\end{minipage}~
		\begin{minipage}{.4\textwidth}
			\includegraphics[width=1\linewidth]{CT-Knee_CT-Bone5_diagonal_optimized_parallelsearch_target_even_visibility_saliency_weighted_chart}
			\subcaption{}
		\end{minipage}
		
		\begin{minipage}{.3\textwidth}
			\includegraphics[width=1\linewidth]{CT-Knee_CT-Bone5_diagonal_optimized_parallelsearch_target_diagonal}
			\subcaption{}
		\end{minipage}~
		\begin{minipage}{.3\textwidth}
			\includegraphics[width=1\linewidth]{tf_CT-Knee_CT-Bone5_diagonal_optimized_parallelsearch_target_diagonal}
			\subcaption{}
		\end{minipage}~
		\begin{minipage}{.4\textwidth}
			\includegraphics[width=1\linewidth]{CT-Knee_CT-Bone5_diagonal_optimized_parallelsearch_target_diagonal_visibility_saliency_weighted_chart}
			\subcaption{}
		\end{minipage}
	\end{minipage}
	\caption[CT-Knee: volume rendered images, transfer functions and VWS graphs]{CT-knee: (a), (b) and (c) are the volume rendered images, transfer functions and VWS graphs respectively; (d), (e) and (f) are the images after VWS-optimization to a even VWS target, where all features have similar VWS; (g), (h) and (i) are the images after VWS-optimization to a diagonal VWS target, where the internal features are clearer.
		%		(e) Gradient descent with fixed step sizes; (f) Gradient descent with parallel line search
	}
	\label{fig:CT-Knee_CT-Bone5_diagonal}
\end{figure}


%-------------------------------------------------------------------------
\section{Conclusions}
This chapter proposes a novel transfer function optimization approach using the visibility-weighted saliency metric.
With this approach, the design of transfer functions becomes more intuitive. This approach allows the user to directly set target visibility-weighted saliency for features of interest and then the transfer function is automatically refined to match the visibility-weighted saliency of the features with user-defined targets. In addition, a parallel line search strategy is presented for exploiting the computing power of multi-core processors to improve the performance of the transfer function optimization approach.
This approach has proven to be effective over several volume data sets.
%-------------------------------------------------------------------------

\chapter{Experiment}

\section{Experiment}
To judge the performance of the proposed approach, a human study was conducted to construct a subjective data set for assessing visual saliency of features in volume visualization as perceived by human users. The aim of this set of experiments is to gather human rating of visual saliency data and gather eye tracking data to evaluate and improve a proposed computational visual saliency metric for volume visualization.

\subsection{Source Images and Participants}
This section describes the human study and experiments performed using it.
The images used for the study were rendered by Voreen \cite{meyer-spradow_voreen:_2009} with a variety of transfer functions highlighting different features in various viewpoints.
** females and ** males participated in the experiment all aged between ** and ** years.

\subsection{Methods and Measurements}
Participants would sit in front of a computer display viewing images generated from volume visualization. The first part of the experiment would investigate how participants perceive the visual saliency of different objects in the images. The participants’ task would be to score the images on the scale of 1 to 5 by keyboard input. In the next part of the experiment, the participants would be asked to score the quality (in terms of sharpness and contrast) of the images (also by keyboard input) and their eye movements would be tracked by a head mounted eye tracker (EyeLink II by SR Research). In both parts of the experiment, each image would be shown to the participant for approximately 10 seconds. The experiment would last approximately 30 minutes. There would be a break scheduled midway through the experiment to allow the participant to rest.

%-------------------------------------------------------------------------

%-------------------------------------------------------------------------



%\addcontentsline {toc}{chapter}{Appendices}       %% Force Appendices to appear in contents
%\begin{appendix}
% \chapter{Estimating Feature Saliency Using 2D Saliency Maps \label{2d_saliency_map}}

A saliency map is a model of visual attention using bottom-up features such as intensity, color and orientation of an image.
%A saliency map 111 is a model of visual selective 
%attention using purely bottom-up features of an image like color, 
%intensity and orientation. Another bottom-up feature of visual 
%input is depth, the distance between eye (or sensor) and objects 
%in the visual field.
However, traditional saliency maps were designed to provide an indication of visual saliency for 2D images with no clue of particular objects or 3D features in the scene.
In order to use 2D saliency maps \cite{itti_model_1998} to estimate visual saliency of 3D features in volume visualization, an inverse distance weighting \cite{shepard_two-dimensional_1968} can be applied to divide a 2D saliency map into several feature saliency maps, one for each feature. Subsequently, the visual saliency of each feature can be estimated with the total intensity of each feature saliency map.

The distance between a pixel of each feature and the pixel in the final image is necessary in computing the inverse distance weighting.
Hence, we perform volume rendering of each feature separately, i.e. other intensity ranges in the transfer function are set to zero except for the feature.
These feature images $ P_{i} (i \in \{1,...,n\})$ are rendered with the same settings (viewpoint, screen size etc.) as the final image.
In addition, a 2D saliency map $ S $ of the final image $ P $ is computed using the model by Itti et al. \cite{itti_model_1998}.

Let $ w_{i} $ be the weight of a pixel $ p $ in the $i$-$th$ feature
\[ w_{i} = \frac{ \frac{1}{d_{i}^{m}} }{ \sum_{j=1}^{n} \frac{1}{d_{j}^{m}} } \]
where $ d_{i} $ is the color distance between the pixel $ p $ in the final image and the corresponding pixel $ p_{i} $ in the $i$-$th$ feature image,
$ n $ is the number of features, and
$ m $ is a user-defined coefficient for controlling the bias of the weighting. Pixels with small distances would have larger weights when $ m $ increases. $ m=1 $ is used and the color distance $ d_{i} $ is computed in the LAB color space in our implementation

Then the corresponding pixel $ s_{i} $ in the $i$-$th$  feature saliency map $ S_{i} $ is
\[ s_{i}=w_{i}s \]
where $ s $ is the pixel in the 2D saliency map $ S $ of the final image.

Therefore, we can obtain $ n $ feature saliency maps by performing the above a pixel-wise operation using the 2D saliency map $ S $ and the final image $ P $ along with each feature images $ P_{i} $ respectively.

Figure~\ref{fig:engine_naive} shows an engine block ($ P $) and its two features ($ P_{1} , P_{2} $).
Figure~\ref{fig:engine_naive_saliencemap} shows the 2D saliency map $ S $ and the two feature saliency maps($ S_{1} , S_{2} $) obtained using the above operation.
The saliency maps in Figure~\ref{fig:engine_naive_saliencemap} are enhanced (multiplied by 8) for better contrast in illustrations. However, the original (i.e. not enhanced) saliency maps are used in the actual computation.

In practice, saliency resulting from a visual feature is not sharply delimited by the boundary of the feature, instead strong feature edges tend to attract attention increasing saliency in a small distribution around the edge.
After distributing the 2D saliency map $ S $ into feature saliency maps $ S_{i} (i \in \{1, ... ,n\})$ using the inverse distance weighting, a small amount of bright pixels around the boundary of the engine block remain in the residual saliency image $ S' $, as shown in Figure~\ref{fig:engine_naive_saliencemap_left}~(a).
Let the residual saliency image be $ S' $.
\[ S'=S- \sum_{j=1}^{n} S_{j} \]

We distribute this residual saliency image $ S' $ to the features according to their influence in the region. The influence of the features are approximated by Gaussians of the feature saliency images, as shown in (b) and (c) of Figure~\ref{fig:engine_naive_saliencemap_left}.
%We distribute these values and add them to the feature saliency maps using another weighting based on the Gaussians of the feature saliency maps $ S_{i}$.
Firstly, we apply a Gaussian filter with kernel size $ k $ to each feature saliency map $ S_{i} $ and get a Gaussian image $ G_{i} $.
In practice, the kernel size $ k $ should be large enough in order to allow the resulting Gaussian images to have non-zero pixels cover most of the bright pixels in the residual saliency image $ S' $.
Let $ g_{i} $ be a pixel in the Gaussian image $ G_{i} $.
Secondly, we distribute the residual saliency image $ S' $ into $ n $ images ($ S'_{1} , ... , S'_{n} $).
%\[ G_{i}=Gaussian(S_{i},k) \]
\[ s'_{i} = \frac{ g_{i} }{ \sum_{j=1}^{n} g_{j} }s' \]
where $ s' $ is a pixel in $ S' $.
Figure~\ref{fig:engine_naive_leftgaussian} displays the residual saliency images ($ S'_{1} $, $ S'_{2} $) of the two features on the engine block. 

Thirdly, we pixel-wisely add the image $ S'_{i} $ to the feature saliency map $ S_{i} $ and obtain the total feature saliency map $ T_{i} $ of the $i$-$th$ feature.
\[ T_{i} =S_{i}+S'_{i}\]

Figure~\ref{fig:engine_naive_saliencemap_features} (a) and (b) display the total feature saliency maps of the red feature and the green feature respectively.

Finally, as shown in Figure~\ref{fig:engine_naive_saliencemap_features} (c), we compute the 2D feature saliency using the sum of intensity values of the total feature saliency maps, i.e. $ T_{i}$ for $ i \in \{1, ... ,n\} $.
Hence, the 2D feature saliency of the $i$-$th$  feature is
\[ FS_{i}=\frac{Intensity(T_{i})}{ \sum_{j=1}^{n} Intensity(T_{j}) } \]

A histogram of the 2D feature saliency of the two features of the engine block is shown in Figure~\ref{fig:engine_naive_saliencemap_features} (c).

\begin{figure}
	\centering
	\begin{minipage}{.33\textwidth}
		\includegraphics[width=1\linewidth]{images/engine_naive}
		\subcaption{$ P $}
	\end{minipage}~
	\begin{minipage}{.33\textwidth}
		\includegraphics[width=1\linewidth]{images/engine_naive_1}
		\subcaption{$ P_{1} $}
	\end{minipage}~
	\begin{minipage}{.33\textwidth}
		\includegraphics[width=1\linewidth]{images/engine_naive_2}
		\subcaption{$ P_{2} $}
	\end{minipage}
	\caption{(a) An engine block; (b) and (c) isolated volume rendering images of the red feature and the green feature}
	\label{fig:engine_naive}
\end{figure}

\begin{figure}
	\centering
	\begin{minipage}{.33\textwidth}
		\includegraphics[width=1\linewidth]{images/engine_naive_saliencemap}
		\subcaption{$ S $}
	\end{minipage}~
	\begin{minipage}{.33\textwidth}
		\includegraphics[width=1\linewidth]{images/engine_naive_saliencemap_1_overlap}
		\subcaption{$ S_{1} $}
	\end{minipage}~
	\begin{minipage}{.33\textwidth}
		\includegraphics[width=1\linewidth]{images/engine_naive_saliencemap_2_overlap}
		\subcaption{$ S_{2} $}
	\end{minipage}
	\caption{(a) The 2D saliency map; (b) and (c) the feature saliency maps of the two features. The saliency maps are enhanced (multiplied by 8) for better contrast in illustrations.}
	\label{fig:engine_naive_saliencemap}
\end{figure}

\begin{figure}
	\centering
	\begin{minipage}{.33\textwidth}
		\includegraphics[width=1\linewidth]{images/engine_naive_saliencemap_left}
		\subcaption{$ S' $}
	\end{minipage}~
	\begin{minipage}{.33\textwidth}
		\includegraphics[width=1\linewidth]{images/engine_naive_gaussian_1}
		\subcaption{$ G_{1} $}
	\end{minipage}~
	\begin{minipage}{.33\textwidth}
		\includegraphics[width=1\linewidth]{images/engine_naive_gaussian_2}
		\subcaption{$ G_{2} $}
	\end{minipage}
	\caption{(a) The residual saliency image; (b) and (c) the Gaussians of the two feature saliency maps with a kernel size of one eighth of the image width}
	\label{fig:engine_naive_saliencemap_left}
\end{figure}

\begin{figure}
	\centering
	\begin{minipage}{.33\textwidth}
		\includegraphics[width=1\linewidth]{images/engine_naive_leftgaussian_1}
		\subcaption{$ S'_{1} $}
	\end{minipage}~
	\begin{minipage}{.33\textwidth}
		\includegraphics[width=1\linewidth]{images/engine_naive_leftgaussian_2}
		\subcaption{$ S'_{2} $}
	\end{minipage}
	\caption{The residual saliency images of the two features}
	\label{fig:engine_naive_leftgaussian}
\end{figure}

\begin{figure}
	\centering
	\begin{minipage}{.33\textwidth}
		\includegraphics[width=1\linewidth]{images/engine_naive_saliencemap_1}
		\subcaption{$ T_{1} $}
	\end{minipage}~
	\begin{minipage}{.33\textwidth}
		\includegraphics[width=1\linewidth]{images/engine_naive_saliencemap_2}
		\subcaption{$ T_{2} $}
	\end{minipage}~
	\begin{minipage}{.33\textwidth}
		\includegraphics[width=1\linewidth]{images/engine_naive_2DFS}
		\subcaption{2D feature saliency}
	\end{minipage}
	\caption{(a) and (b) The total feature saliency maps of the two features; (c) 2D feature saliency of the two features}
	\label{fig:engine_naive_saliencemap_features}
\end{figure}

%% \chapter{Experiment Questionnaire \label{Experiment_pdf}}

This is the questionnaire used in the experiment presented in Section~\ref{experiment_section}.

\includepdf[pages={-}]{Questionnaire_real.pdf}

%\end{appendix}


%\addcontentsline {toc}{chapter}{Bibliography}     %% Force Bibliography to appear in contents

%\begin{thebibliography}{ieeetr}                   %% Start your bibliography here; you can
%%\bibliography{refs}                               %% also use the \bibliography command
%\end{thebibliography}                             %% to generate your bibliography.

%%%%%%%%%%%%%%%%%%%%%%%%%%%%%%%%%%%%%%%%%%%%%%%%%%%%%%%%%%%%%
%% BIBLIOGRAPHY AND OTHER LISTS
%%%%%%%%%%%%%%%%%%%%%%%%%%%%%%%%%%%%%%%%%%%%%%%%%%%%%%%%%%%%%
\bibliographystyle{ieeetr}
\bibliography{bibliography}

\end{document}                                    %% END THE DOCUMENT
