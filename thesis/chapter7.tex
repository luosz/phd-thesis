\chapter{Conclusions \label{conclusions}}

This chapter provides an overview of the contributions of this thesis and directions for future work on visualizing volume data with automated techniques.

\section{Summary of Contributions}
Volume visualization is a broad field covering many subtopics. The goal of this thesis was to provide a framework of automated transfer function techniques for obtaining clear visualization of features of interest in volume data. This thesis proposed and investigated novel automated optimization techniques for emphasizing features of interest in volume visualization using entropy, visual saliency and visibility of voxels.

We have presented a transfer function refinement approach, which exploits the entropy of voxels to equalize the opacity transfer function, in order to reduce general occlusion and improve the clarity of features of interest in the resulting visualization.
Furthermore, this approach assists the user in exploring and enhancing features of interest by interactively specifying either priority intensity ranges in the transfer function domain or regions of interests in the resulting visualization.
Our approach is different from Ruiz et al. \cite{ruiz_automatic_2011}, where the transfer function is adjusted towards a user-defined target distribution by minimizing the informational divergence between the transfer function and the user-defined target distribution.

In addition to view-independent information, we have proposed visibility-weighted saliency for measuring the view-dependent saliency of features of interest for volume visualization.
This metric aims to assist users in choosing suitable viewpoints and designing effective transfer functions to visualize the features of interest.
Compared to existing approaches such as visibility histogram \cite{correa_visibility_2011} and feature visibility \cite{wang_efficient_2011} which only measure the visibility of voxels, our approach reflects two aspects of the resulting visualization, i.e. voxel visibility and visual saliency. Our approach is better at indicating the change of saturation and brightness of volume visualization than the existing approaches which do not take into account the visual saliency information. A user study was conducted to evaluate the efficiency of our metric in comparison to feature visibility and 2D feature saliency.

Subsequently, we have described an automated transfer function optimization method based on the visibility-weighted saliency metric. This method takes into account the perceptual importance of voxels and the visibility of features, and automatically adjusts the transfer function to match the target saliency levels specified by the user. In addition, a parallel line search strategy is presented to improve the performance of the optimization algorithm.

Finally, we have developed a novel visualization approach which modulates focus, emphasizing important information, by adjusting saturation and brightness of voxels based on an importance measure derived from temporal and multivariate information.
By conducting a voxel-wise analysis of a number of consecutive frames, we acquire a volatility measure of each voxel. We then use intensity, volatility and additional multivariate information to determine opacity, saturation and brightness of the voxels.

\section{Limitations and Future Work}
\subsection{Automated Transfer Function Approaches}
The main limitation of our automated transfer function approaches (Chapter~\ref{transfer_function_refinement} and Chapter~\ref{transfer_function_optimization}) is that the variations to transfer functions are limited to color and opacity, and an initial setup of control points over the intensity ranges has to be defined by the user. Therefore, prior knowledge of the data sets may be necessary in choosing the most ideal intensity ranges for placing the control points.

In future work we plan to develop transfer function generation methods which identify important features in volume data sets and combine them with our automated transfer function optimization approaches. A promising direction would be the rule-enhanced transfer function generation method \cite{cai_rule-enhanced_2015}, which defines features with rules based on the frequency distribution of data attributes (e.g. intensity and gradient magnitude) and employs machine learning algorithms to select a set of rules that are most effective in distinguishing the target tissue from other tissues.
We would also like to examine the feasibility of optimizing other color components such as saturation, brightness and even hue. 

The initial choice of intensity ranges, number of control points and color mapping across the histograms can affect the quality of the final output and some prior knowledge of the data sets may be of benefit for optimal results. On the other hand the simple and straightforward techniques presented in this paper should be fully compatible with independent mechanisms for choosing optimal combinations of other visual parameters or indeed if the user wishes to combine these with more manual choices of parameters such as the color map.
In addition, the transfer functions in our proposed system are intuitive and easy to use. Users may benefit from the flexibility of further interactive tweaking the intensity or opacity of the control points after the application of the automated optimization techniques discussed in this thesis.

\subsection{Visibility-Wighted Saliency}
The major limitation of the visibility-weighted saliency metric presented in Chapter~\ref{visibility-weighted_saliency} is that it cannot detect changes in hue and orientation in volume visualization. The saliency field used in the metric is computed using the center surround mechanism based on saturation and brightness of voxels, but other factors such as hue of the voxels and orientation of the local neighborhoods around the voxels are not taken into account.

The center-surround operator \cite{kim_saliency-guided_2006} we used in computing saliency fields of volume data is essentially a Laplace differential operator. This operator can handle scalar fields of data attributes such as opacity, saturation and brightness. However, computing the difference between colors and orientations is more complicated, we have not yet integrated it in our model.

In the future, we would like to examine the feasibility of using other data attributes, e.g. hue and orientation. The difference between colors could be computed for each pair of voxels in a color space, e.g. the perceptually uniform LAB color space. The orientations of voxels could be estimated using the Gabor filter, which would be similar to the construction of the orientation conspicuity map \cite{itti_model_1998}.

%For instance, ch4: what are the outstanding questions  and limitations of the work on the metric e.g. that for saliency it assumes centre surrond mechanism base on saturation/brightness but not other factors e.g. hue, orientation etc.
\subsection{Selective Saturation and Brightness}
We have not applied the visibility-weighted saliency metric to the work in Chapter~\ref{selective_saturation_brightness}. One reason is that the work in Chapter~\ref{selective_saturation_brightness} was done before Chapter~\ref{visibility-weighted_saliency} and Chapter~\ref{transfer_function_optimization}. On the other hand, it is difficult to apply the proposed metric to measure the features in Chapter~\ref{selective_saturation_brightness}'s work, because the feature definition and transfer functions are very different from those in the previous chapters. Instead of defining features on intensity ranges, volatility (temporal standard deviation) are used in the feature definition in Chapter~\ref{selective_saturation_brightness}.

In the future, we would like to investigate how to apply the visibility-weighted saliency metric to other types of feature definitions and how automated optimization techniques can be exploited to better visualize time-varying and multivariate volume data sets. Automatic color scheme generation \cite{zhou_automatic_2009} for transfer functions would also be studied in the future work.
In addition to transfer function optimization, we would like to implement an automatic viewpoint optimization technique based on the proposed metric.

%-------------------------------------------------------------------------
