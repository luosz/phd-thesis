\chapter{Conclusions \label{conclusions}}

This chapter provides an overview of the contributions of this thesis and directions for future work on visualizing volume data with automated techniques.

\section{Summary of Contributions}
We have presented a transfer function refinement approach, which exploits the entropy of voxels derived from volume to equalize the opacity transfer function, in order to reduce general occlusion and improve the clarity of features of interest in the resulting visualization.
Moreover, this approach assists the user in exploring and enhancing features of interest by interactively specifying different priority intensity ranges.

In addition to view-independent information (i.e. entropy of voxels), we propose visibility-weighted saliency for measuring the view-dependent saliency of features of interest for volume visualization.
This metric aims to assist users in choosing suitable viewpoints and designing effective transfer functions to visualize the features of interest.

Subsequently, we describe an automated transfer function optimization method based on the visibility-weighted saliency metric. This method takes into account the perceptual importance of voxels and the visibility of features, and automatically adjusts the transfer function to match the target saliency levels specified by the user. In addition, a parallel line search strategy is presented to improve the performance of the optimization algorithm.

Finally, we develop a novel visualization approach which modulates focus, emphasizing important information, by adjusting saturation and brightness of voxels based on an importance measure derived from temporal and multivariate information.
By conducting a voxel-wise analysis of a number of consecutive frames, we acquire a volatility measure of each voxel. We then use intensity, volatility and additional multivariate information to determine opacity, saturation and brightness of the voxels.

%-------------------------------------------------------------------------
