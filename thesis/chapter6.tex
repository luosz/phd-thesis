\chapter{Transfer Function Optimization Using Visibility-Weighted Saliency}

\section{Introduction}
Volume visualization has proven to be an effective means of discovering meaningful features in volume data sets. By specifying appropriate opacity values for various features, the exterior and interior features can be simultaneously revealed in a semi-transparent manner.

%-------------------------------------------------------------------------
\section{Method}
In Chapter \ref{visibility-weighted_saliency}, visibility-weighted saliency was proposed as a measure of visual saliency of features in volume rendered images, in order to assist users in choosing suitable viewpoints and designing effective transfer functions to visualize the features of interest. In this chapter, we describe a transfer function optimization approach based on the visibility-weighted saliency metric, which indicates the perceptual importance of voxels and the visibility of features in volume rendered images.

The approach described in Chapter \ref{transfer_function_refinement} is an automated method of optimizing transfer functions, based on the intensity distribution of voxels in the volume data set. However, this approach does not take into account the spatial distribution of voxels and the viewpoint of the visualization. Visibility-weighted saliency, on the other hand, takes into account both of these two aspects. The visibility-weighted saliency consists of two component fields, i.e. saliency field and visiblity fields. Saliency fields are essentially difference of Gaussians, which include the information of local neighborhoods of voxels, and visibility fields are computed from opacity contribution of voxels to volume rendered images, which indicate viewpoint dependent occlusions of the voxels.

%-------------------------------------------------------------------------
\section{Conclusions}
