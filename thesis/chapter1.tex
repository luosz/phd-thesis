\chapter{Introduction \label{section_introduction}}
%\epigraph{``What is above form is called Tao; what is within form is called tool."}{--- \textup{I Ching (the Classic of Changes)}}

%Volume visualization is an active branch of scientific visualization. It is a method of extracting meaningful information from volume data (3D discretely sampled data sets) using interactive graphics and imaging. The study of volume visualization involves volume data representation, modeling, manipulation and rendering \cite{kaufman_volume_1997}.
%First introduced by Levoy \cite{levoy_display_1988}, volume visualization has been widely used in various sciences to create insightful visualizations from both simulated and measured data.
%Volume visualization is a powerful technique which aims to visualize the 3D structures in volume data sets and thus facilitates the user's exploration into the data. It has become an important technique for various applications such as medical imaging and scientific visualization.
%%Furthermore, it is especially useful in diagnostics for physicians in medicine.
%Recent advances in volume data acquisition and scientific simulations have led to dramatically increasing volume data sets, both in size and complexity, that must be visualized and analyzed \cite{beyer_state---art_2015}.

Volume visualization is an active branch of scientific visualization concerned with extracting meaningful information from volume data (3D discretely sampled data sets) using interactive graphics and imaging. The study of volume visualization involves volume data representation, modeling, manipulation and rendering \cite{kaufman_volume_1997} and it aims, in particular, to facilitate visual exploration of 3D structures allowing users to more deeply understand and analyze volume data sets. 
First introduced by Levoy \cite{levoy_display_1988} in 1988, volume visualization has been widely used in various sciences to create insightful visualizations from both simulated and measured data. However, recent advances in volume data acquisition and scientific simulations have led to dramatic increases in both size and complexity of data sets, which present new and ongoing challenges to be addressed \cite{beyer_state---art_2015}.

%Time-varying volume data sets are increasing dramatically both in size and complexity as various data acquisition devices rapidly evolve in recent years.
%However, the visualization of time-varying volume data remains a challenging problem due to the large size and the dynamic nature of the underlying information.
%Previously established techniques, such as in flow visualization, struggle to deal with the increasing complexity of the most recent data sets.

The rendering of volume data requires every sample value (also called voxel, which is a volume element or volumetric pixel) to be mapped to visual properties (e.g. opacity and color). This mapping is done with a transfer function, which can be a simple ramp, a piecewise linear function or an arbitrary table.
The design of an effective transfer function (see Section~\ref{literature_of_transfer_function} for details) is essential for visualizing volume data.
%A wealth of techniques have been developed for transfer function design for static volume data \cite{pfister_transfer_2001} \cite{bernardon_transfer-function_2008} \cite{arens_survey_2010}.

With volume rendering, both the exterior and interior of a volume data set can be revealed semi-transparently by specifying appropriate transfer functions.
However, because of 3D occlusion between structures and the indirect control over the final visualization, it is time-consuming and unintuitive for users to specify appropriate transfer functions.
%In practice, transfer functions are often specified in a trial and error manner with careful observation of changes in the resulting visualization \cite{kniss_interactive_2001}.
In practice, this is typically achieved using a trial-and-error approach: modifications are made to the transfer function and changes in the resulting visualization are carefully observed in order to inform further modifications to the transfer function \cite{kniss_interactive_2001}.
The adjustments users make in transfer function specification are based on subjective perception of important features in a certain viewpoint.

%By speci-
%fying appropriate opacities for extracted features, the ex-
%terior and interior features can be simultaneously revealed
%in a semi-transparent manner. 
% 
%However, in practice it is
%rather difficult and time-demanding to specify appropriate
%opacities, due to 3D occlusion between features. The main
%reason is that interaction in the transfer function domain
%is mainly guided by careful observation of changes in the
%rendered image [KKH01]. Decisions based on this kind
%of adjustment are subjective and view-dependent.
%
%there is no quantitative metric to measure the influ-
%ence of each feature to the rendered image. While most
%of the previous research focused on how to extract fea-
%tures 
%
%very little attention has
%been paid to quantitative analysis of the visibility of classi-
%fied features in the rendered image.
 


%However, transfer function design for time-varying volume data has not been studied thoroughly.
%A fundamental challenge in the analysis and classification of time-varying volume data is the lack of capability to track data change or evolution over time \cite{gu_transgraph_2011}.
%Much of the work in the field of volume visualization has been focused on the synthesis of photorealistic images to assist in the visualization of structures contained in volume data sets.
%However, traditional depictions of the same types of data, such as those found in medical textbooks, deliberately use non-realistic techniques to draw the viewer's attention to important aspects \cite{bruckner_style_2007}. Using abstraction, visual overload is prevented and thus result in a more effective visualization.
%NPR techniques are effective forms of abstraction. They are commonly inspired by artistic styles and techniques that do not focus on a realistic depiction of scenes and objects, therefore, they can express features that cannot be shown using physically correct light transport. Non-photorealistic images are used in preference to photorealistic images in specific circumstances. For instance, an empirical study \cite{schumann_assessing_1996} reported that, when asked to compare a computer-produced sketch against a photorealistically rendered CAD image, architects showed a great preference for the sketch.
%NPR models were adopted in visualization and hence formed the field of illustrative visualization.
%Although illustrative visualization is a relatively novel category of visualization approaches, it has been successfully employed in medical and other visualization sub-fields \cite{svakhine_illustration_2005} \cite{svakhine_illustration-inspired_2009}.
%Illustrative visualization has proven its usefulness in revealing 3D structures due to its ability to hide less relevant details while emphasizing important details. The goal of illustrative visualization is to gain clarity compared to photo-realistic rendering by emphasizing important features and improving data exploration. In order to obtain more comprehensive images, it is necessary to highlight important aspects and omit less relevant details.

\section{Motivation \label{motivation}}
%Understanding and analyzing complex volumetrically varying data is a challenging problem. Many visualization techniques have had only limited success in succinctly portraying the structure of volume data sets.
%There are only limited success have been reported about various visualization techniques which succinctly portray the structure of 3D time-varying volume data.

%The main goal of our research is to investigate the optimization of visualization parameters (in particular transfer functions) and the use of NPR techniques, and develop a methodology which incorporates these two types of techniques to facilitate the user's exploration of the data sets. NPR techniques are effective forms of abstraction and they have proven their usefulness in expressing features that cannot be shown using realistic depiction of scenes and objects. The combination of standard volume visualization and NPR techniques will bring the opportunity to provide expressive visualization and assist the user in accomplishing his/her task efficiently.

Objective measures such as voxel information \cite{bordoloi_view_2005} \cite{wang_information_2011}, visibility histograms \cite{emsenhuber_visibility_2008} \cite{correa_visibility-driven_2009} and volume saliency \cite{kim_saliency-guided_2006}, provide the basis for powerful feedback mechanisms in volume rendering.
In current volume rendering systems, appropriate transfer functions are often obtained by trial-and-error.
It is desirable to take advantage of these objective measures in order to automate the specification of transfer functions for emphasizing features of interest in volume visualization.

The main goal of our research is to investigate the optimization of visualization parameters (in particular transfer functions) with information derived from volume data based on feedback mechanisms from the volume rendering process.
We hypothesize that the importance of voxels (sample values in volume data) are associated with their information content. Therefore, the transfer functions of volume visualization can be optimized based on the information inherent within the data sets and user input which indicates the user's interest.
Furthermore, we hypothesize that combining automated optimization techniques with feedback mechanisms such as visibility and visual saliency can provide a more intuitive means for obtaining clear visualization of features of interest in volume data.

% (e.g. visibility and visual saliency in resulting images) 

%feasibility
%In all of the above, the issue of visibility is more a
%consequence of transfer function design than a design
%parameter. 

%In addition, we investigate the feasibility of propagating the optimization approach from static volume data to time-varying data.

%A small number of approaches have been proposed for using information theory in volume visualization. However, most available approaches are not designed for the visualization of time-varying data and thus they have not taken the coherence issues into account.

\section{Scope}
%This thesis focuses on methods for enhancing user understanding of volume data by optimizing visualization parameters and incorporating NPR techniques.
%I investigate the optimization of transfer functions by exploiting the information inherent within the volume data and input from user interaction.
%In addition, we investigate NPR techniques which we believe are well suited for highlighting important details as well as simplifying less important details. Hence, we combine the optimization of visualization parameters and NPR techniques in order to provide meaningful exploration of complex data.

The focus of this thesis is on methods for enhancing user understanding of features of interest in volume visualization by optimizing transfer functions based on information derived from volume data (e.g. entropy and saliency of voxels). In addition to the information inherent in the volume data, view-dependent information (e.g. visibility of voxels) obtained in the volume rendering process is also exploited in the optimization of transfer functions.

In this research, we focus on the visualization of volume data sets, particularly the scalar field data acquired from medical imaging (e.g. CT and MRI scans) and generated from flow simulations (e.g. computational fluid dynamics).

The features of interest in a volume data set are specified by user-defined transfer functions.
Therefore, manual segmentation by domain experts or computational expensive automatic segmentation techniques are not in the scope of this thesis.

%Flow data which are often in the form of vector fields are not under the focus of our research.
%Furthermore, we investigate techniques which are applicable to consumer level devices rather than expensive dedicated visualization hardware. As such, this constrains the amount of memory and processing power available, and therefore fast techniques with low memory requirement are necessary.

%\section{Research Question and Design \label{section_research_question}}
%\paragraph{Research Question}
%Visualization is concerned with the creation of images from data to enhance the user's ability to reason and understand properties related to the user's underlying problem.
%This research project focuses on methods to optimize important parameters of volume visualization (in particular transfer functions) and utilize NPR techniques in order to facilitate the user's exploration and understanding of the volume data. We address the following problem in volume visualization.
%
%\begin{itemize}
%\item How do we improve the understanding of volume data by optimization of visualization parameters and through the use of NPR techniques?
%	\begin{itemize}
%	\item We investigate techniques that will allow us to exploit the information inherent within the volume data to automatically and also semi-automatically (in user driven ways) optimise the transfer function in volume visualization. 
%%	Additionally we explore illustrative rendering strategies to enhance the perception of structures within the volume data.
%	
%	Additionally we explore illustrative rendering strategies to enhance the perception of structures within the volume data as well as depict the dynamic aspects of time-varying volume data.
%%	How do we exploit the information in the volume data to optimize the transfer function in volume visualization and enhance the perception of structures using illustrations?
%	\end{itemize}
%%\item How do we convey motion information of time-varying volume data through non-photorealistic techniques?
%%	\begin{itemize}
%%	\item How do we exploit illustrations and painterly stylization (such as stroke orientation, size and color) to express motion information (such as direction and speed)?
%%	\end{itemize}
%\end{itemize}
%
%%\paragraph{Sub-questions stem from the overall research questions}
%%
%%\begin{itemize}
%%\item How de we identify features of interest in volume data?
%%\item How do we extract a succinct representation from a time-step of time-varying volume data?
%%\\How do we design coherent transfer functions for time-varying volume rendering?
%%\item How do we track features through a series of time-steps in time-varying volume data?
%%\\How does clustering techniques help gain insights into time-varying volume data?
%%\item How do we enhance time-invariant features of volume data with illustrative techniques such as boundary enhancement and oriented feature enhancement?
%%\item How do we enhance temporal features of time-varying volume data with illustration-inspired techniques?
%%\item How do we render time-varying time-varying volume data with coherence?
%%\\How do we tackle the coherence issues in transfer function design and illustrative rendering for time-varying volume data?
%%\item How do we evaluate the effectiveness of our approaches?
%%\end{itemize}
%
%\paragraph{Research Design}
%In this thesis, we use entropy from information theory to measure the information content associated with sample values in volume data and establish importance measurements based on information within the volume data as well as input from user-driven techniques.
%
%NPR techniques are adopted in our approaches to visualize both static and time-varying data in order to enhance the expressiveness of the visualization. The combination of standard volume rendering techniques and NPR techniques provides us with the freedom to depict data content as abstraction according to its importance.
%
%The most fundamental objective of visualization is to enhance the user's ability to reason and understand properties related to the underlying problem. Therefore, we will conduct user studies that will measure the user's task performance and accuracy in order to evaluate the effectiveness of the proposed visualization approaches.

%\section{Methodology}
%\section{Scope}

\section{Contributions}
We present a transfer function refinement approach, which exploits the entropy of voxels derived from volume to equalize the opacity transfer function, in order to reduce general occlusion and improve the clarity of layers of structures in the resulting images.
%Moreover, the user can explore a volume by interactively specifying different priority intensity ranges and observe which layers of structures are revealed.
Moreover, this approach assists the user in exploring and enhancing features of interest by interactively specifying different priority intensity ranges.

In addition to view-independent information (i.e. entropy of voxels), we propose visibility-weighted saliency for measuring the view-dependent saliency of features of interest for volume visualization.
This metric aims to assist users in choosing suitable viewpoints and designing effective transfer functions to visualize the features of interest.

Subsequently, we describe an automated transfer function optimization method based on the visibility-weighted saliency metric. This method takes into account the perceptual importance of voxels and the visibility of features, and automatically adjusts the transfer function to match the target saliency levels specified by the user. In addition, a parallel line search strategy is presented to improve the performance of the optimization algorithm.

Finally, we develop a novel visualization approach which modulates focus, emphasizing important information, by adjusting saturation and brightness of voxels based on an importance measure derived from temporal and multivariate information.

\section{Summary of Chapters}
The rest of this thesis is structured as follows:

Chapter~\ref{related_work_chapter}
provides an overview of the background and related work in the field of volume visualization, with particular focus on the design and optimization of transfer functions.

Chapter~\ref{transfer_function_refinement}
presents a novel approach for transfer function refinement, which is an optimization of transfer functions based on the distribution (i.e. the histogram) of the volume data. This optimization also allows the user to prioritize specific regions by generating weightings for transfer function components based on user-selected regions.

Chapter~\ref{visibility-weighted_saliency}
describes visibility-weighted saliency as a measure of visual saliency of features in volume rendered images, in order to assist users in choosing suitable viewpoints and designing effective transfer functions to visualize the features of interest. Visibility-weighted saliency is based on a computational measure of perceptual importance of voxels and the visibility of features in volume rendered images.

Chapter~\ref{transfer_function_optimization}
provides a detailed description of an automated transfer function optimization approach based on the visibility-weighted saliency metric, which indicates the perceptual importance of voxels and the visibility of features in volume rendered images.

Chapter~\ref{selective_saturation_brightness}
outlines a novel visualization approach which modulates focus, emphasizing important information, by adjusting saturation and brightness of voxels based on an importance measure derived from temporal and multivariate information.

Chapter~\ref{conclusions}
summarizes our contributions and provides a discussion of possible avenues of future work.

%...
%
%	\section{Section 1.1}
%
%	...

%-------------------------------------------------------------------------
